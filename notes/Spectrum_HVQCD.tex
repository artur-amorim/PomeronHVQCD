\documentclass[10 pt]{article}

\usepackage{graphicx}
\usepackage[utf8]{inputenc}
\usepackage{amssymb}
\usepackage{amsfonts}
\usepackage{amsmath}
\usepackage[T1]{fontenc}
\usepackage[margin=0.5in]{geometry}


\title{Spectrum of HVQCD}
\author{Artur Amorim }
\date{}

\linespread{1.5}

\begin{document}
\maketitle

\section{How do we compute the spectrum}

In this section we review how we compute the spectrum of HVQCD in our fits.

\subsection{HVQCD action}

The theory in the bulk that we will consider in these notes is dual to QCD in the Veneziano limit. Its action can be divided in a gluon and flavour sector and in the Einstein frame takes the form
\begin{align}
&S = S_g + S_f, \\
&S_g =  M_p^3 N_c^2 \int d^5X \sqrt{g} \left[ R - \frac{4}{3} \frac{{\left( \partial \lambda \right)}^2}{\lambda^2}+ V_g \left( \lambda\right) \right], \\
&S_f = - x M_p^3 N_c^2 \int d^5 X V_f\left(\lambda, T\right) \sqrt{det\left(g_{ab} + k\left(\lambda\right) \partial_a \tau \partial_b \tau \right)},
\label{eq: HVQCD action}
\end{align}
where $M_p$ is the Planck mass, $N_c$ is the number of colours, $x = N_f / N_c$ with $N_f$ being the number of quark flavours and $\lambda = e^{\Phi}$ with $\Phi$ being the dilaton field. In the Veneziano limit $N_c \to \infty, \, N_f \to \infty$ with $x$ fixed. It is assumed $X = \left(z, x^\mu \right)$ and a metric of the form
\begin{equation}
\label{eq: metric definition}
ds^2 = e^{2 A} \left( d z^2 + \eta_{\mu \nu} dx^\mu dx^\nu \right),
\end{equation}
where $\eta_{\mu\nu}$ is the 4d Minkowski metric. Latin indices refer to general space-time indices and greek indices strictly mean boundary indices.

We have three background fields whose profiles will be found by solving the equations of motion resulting from the action (\ref{eq: HVQCD action}): the warp factor $A$, the dilaton $\Phi$ and the tachyon $\tau$. Because the metric is invariant under translation in $x^\mu$ the background fields will depend only on the radial coordinate $z$.

The action (\ref{eq: HVQCD action}) is not the most general one. The discussion that follows can be found, for example, in~\cite{Jarvinen:2011qe}. The reason we reproduce it here is because it will be useful to keep these ideas in mind when we derive the gravitational coupling between the graviton in this background and the $U(1)$ gauge field dual to the current operator $J^\mu$ in the boundary. The flavour action is motivated by considering bulk fields $T_{ij}$, $A^\mu_{L, i j}$ and $A^\mu_{R, i j}$ dual to the boundary operators $\bar{\psi}^i_R \psi^j_L$, $\bar{\psi}^i_L \sigma^\mu \psi^J_L$ and $\bar{\psi}^i_R \sigma^\mu \psi^J_R$, respectively. The field $T_{ij}$ transforms as $\left(N_f, \bar{N}_f \right)$ of the flavour symmetry $U\left(N_f\right)_R \times U \left(N_f\right)_L$ while the fields $A^\mu_{L, i j}$ and $A^\mu_{R, i j}$ transform in the adjoint representations of $U\left(N_f\right)_L$ and  $U \left(N_f\right)_R$ respectively. In string theory this is modelled by considering $N_f$ flavour branes (R) and $N_f$ flavour antibranes (L). In particular $A^\mu_R$ and $A^\mu_L$ are the lowest modes of open strings with both ends in the D branes and anti-D branes respectively. The $T_{ij}$ fields are the lowest modes of open strings with one end in a D brane and another end in the anti-D brane which corresponds to the tachyon. For our purposes such a system obeys a Dirac-Born-Infeld like action (let's ignore the WZ term for now)
\begin{align}
&S_{DBI} = - \frac{1}{2} M_p^3 N_c N_f \int dz d^4 x \, \bold{Str} \left[ V_f \left(\lambda, T^{\dagger} T\right) \sqrt{det\left( g_{ab} + k\left(\lambda\right)D_{(a}T^{\dagger}D_{b)}T + w\left(\lambda\right) F^{L}_{ab}\right)} + \right. \\ \notag
&\left. + V_f \left(\lambda, T T^{\dagger} \right) \sqrt{det\left( g_{ab} + k\left(\lambda\right)D_{(a}TD_{b)}T^{\dagger} + w\left(\lambda\right) F^{R}_{ab}\right)}  \right]
\label{eq: DBI action}
\end{align}
where $\bold{Str}$ is the symmetric trace over the gauge indices while the determinant is taken with respect to the Minkowski indices (since we are going to work up to quadratic order we can treat the symmetric trace as the trace with respect to the gauge indices). $T_{ij}$ is a complex $N_f \times N_f$ matrix and $A^{L,R}_a$ are the world-volume gauge fields of the $U(N_f)_L \times U(N_f)_R$ group. The covariant derivative terms are given by
\begin{equation}
D_a T = \partial_a T - i T A^L_a + i A^R_a T \, , \, D_a T^{\dagger} = \partial_a T^{\dagger} - i  A^L_a T^{\dagger}+ i T^{\dagger} A^R_a 
\end{equation}
and the field strengths by
\begin{align}
F^{L,R} = d A^{L,R} - i A^{L,R} \wedge A^{L,R}
\end{align}
In this work we are assuming that the light quark masses are equal and under this assumption the tachyon is just $T = \tau \mathbf{1}$. Furthermore for the QCD vacuum $A^R_a = 0 = A^L_a$. Using these conditions the action (\ref{eq: DBI action}) reduces to (\ref{eq: HVQCD action}).

\subsection{Potentials definitions}

To compute the profile of the background fields as well the spectrum of their fluctuations it is necessary to define the potentials $Vg$, $Vf$, $k$ and $w$ that appear in the previous actions:
\begin{align}
&V_g(\lambda) = 12 + V_1 \lambda + V_2 \frac{\lambda^2}{1 + \frac{sc \lambda}{\lambda_0}} + 3 e^{-\frac{\lambda_0}{sc \lambda}} V_gIR \frac{\lambda^{4/3}}{4 \pi^{8/3}}\sqrt{\log\left(1+\frac{sc \lambda}{\lambda_0}\right)}, \\ \notag
& V_1 = \frac{44}{9 \pi^2}, \, V_2 = \frac{4619}{3888 \pi^4}, \, \lambda_0 = 8 \pi^2 \\
& V_f(\lambda, \tau) = V_{f0} \left(\lambda\right) e^{-\tau^2}, \\ \notag
& V_{f0}\left(\lambda\right) = W_0 + \frac{24 + \left(11- 2 x\right) W_0}{27 \pi^2} \lambda + \frac{24 (857 - 46 x) + W_0 (4619 - 1714 x + 92 x^2)}{46656 \pi^4} \frac{\lambda^2}{1+ \frac{sc \lambda}{\lambda_0}} + \\ \notag
& + \frac{3 W_{IR}}{16 \pi^4} {\left(sc \lambda \right)}^2 e^{- \frac{\lambda_0}{sc \lambda}} \left( 1 + \frac{\lambda_0 W_1}{sc \lambda}\right) \\
& \frac{1}{k(\lambda)} =  \left( \frac{3}{2} - \frac{W_0 x}{8} \right) \left( 1+ \frac{sc k_{U1} \lambda}{\lambda_0} + k_{IR} e^{-\frac{\lambda_0}{ksc \lambda}} \left( 1 + \frac{\lambda_0 k_1}{ksc \lambda}\right) \frac{{\left(\frac{ksc \lambda}{\lambda_0}\right)}^{4/3}}{\sqrt{\log\left(1 + \frac{ksc \lambda}{\lambda_0}\right)}}\right), \\
& \frac{1}{w\left(\lambda\right)} = w_0 \left(1 + \frac{sc w_{U1} \lambda}{\lambda_0 \left( 1+ \frac{sc \lambda}{\lambda_0}\right)} + w_{IR} e^{- \frac{\lambda_0}{wsc \lambda}} \left(1+\frac{\lambda_0 w_1}{wsc \lambda}\right)\frac{{\left( \frac{wsc \lambda}{\lambda_0}\right)}^{4/3}}{\log\left(1+\frac{wsc \lambda}{\lambda_0}\right)}) \right)
\end{align}

Althogh $w\left(\lambda\right)$ will not be important for determination of the profile of $A$, $\Phi$ and $\tau$, it will be necessary to the evaluation of the spectrum of the mesons as it enters the expressions of the Schrodinger potentials.  Hence we need to provide the numerical values of the parameters $sc$, $ksc$, $wsc$, $W_0$, $w_0$, $k_{U1}$, $w_{U1}$, $V_{gIR}$, $W_{IR}$, $k_{IR}$,  $w_{IR}$, $W_1$, $k_1$ and $w_1$ to compute the spectrum. In our fits we will use $x = 2/3$ and as such we will look only at the mesons with quarks up and down in their composition. There is only one parameter to specify in order to compute numerically the spectrum. We will present it in the next section.

\subsection{Solving the Equatios of Motion}

In this section we give details on how the equations of motion are solved. It turns out to be convenient to write the resulting equations in terms of $A$ instead of the radial coordinate $z$. To do this one introduces the variable
\begin{equation}
q\left(A\right) = \frac{dz}{dA} e^{A};
\label{eq: q definition}
\end{equation}
From now on all the background fields will be functions of $A$ and the differential equations we present are all with respect to A being the independent variable.

\subsubsection{Yang-Mills Equations of Motion}

If we make $x = 0$ we get the action of the IHQCD model and hence we are dealing with a pure Yang-Mills theory in the large $N_c$ limit. The equations of motion are then
\begin{align}
	&\frac{dq}{dA} = \frac{1}{3} \left( 12 q - q^3 V_g \left(\Phi\right) \right) \\
	&\frac{d\Phi}{dA} = - \frac{\sqrt{3}}{2} \sqrt{12 - q^2 V_g\left(\Phi\right)}
	\label{eq: YM equations}
\end{align}
We will solve this coupled differential equations from the IR to the UV. We set $A_{IR} = -150$ and $A_{UVYM} = 50$ as the lower and upper bound on A. In the IR $A$ and $\Phi$ take the asymptotic form
\begin{align}
&A_{IR} \left(z\right) = \frac{23}{24} - \frac{173}{3456 z^2} - z^2 + 0.25 \log(6 z^2) - \frac{\log(Vg_{IR})}{2} \\
&\Phi_{IR} \left(z \right) = - \frac{23}{16} - \frac{151}{2304 z^2} + \frac{3}{2} z^2 - \log(\frac{sc}{\lambda_0})
\label{eq: YM fields IR asymptotics}
\end{align}
This allows us to determine $z_{IRYM}$ such that $A_{IR}\left(z_{IRYM}\right) = -150$ and use its numerical value to compute $\Phi\left(A_{IR}\right) = \Phi_{IR} \left( z_{IRYM}\right)$ and $q\left(A_{IR}\right) = e^{A_{IR}} / \frac{dA_{IR}}{d z} $ and hence defining boundary conditions to solve equations~\ref{eq: YM equations}. Note that $sc$ and $V_{gIR}$ determine uniquely both the initial conditions as well the evolution of the background fields in pure YM.

After we solve the YM equations we can determine $z\left(A\right)$ by using equation (\ref{eq: q definition}). We solve this numerically by taking $z(A_{IRYM})$ = $z_{IRYM}$ as initial condition and at the end we perform the shift $z\left(A\right) \to z\left(A\right) - z\left(A_{UVYM} \right) $ in order to have the UV singularity at $z = 0$. For the spectrum we will consider only  $A$ values that satisfy $z\left(A\right) > 10^{-6}$ and $\Phi\left(A\right) < 120$.

We finish this section by specifying which algorithms we have used to find the numerical YM background. To compute $z_{IRYM}$ we have used the root finding Van Wijngaarden-Dekker-Brent method described in~\cite{10.5555/1403886}. The differential equations (\ref{eq: YM equations}) were solved using $integrate\_const$ with a Runge-Kutta-Dormand-Prince stepper from the Boost $C++$ library.

\subsubsection{Solving HVQCD equations of motion}

The solution of HVQCD equations is done in four stages. In the first two we are deep in the IR and the tachyon is decoupled from the other background fields and $q\left(A\right)$ and $\Phi\left(A\right)$ obey equations (\ref{eq: YM equations}). In the third stage the tachyon will couple to the other background fields until it reaches the UV where it decouples again and we are in the fourth and last stage of the solution of the EOMs.

As in the YM case we start by first specifying the IR boundary conditions. We first define $A_{IR} = -150$, $A_{UVYM} = 50$, $A_{UVc} = 100$ and $A_{UVf} = 1000$.
Again  we compute $z_{IRYM}$ as in the YM case and we use it to compute $q\left(A_{IR}\right)$ and $\Phi\left(A_{IR}\right)$ as in YM and $\tau\left(A_{IR}\right)$ using the tachyon IR asymptotics
\begin{equation}
\tau_{IR} \left(z \right) = \tau_0 z^{\frac{(12 - x W_0) k_{IR}}{6 V_{gIR}}}
\end{equation}
where $\tau_0$ is a paremter that is gonna be fitted by the spectrum. This is the last parameter to be introduced.
The constants $\tau_{cut} = 1000$ and $V_{f cut} = 10^{-8}$ are defined and they signal the end of the first and second stages of the construction of the numerical background, respectively.

We start the construction of the background by computing the YM profile of $q\left(A\right)$ and $\Phi\left(A\right)$ from $A_{IR}$ to $A_{UVYM}$. If $\tau_{IR} > \tau_{cut}$ the tachyon profile will be given by the solution of the differential equation
\begin{equation}
	\frac{d\tau}{dA} = - 4 \frac{q^2 V_{f0}\left(\Phi\right) \tau}{8 V_{f0}\left( \Phi\right) k\left(\Phi\right) + 2 k\left(\Phi\right) \frac{d V_{f0}}{d\Phi} \frac{d\Phi}{dA} + V_{f0} \left(\Phi\right) \frac{dk}{d\Phi} \frac{d\Phi}{dA} }
\end{equation}
until $\tau < \tau_{cut}$, marking the end of the first stage. The value of $A$ such that this condition is met is called $A_{UV1}$ The values of $q$ and $\Phi$ used are the ones given by YM.

In the second stage we solve the tachyon differential equation
\begin{align}
&\frac{d^2\tau}{d A^2} = - 2 \frac{q^2 \tau}{k\left(\Phi\right)} - 4 \frac{d \tau}{dA} + \frac{d \log q}{dA} \frac{d\tau}{dA} - 2 \tau {\left(\frac{d\tau}{dA}\right)}^2 - 4 k\left(\Phi\right) \frac{{\left( \frac{d\tau}{dA}\right)}^3}{q^2} - \\ \notag
& - \frac{d \log V_{f0}}{d\Phi} \frac{d\tau}{dA} \frac{d\Phi}{dA} - \frac{d \log k}{d\Phi} \frac{d\tau}{dA} \frac{d\Phi}{dA} - k\left(\Phi\right) \frac{d \log V_{f0}}{d\Phi} {\left(\frac{d\tau}{dA}\right)}^3 \frac{\frac{d\Phi}{dA}}{q^2} - \frac{d k}{d\Phi} \frac{d\Phi}{dA} \frac{{\left(\frac{d\tau}{dA}\right)}^3}{2 q^2}
\label{eq: tau YM 2}
\end{align}
from the value of $A_{UV1}$ to $A_{UVYM}$. Using the profiles of $q$, $\Phi$ and $\tau$ we compute $A_{UV2}$ such that $A_{UV1} < A_{UV2} < A_{UVYM}$ and $V_{f0}\left(\Phi\right) e^{-\tau^2} = V_{fcut} V_g \left(\Phi\right)$ are satisfied. This condition marks when the tachyon starts to couple with $q$ and $\Phi$ from the IR to the UV.\footnote{Matti's code rejects such backgrounds while my code just takes them as valid backgrounds. This, I think, explains why Matti could not get the same results for the spectrum as me in that fit. This also means that in the numerical background I have obtained the tachyon never couples to $q$ and $\Phi$. }

In the stage where the tachyon is coupled the backgroud fields' dynamics obeys
\begin{align}
& \frac{dq}{dA} = \frac{4}{9} q {\left(\frac{d\Phi}{dA}\right)}^2 + q V_f \left(\Phi, \tau\right) k\left(\Phi\right) \frac{{\left(\frac{d\tau}{dA}\right)}^2}{6\sqrt{1+ k\left(\Phi\right) {\left(\frac{\frac{d\tau}{dA}}{q}\right)}^2}} \\
& \frac{d^2 \Phi}{d^2A} = - \frac{3}{8} q^2 \frac{d V_g}{d\Phi} + \frac{9}{\frac{d\Phi}{dA}} + \frac{3}{4} \frac{q^2}{\frac{d\Phi}{dA}} \left( \frac{V_f \left(\Phi, \tau\right)}{\sqrt{1+ k \frac{{\left(\frac{d\tau}{dA}\right)}^2}{q^2}}} - V_g \left(\Phi\right) \right) - \\ \notag
& - 5 \frac{d\Phi}{d A} + V_f \left(\Phi, \tau\right) k \left(\Phi\right) \frac{{\left(\frac{d\tau}{dA}\right)}^2 \frac{d\Phi}{dA}}{6 \sqrt{1+ k\left(\Phi\right) {\left(\frac{\frac{d\tau}{dA}}{q}\right)}^2 }} + \frac{4}{9} {\left(\frac{d\Phi}{dA}\right)}^3 + \\ \notag
& + \frac{3}{8} q^2 \frac{\frac{d V_f}{d\Phi}}{\sqrt{1+ k\left(\Phi\right) {\left(\frac{\frac{d\tau}{dA}}{q}\right)}^2}} + \frac{3}{8} k\left(\Phi\right) {\left( \frac{d\tau}{dA}\right)}^2 \frac{\frac{dV_f}{d\Phi}}{\sqrt{1+ k\left(\Phi\right) {\left(\frac{\frac{d\tau}{dA}}{q}\right)}^2}} + \\ \notag
& + \frac{3}{16} V_f\left(\Phi, \tau\right) {\left( \frac{d\tau}{dA}\right)}^2 \frac{\frac{dk}{d \Phi}}{\sqrt{1+ k\left(\Phi\right) {\left(\frac{\frac{d\tau}{dA}}{q}\right)}^2}} \\ \notag
& \frac{d^2 \tau}{d A^2} = -2 q^2 \frac{\tau}{k\left(\Phi\right)} - 4 \frac{d\tau}{d A} -2 \tau {\left(\frac{d\tau}{dA}\right)}^2 - 4\frac{k\left(\Phi\right)}{q^2} {\left(\frac{d\tau}{dA}\right)}^3 + k\left(\Phi\right) V_f \left(\Phi, \tau \right) \frac{{\left(\frac{d\tau}{dA}\right)}^3}{6 \sqrt{1+ k\left(\Phi\right) {\left(\frac{\frac{d\tau}{dA}}{q}\right)}^2}} - \\
& - \frac{d \log(V_{f0})}{dA} \frac{d\Phi}{dA} \frac{d\tau}{dA} - k \left(\Phi\right) \frac{d\log(V_{f0}}{dA} \frac{d\Phi}{dA}\frac{{\left(\frac{d\tau}{d A}\right)}^3}{q^2} + \frac{4}{9} {\left(\frac{d\Phi}{dA}\right)}^2 \frac{d\tau}{dA} - \frac{d \log(k)}{d\Phi} \frac{d\Phi}{dA} \frac{d\tau}{dA} - \\ \notag
& - \frac{1}{2} \frac{d k}{d\Phi} \frac{d\Phi}{dA} \frac{{\left( \frac{d\tau}{dA}\right)}^3}{q^2}
\end{align}
This is a stiff system of differential equations and for this reason we had used the function integrate\_adaptive with the rosenbrock4 stepper from Boost. This system is solved from $A_{UV2}$ to $A_{UCc}$. Because $A_{UVc} = 50$ we believe that at this point we are in the UV and here the tachyon decouples again from $q$ and $\Phi$.

The UV equations of motion are
\begin{align}
&\frac{dq}{dA} =  \frac{4}{9} q {\left(\frac{\frac{d\lambda}{dA}}{\lambda}\right)}^2 \\
&\frac{d^2\lambda}{dA^2} = -\frac{3}{8}  {\left(q \lambda \right)}^2 \frac{d V_g}{d\lambda} + 9 \frac{\lambda^2}{\frac{d\lambda}{dA}} + \frac{3}{4} \frac{{\left(q \lambda \right)}^2}{\frac{d\lambda}{dA}} \left( V_f\left(\lambda, 0\right) - V_g \left(\lambda\right) \right) - 5 \frac{d\lambda}{dA}  + \\
& + \frac{d \log q}{dA} \frac{d\lambda}{d A} + \frac{{\left( \frac{d\lambda}{dA}\right)}^2}{\lambda} + \frac{3}{8} {\left( q \lambda \right)}^2 \frac{d V_f}{d\lambda} \left(\lambda, 0 \right) \\ \notag
& \frac{d^2 \tau_n}{dA^2} = 3 \tau_n - 2 \frac{q^2 \tau_n}{k\left(\lambda\right)} - \tau_n \frac{d \log q}{dA} + \tau_n \frac{d \log V_{f0}}{d\lambda} \frac{d\lambda}{dA} - 2 \frac{d\tau_n}{d A} + \frac{d\log q}{d A} \frac{d \tau_n}{dA} - \\
& - \frac{d \log V_{f0}}{d\lambda} \frac{d\lambda}{dA} \frac{d\tau_n}{dA} + \tau_n \frac{d\lambda}{dA} \frac{d \log k}{d\lambda} - \frac{d\lambda}{dA} \frac{d\tau_n}{dA} \frac{d \log k}{d\lambda}
\end{align}
where $\tau_n = e^A \tau$. This system is solved from $A_{UVc}$ to $A_{UVf}$. With the profile of $\tau_n$ and $\lambda$ in this region we can compute the quark mass $m_q$ through the formula
\begin{align}
&m_q = \frac{\lambda \left(A_{UVf} - 10\right) \tau_n\left(A_{UVf}\right)e^{- \log l_{UV} - \tau_{corr}\left(\lambda\left(A_{UVf}\right)\right)} - \lambda \left(A_{UVf} \right) \tau_n\left(A_{UVf} - 10\right)e^{- \log l_{UV} - \tau_{corr}\left(\lambda\left(A_{UVf} - 10\right)\right)} }{\lambda\left(A_{UVf} - 10\right) - \lambda\left(A_{UVf}\right)}, \\
& \tau_{corr} = \frac{\left(-88 + 16 x + 27 sc k_{U1}\right) \log\left( \frac{24 \pi^2}{\left(11 - 2 x\right) \lambda}\right)}{12 x - 66}
\end{align}

As in the YM case we can determine $z\left(A\right)$ by using equation (\ref{eq: q definition}) and perform the shift $z\left(A\right) \to z\left(A\right) - z\left(A_{UVYM} \right) $ in order to have the UV singularity at $z = 0$. For the spectrum we will consider only  $A$ values that satisfy $z\left(A\right) > 10^{-6}$ and $\Phi\left(A\right) < 120$.











\section{Spectrum computation}
We will simply use the formulas given in~\cite{Arean:2013tja} to solve the Schrodinger problem associated with each kind of fluctuation. A detail in my C++ code is worth mentioning. Basically I expand the $\Xi$ expressions in terms of the background fields and potentials as well their derivatives. This is done in Mathematica. The resulting expression in Mathematica is then coded in C++ in order to compute the Schrodinger potential for each kind of fluctuation.



\section{Fits}

In the last sections we discussed how to compute the numerical backgrounds of a bulk theory dual to QCD in the Veneziano limit. We also saw how these backgrounds depend on a given set of parameters. Here we describe how those parameters can be fixed by using ratios of masses of particles that have been detected experimentally or in lattice simulations of QCD. The fluctuations of our backgrounds fields can be classified in flavour singlet and non-singlet fluctuations. We can also divide the different  fluctuations by using the quantum numbers $J^{PC}$.  Each of this classes of fluctuations will be identified with a group of mesons or glueballs. To simplify this identification we will consider only mesons with up and down quarks in their quark content. This means that in our fits $x = 2 / 3$.

Our background fields are determined by 15 parameters: sc, ksc, wsc, $W_0$, $w_0$, $k_{U1}$, $w_{U1}$, $V_{gIR}$, $W_{IR}$, $k_{IR}$, $w_{IR}$, $W_1$, $k_1$, $w_1$ and $\tau_0$. We are then in presence of non-trivial problem of finding the best set of parameters that describes the spectrum of observed mesons and glueballs. To simplify a bit the search for such set we can start by considering the spectrum of pure YM ($x = 0$) and fit sc and $V_{gIR}$ to the ratios $m_{0^{++}*} / m_{0^{++}}$ and $m_{2^{++}}/m_{0^{++}}$. If we do that we find that $sc = 2.50485$ and $V_{gIR} = 3.47852$ are the best set of values. Having said this we will perform fits with and without sc and $V_{gIR}$ fixed.

For the tensor glueball $2^{++}$ we will use the pure YM value of 2.150 GeV. The flavour non-singlet mesons and flavour singlet mesons used in the fits are displayed in table~\ref{table: ns mesons} and table~\ref{table: s mesons}, respectively.

\begin{table}
\centering
\begin{tabular}{ | c | c | c | }

\hline
$J^{PC}$ & Meson & Mass Measured (GeV) \\
\hline
$1^{--}$ & $\rho$ & 0.7755 \\
\hline
$1^{--}$ &  $\rho(1450)$ & 1.465 \\
\hline
$1^{--}$ & $\rho(1700)$ & 1.720 \\
\hline
$1^{--}$ & $\rho(1900)$ & 1.909 \\
\hline
$1^{--}$ & $\rho(2150)$ & 2.149 \\
\hline
$1^{--}$ & $\rho(2270)$ & 2.265\\
\hline
$1^{++}$ & $a_1(1260)$ & 1.230\\
\hline
$1^{++}$ & $a_1(1640)$ & 1.647 \\
\hline
$1^{++}$ & $a_1(1930)$ & 1.930 \\
\hline
$1^{++}$ & $a_1(2096)$ & 2.096 \\
\hline
$1^{++}$ & $a_1(2270)$ & 2.270 \\
\hline
$0^{++}$ &  $a_0(1450)$ & 1.474 \\
\hline
$0^{++}$ &  $a_0(2020)$ & 2.025 \\
\hline
$0^{-+}$ &  $\pi_0$ & 0.135 \\
\hline
$0^{-+}$ &  $\pi_0(1300)$ & 1.300 \\
\hline
$0^{-+}$ &  $\pi_0(1800)$ & 1.812 \\
\hline
$0^{-+}$ &  $\pi_0(2070)$ & 2.070 \\
\hline
$0^{-+}$ &  $\pi_0(2360)$ & 2.360 \\
\hline
\end{tabular}
\caption{Non-singlet mesons used in the fits.}
\label{table: ns mesons} 
\end{table}

\begin{table}
\centering
\begin{tabular}{ | c | c | c | }

\hline
$J^{PC}$ & Meson & Mass Measured (GeV) \\
\hline
$1^{--}$ & $\omega(782)$ & 0.78265 \\
\hline
$1^{--}$ &  $\omega(1420)$ & 1.420\\
\hline
$1^{--}$ & $\omega(1650)$ & 1.670 \\
\hline
$1^{++}$ & $f_1(1285)$ & 1.2819\\
\hline
$1^{++}$ & $f_1(1420)$ & 1.4264 \\
\hline
\end{tabular}
\caption{Singlet mesons used in the fits.}
\label{table: s mesons} 
\end{table}


The cost function that we use for the fits is
\begin{align}
J = \sum_{i} \frac{|R_{pred_i} - R_{obs_i}|}{R_{obs_i}} + \lambda e^{- \left(\frac{12-x W_0)k_{IR}}{6 V_{gIR}} - 1\right)} + \lambda e^{-\left( \tau_M^2 - 3.5 \right)},
\end{align}
where $\tau_M^2$ is the square of the tachyon mass in the IR. Here $R_{obs_i}$ is the experimental observed ratio between one of the particles in the tables~\ref{table: ns mesons} and~\ref{table: s mesons} and the $\rho$ meson and $R_{pred_i}$ is the same ratio predicted by our model. The exponential terms are included in order to impose the constraints $\frac{\left(12-x W_0\right)k_{IR}}{6 V_{gIR}} > 1$ and $\tau_M^2 > 3.5$. The results presented were obtained with $\lambda = 0.5$ but I will redo the fits with $\lambda = 0.1$ or even smaller values. Maybe this way we can find a set of parameters that satisfy the constraints and with better ratio values.

\subsection{Global fit}

In the global fit we use the tensor glueball  and all the mesons in tables~\ref{table: ns mesons} and~\ref{table: s mesons} while varying all parameters. The best fit parameters are found in table~\ref{table: global fit parameters} and the ratios are shown in figure~\ref{figure: global fit ratios}.

\begin{table}
\centering
\begin{tabular}{ | c | c | c | c |}
\hline
sc & 3.25546 & ksc & 4.12754 \\
\hline
wsc & 1.31668& $W_0$ & 2.868 \\
\hline
$w_0$ & 1.8009 & $k_{U1}$ & 0.533949 \\
\hline
$w_{U1}$ & -0.766744 & $V_{gIR}$ & 1.54926 \\
\hline
$W_{IR}$ & 0.502304 & $k_{IR}$ & 2.8191 \\
\hline
$w_{IR}$ & 7.74607 & $W_1$ & 0.986571 \\
\hline
$k_1$ & -1.36173 & $w_1$ & 0.815957 \\
\hline
x & $2/3$ & $\tau_0$ & 0.457303\\
\hline
$Z_a$ & -0.513725 & $c_a$ & $4.58782 \times 10^{-9} $\\
\hline
\end{tabular}
\caption{Best fit parameters for global fit.}
\label{table: global fit parameters} 
\end{table}

\begin{figure}
  \center
  \includegraphics[scale = 0.5]{images/HVQCDGlobal_WithSAVM.png} 
  \caption{Ratios for the best fit parameters for global fit.}
  \label{figure: global fit ratios}
\end{figure}

\subsection{Global fit with sc and VgIR fixed}

In this fit we also use the tensor glueball  and all the mesons in tables~\ref{table: ns mesons} and~\ref{table: s mesons} while varying all parameters except sc and $V_{gIR}$. The best fit parameters are found in table~\ref{table: global fit parameters sc VgIR fixed} and the ratios are shown in figure~\ref{figure: global fit ratios sc VgIR fixed}.

\begin{table}
\centering
\begin{tabular}{ | c | c | c | c |}
\hline
sc & 2.50485 & ksc & 2.80494 \\
\hline
wsc & 4.75507 & $W_0$ & 2.04542 \\
\hline
$w_0$ & 0.635567 & $k_{U1}$ & 0.898244 \\
\hline
$w_{U1}$ & -4.79273 & $V_{gIR}$ & 3.47852 \\
\hline
$W_{IR}$ & 2.10562 & $k_{IR}$ & 2.81353 \\
\hline
$w_{IR}$ & 6.93792 & $W_1$ & 1.90077 \\
\hline
$k_1$ & -0.0484519 & $w_1$ & 0.767193 \\
\hline
x & 2/3 & $\tau_0$ & 0.351475\\
\hline
$Z_a$ & 0.187512 & $c_a$ & $1.13332 \times 10^{-13}$ \\
\hline
\end{tabular}
\caption{Best fit parameters for global fit with sc and VgIR fixed.}
\label{table: global fit parameters sc VgIR fixed} 
\end{table}

\begin{figure}
  \center
  \includegraphics[scale = 0.5]{images/HVQCDGlobal_sc_VgIR_fixed_WithSAVM.png} 
  \caption{Ratios for the best fit parameters for global fit with sc and VgIR fixed.}
  \label{figure: global fit ratios sc VgIR fixed}
\end{figure}

\subsection{Fit Without Glueballs but with sc and VgIR fixed}

In this fit we also use all the mesons in tables~\ref{table: ns mesons} and~\ref{table: s mesons} while varying all parameters except sc and $V_{gIR}$. The best fit parameters are found in table~\ref{table: fit without glueball sc VgIR fixed parameters} and the ratios are shown in figure~\ref{figure: fit without glueball sc VgIR fixed}.

\begin{table}
\centering
\begin{tabular}{ | c | c | c | c |}
\hline
sc & 2.50485 & ksc & 3.04602 \\
\hline
wsc & 1.93123 & $W_0$ & 2.18834 \\
\hline
$w_0$ & 2.33377 & $k_{U1}$ & 0.463568 \\
\hline
$w_{U1}$ & -2.73495 & $V_{gIR}$ & 3.47852 \\
\hline
$W_{IR}$ & 2.83069 & $k_{IR}$ & 2.53127 \\
\hline
$w_{IR}$ & 4.94608 & $W_1$ & 1.15435\\
\hline
$k_1$ & 0.151849 & $w_1$ & 0.0495368\\
\hline
x & 2/3 & $\tau_0$ & 0.519466\\
\hline
$Z_a$ & 0.107029 & $c_a$ & $8.72313 \times 10^{-13}$\\
\hline
\end{tabular}
\caption{Best fit parameters for fit without glueball but with sc and VgIR fixed.}
\label{table: fit without glueball sc VgIR fixed parameters} 
\end{table}

\begin{figure}
  \center
  \includegraphics[scale = 0.5]{images/HVQCDWithoutGlueball_sc_VgIR_fixed_WithSAVM.png} 
  \caption{Ratios for the best fit parameters for fit without glueball but with sc and VgIR fixed.}
  \label{figure: fit without glueball sc VgIR fixed}
\end{figure}

\subsection{Fit Without Glueballs}

In this fit we also use all the mesons in tables~\ref{table: ns mesons} and~\ref{table: s mesons} while varying all parameters. The best fit parameters are found in table~\ref{table: fit without glueball parameters} and the ratios are shown in figure~\ref{figure: fit without glueball}.

\begin{table}
\centering
\begin{tabular}{ | c | c | c | c |}
\hline
sc & 3.28916 & ksc & 3.17356 \\
\hline
wsc & 1.64997 & $W_0$ & 4.03026 \\
\hline
$w_0$ & -0.442523 & $k_{U1}$ & 2.50302 \\
\hline
$w_{U1}$ & 0.0457434 & $V_{gIR}$ & 3.12627 \\
\hline
$W_{IR}$ & 1.61772 & $k_{IR}$ & 1.30104 \\
\hline
$w_{IR}$ & 5.95641 & $W_1$ & 2.464 \\
\hline
$k_1$ & -1.42264 & $w_1$ & 0.0215877 \\
\hline
x & 2/3 & $\tau_0$ & 0.108259\\
\hline
$Z_a$ & 1.9721 & $c_a$ & 0.000100825\\
\hline
\end{tabular}
\caption{Best fit parameters for fit without glueball.}
\label{table: fit without glueball parameters} 
\end{table}

\begin{figure}
  \center
  \includegraphics[scale = 0.5]{images/fitHVQCDWithoutGlueball_withSAVM.png} 
  \caption{Ratios for the best fit parameters for fit without glueball.}
  \label{figure: fit without glueball}
\end{figure}


\subsection{Fit Without Scalars}

In this fit we also use the tensor glueball and the mesons in tables~\ref{table: ns mesons} and~\ref{table: s mesons} except the scalars and pseudoscalar mesons. We vary all parameters. The best fit parameters are found in table~\ref{table: fit without scalars parameters} and the ratios are shown in figure~\ref{figure: fit without scalars}.

\begin{table}
\centering
\begin{tabular}{ | c | c | c | c |}
\hline
sc & 2.75988 & ksc & 3.16245 \\
\hline
wsc &1.86214 & $W_0$ & 2.67223 \\
\hline
$w_0$ & 1.2902 & $k_{U1}$ & 1.20368 \\
\hline
$w_{U1}$ & 0.28155 & $V_{gIR}$ & 1.40383 \\
\hline
$W_{IR}$ & 1.06798 & $k_{IR}$ & 2.44736 \\
\hline
$w_{IR}$ & 5.30943 & $W_1$ & 0.392724 \\
\hline
$k_1$ & 0.305196 & $w_1$ & -0.47342 \\
\hline
x & 2/3 & $\tau_0$ & 0.424401\\
\hline
$Z_a$ & -1.21038 & $c_a$ & $1.11625 \times 10^{-7}$ \\
\hline
\end{tabular}
\caption{Best fit parameters for fit without scalars.}
\label{table: fit without scalars parameters} 
\end{table}

\begin{figure}
  \center
  \includegraphics[scale = 0.5]{images/HVQCDWithoutScalars_WithSAVM.png} 
  \caption{Ratios for the best fit parameters for fit without scalars.}
  \label{figure: fit without scalars}
\end{figure}


\subsection{Fit Without Scalars but sc and VgIR fixed}

In this fit we also use the tensor glueball and the mesons in tables~\ref{table: ns mesons} and~\ref{table: s mesons} except the scalars and pseudoscalar mesons. We vary all parameters except sc and $V_{gIR}$. The best fit parameters are found in table~\ref{table: fit without scalars sc VgIR fixed} and the ratios are shown in figure~\ref{figure: fit without scalars sc VgIR fixed}.

\begin{table}
\centering
\begin{tabular}{ | c | c | c | c |}
\hline
sc & 2.50485 & ksc & 1.29058 \\
\hline
wsc & 0.0603466 & $W_0$ & 0.0110914 \\
\hline
$w_0$ & 2.57011 & $k_{U1}$ & 2.66727  \\
\hline
$w_{U1}$ & -1.71986 & $V_{gIR}$ & 3.47852 \\
\hline
$W_{IR}$ & 1.73027 & $k_{IR}$ & 4.26511 \\
\hline
$w_{IR}$ & 4.72176 & $W_1$ & 2.65188 \\
\hline
$k_1$ & 0.572573 & $w_1$ & 0.580926 \\
\hline
x & 2/3& $\tau_0$ & 0.574017\\
\hline
$Z_a$ & -0.0790245 & $c_a$ & 0.000172359 \\
\hline
\end{tabular}
\caption{Best fit parameters for fit without scalars but with sc and VgIR fixed.}
\label{table: fit without scalars sc VgIR fixed} 
\end{table}

\begin{figure}
  \center
  \includegraphics[scale = 0.5]{images/HVQCDWithoutScalars_sc_VgIR_fixed_WithSAVM.png} 
  \caption{Ratios for the best fit parameters for fit without scalars but with sc and VgIR fixed.}
  \label{figure: fit without scalars sc VgIR fixed}
\end{figure}


\subsection{Fit Without Glueball and Scalars but with sc and VgIR fixed}

In this fit we don't use the tensor glueball and the scalar and pseudoscalar mesons. We vary all parameters except sc and $V_{gIR}$. The best fit parameters are found in table~\ref{table: fit without glueball scalars sc VgIR fixed} and the ratios are shown in figure~\ref{figure: fit without glueball scalars sc VgIR fixed}.

\begin{table}
\centering
\begin{tabular}{ | c | c | c | c |}
\hline
sc & 2.50485 & ksc & 4.28812 \\
\hline
wsc & 5.10253 & $W_0$ & 1.24218 \\
\hline
$w_0$ & 2.02123 & $k_{U1}$ & -2.80363  \\
\hline
$w_{U1}$ & 4.11302 & $V_{gIR}$ & 3.47852 \\
\hline
$W_{IR}$ & 1.65571 & $k_{IR}$ & 12.5991 \\
\hline
$w_{IR}$ & 4.04552 & $W_1$ & -0.377037 \\
\hline
$k_1$ & -7.68695 & $w_1$ & - 0.674825 \\
\hline
x & 2/3 & $\tau_0$ & $2.2559 \times 10^{-7}$\\
\hline
$Z_a$ & 0.0500001 & $c_a$ & $8.9016 \times 10^{-22}$ \\
\hline
\end{tabular}
\caption{Best fit parameters for fit without glueball and scalars but with sc and VgIR fixed.}
\label{table: fit without glueball scalars sc VgIR fixed} 
\end{table}



\begin{figure}
  \center
  \includegraphics[scale = 0.5]{images/HVQCDWithoutGlueballsScalars_sc_VgIR_fixed_WithSAVM.png} 
  \caption{Ratios for the best fit parameters for fit without glueball and scalars but with sc and VgIR fixed.}
   \label{figure: fit without glueball scalars sc VgIR fixed}
\end{figure}


\subsection{Fit Without Glueball and Scalars}

In this fit we don't use the tensor glueball and the scalar and pseudoscalar mesons. We vary all parameters. The best fit parameters are found in table~\ref{table: fit without glueball scalars} and the ratios are shown in figure~\ref{figure: fit without glueball scalars}.

\begin{table}
\centering
\begin{tabular}{ | c | c | c | c |}
\hline
sc & 2.97492 & ksc & 2.99657 \\
\hline
wsc & 1.43836 & $W_0$ & 3.94432 \\
\hline
$w_0$ & 0.681841 & $k_{U1}$ & 0.983282  \\
\hline
$w_{U1}$ & 0.687317 & $V_{gIR}$ & 2.85167 \\
\hline
$W_{IR}$ & 1.58022 & $k_{IR}$ & 1.64482 \\
\hline
$w_{IR}$ & 5.28076 & $W_1$ & 2.40908 \\
\hline
$k_1$ & -2.50085 & $w_1$ & - 0.410125 \\
\hline
x & 2/3 & $\tau_0$ & 1.03513 \\
\hline
$Z_a$ & 1.86124 & $c_a$ & $6.22234 \times 10^{-07}$ \\
\hline
\end{tabular}
\caption{Best fit parameters for fit without glueball and scalars.}
\label{table: fit without glueball scalars} 
\end{table}



\begin{figure}
  \center
  \includegraphics[scale = 0.5]{images/fitHVQCDWithoutGlueballScalars_with_SAVM.png} 
  \caption{Ratios for the best fit parameters for fit without glueball and scalars.}
   \label{figure: fit without glueball scalars}
\end{figure}


\section{Spin J field EOM}

In this section we will propose a phenomenological equation of motion for the spin J fields. First we will derive the equation of motion of the graviton in the Einstein frame which, as we shall see, is the same as in the Improved Holographic QCD model. This result is consistent with equation (A.108) from 1309.2286 which is the same result found in IHQCD.

To find the equation of motion of the graviton we ned to perturb the actions (\ref{eq: HVQCD action}) with respect to the metric. The $S_g$ term contributes with
\begin{align}
&\delta S_g = M^3 N_c^2 \int d^5X \sqrt{-g} \left[ R_{ab} - \frac{1}{2} g_{ab} \left( R -\frac{4}{3} \nabla_a \Phi \nabla^a \Phi + V_g \left(\lambda\right) \right)- \frac{4}{3} \nabla_a \Phi \nabla_b \Phi \right],  \notag \\
& \delta S_f = - M^3 N_c N_f \int d^5 X V_f \left( \lambda, \tau \right) \delta \sqrt{- \text{det} \left( g_{MN} + k\left(\lambda\right) \partial_M \tau \partial_N \tau \right)}.
\end{align}
To make further progress we define $g_{eff. MN} = g_{MN} +  k\left(\lambda\right) \partial_M \tau \partial_N \tau $ and its matrix inverse $g_{eff.}^{MA}g_{eff. AN} = \delta^M_N$. The following identity will be useful
\begin{align}
\left( g_{eff}^{AB} + \delta  g_{eff}^{AB}\right) \left( g_{eff. BC} + \delta g_{BC}\right) = \delta^A_C \Leftrightarrow \delta g_{eff.}^{AD} = - g_{eff}^{AB} g_{eff.}^{DC} \delta g_{BC}.
\end{align}
From Jacobi's formula it follows
\begin{align}
\delta \sqrt{-\text{det} \left(g_{eff. MN}\right)} = -\frac{1}{2} \sqrt{-\text{det} \left(g_{eff. MN}\right)} g_{eff. MN} \delta g_{eff.}^{MN} = \frac{1}{2} \sqrt{-\text{det} \left(g_{eff. MN}\right)} g_{eff.}^{AB} \delta g_{AB}
\end{align}
Furthermore $\delta g_{AB} = - g_{AC} g_{BD} \delta g^{CD}$ and the contribution from $S_f$ is
\begin{align}
\delta S_f = \frac{1}{2} M^3 N_c N_f \int d^5X V_f\left(\lambda, \tau\right) \sqrt{-\text{det} g_{eff. MN}} g_{eff.}^{AB}g_{AC}g_{BD} \delta g^{CD}.
\end{align}

The Einstein equations are then
\begin{align}
R_{CD} - \frac{1}{2} g_{CD} \left( R - \frac{4}{3} {\left( \nabla \Phi \right)}^2 + V_g \left(\lambda\right) \right) - \frac{4}{3} \nabla_C \Phi \nabla_D \Phi + \frac{1}{2} x_f V_f \frac{\sqrt{-g_{eff.}}}{\sqrt{-g}} g_{eff.}^{AB} g_{AC} g_{BD} = 0.
\end{align}
Taking the trace of this equation we find that the Ricci scalar is given by
\begin{equation}
R = \frac{4}{3} {\left( \nabla \Phi \right)}^2 - \frac{5}{3} V_g \left( \lambda \right) + \frac{1}{3} x V_f \frac{\sqrt{-g_{eff.}}}{\sqrt{-g}} g_{eff.}^{AB} g_{AB},
\end{equation}
and substituting it in the Einstein equations we have
\begin{align}
R_{CD} - \frac{1}{2} g_{CD} \left(-\frac{2}{3} V_g \left(\lambda\right) + \frac{1}{3} x V_f \frac{\sqrt{-g_{eff.}}}{\sqrt{-g}} g_{eff.}^{AB} g_{AB} \right) - \frac{4}{3} \nabla_C \Phi \nabla_D \Phi + \frac{1}{2} x_f V_f \frac{\sqrt{-g_{eff.}}}{\sqrt{-g}} g_{eff.}^{AB} g_{AC} g_{BD} = 0.
\end{align}
Moreover $g_{eff.} = g + k\left( \lambda \right) d \tau \otimes d \tau$ meaning that $\sqrt{-g_{eff.}} = \sqrt{-g} \sqrt{\text{det} \left( 1+ k \left( \lambda \right) g^{-1} d \tau \otimes d \tau \right)}$ resultin in equation
\begin{align}
&R_{CD} - \frac{1}{2} g_{CD} \left( - \frac{2}{3} V_g \left( \lambda \right)  + \frac{1}{3} x V_f \sqrt{\text{det} \left( 1+ k \left( \lambda \right) g^{-1} d \tau \otimes d \tau \right)} g_{eff.}^{AB} g_{AB} \right) - \frac{4}{3} \nabla_C \Phi \nabla_D \Phi + \notag \\\ 
&+ \frac{1}{2} x V_f \sqrt{\text{det} \left( 1+ k \left( \lambda \right) g^{-1} d \tau \otimes d \tau \right)} g_{eff.}^{AB} g_{AC} g_{BD} = 0
\end{align}

The linearized Einstein equations are
\begin{align}
& \delta R_{CD} - \frac{1}{2} h_{CD} \left( - \frac{2}{3} V_g \left(\lambda\right) + \frac{1}{3} x V_f \sqrt{1 + k\left(\lambda\right) e^{-2A} \dot{\tau}^2} g_{eff.}^{AB} g_{AB} \right) - \frac{1}{2} g_{CD} \frac{1}{3} x V_f \times \notag \\
& \times \left( \delta  \sqrt{1 + k\left(\lambda\right) g^{-1} d \tau \otimes d \tau} g_{eff.}^{AB}g_{AB} +  \sqrt{1 + k\left(\lambda\right) e^{-2A} \dot{\tau}^2} \delta g_{eff.}^{AB} g_{AB} +  \sqrt{1 + k\left(\lambda\right) e^{-2A} \dot{\tau}^2} g^{AB}_{eff.} h_{AB} \right) + \notag \\
& + \frac{1}{2} x V_f \left(  \delta  \sqrt{1 + k\left(\lambda\right) g^{-1} d \tau \otimes d \tau} g_{eff.}^{AB} g_{AC} g_{BD} +  \sqrt{1 + k\left(\lambda\right) e^{-2A} \dot{\tau}^2} \delta g_{eff.}^{AB} g_{AC} g_{BD} + \right. \notag \\
&\left. +  \sqrt{1 + k\left(\lambda\right) e^{-2A} \dot{\tau}^2} g_{eff.}^{AB} h_{AC} g_{BD} +  \sqrt{1 + k\left(\lambda\right) e^{-2A} \dot{\tau}^2} g_{eff.}^{AB} g_{AC} h_{BD}\right) = 0.
\end{align}
In order to make progress we first compute
\begin{align}
&g_{eff.}^{AB} g_{AB} = 4 + \frac{1}{1 + k\left( \lambda \right) e^{-2A} {\dot{\tau}}^2} \\
& \delta g_{eff}^{AB} = - g_{eff}^{AM} g_{eff.}^{BN} h_{MN} \\
& \delta  \sqrt{1 + k\left(\lambda\right) g^{-1} d \tau \otimes d \tau} = - \frac{1}{2} \frac{k\left( \lambda \right) {\dot{\tau}}^2}{\sqrt{1+e^{-2A} k\left(\lambda\right) {\dot{\tau}}^2}} h^{zz} \\
& g_{eff}^{AB}h_{AC}g_{BD} + g_{eff.}^{AB} g_{AC} h_{BD} = 2 g_{eff.}^{AB} h_{A\left(C\right.}g_{\left.D\right)B}
\end{align}
and define $G = \sqrt{1+e^{-2A} k\left(\lambda\right) {\dot{\tau}}^2}$. The linearized Einstein equations are then
\begin{align}
&\delta R_{CD} - \frac{1}{2} h_{CD} \left[ -\frac{2}{3} V_g + \frac{1}{3} x V_f G \left(4 + \frac{1}{G^2} \right)\right] - \frac{1}{6} g_{CD} x V_f \left[ -\frac{1}{2} \frac{k\left(\lambda\right) {\dot{\tau}}^2}{G^2} h^{zz} \left(4 + \frac{1}{G^2}\right) - \right. \notag \\
& \left. - G g_{eff}^{AM} g_{eff}^{BN} g_{AB} h_{MN} + G g_{eff}^{AB} h_{AB}  \right]  + \frac{1}{2} x V_f \left[ - \frac{1}{2} \frac{k\left(\lambda\right) {\dot{\tau}}^2}{G} h^{zz} g_{eff}^{AB} g_{AC} g_{BD}  - \right. \notag \\
&\left. - G g_{eff}^{AM} g_{eff}^{BN} h_{MN} g_{AC} g_{BD} + 2 G g_{eff}^{AB} h_{A(C}g_{D)B} \right] = 0
\end{align}
The perturbed Ricci tensor is given by
\begin{align}
&\delta R_{CD} = R_{P(C}h_{D)}^P - R_{CPDQ}h^{PQ} - \frac{1}{2} \nabla^2 h_{CD} \\
&R_{P(C}h_{D)}^P = \frac{1}{2} h_{CD} \left[ - \frac{2}{3} V_g + \frac{1}{3} x V_f G \left( 4 + \frac{1}{G^2}\right) \right] + \frac{4}{3} \nabla_P \Phi \nabla_{(C}\Phi h_{D)}^P - \frac{1}{2} x V_f G g_{eff}^{AB} h_{A(C}g_{D)B}
\end{align}
where in the last equation we have used Einstein equations.
Putting everything together we get
\begin{align}
&\nabla^2 h_{CD} + 2 R_{CPDQ} h^{PQ} - \frac{8}{3} \nabla_P \Phi \nabla_{(C} \Phi h_{D)}^P + \frac{1}{3} x V_f g_{CD} \left[ - \frac{1}{2} \frac{k\left(\lambda\right) {\dot{\tau}}^2}{G} h^{zz} \left( 4 + \frac{1}{G^2} \right) - \right.  \notag \\
&\left. - G g_{eff}^{AM} g_{eff}^{BN} g_{AB} h_{MN} + G g_{eff}^{AB} h_{AB}  \right] - x V_f \left[ - \frac{1}{2} \frac{k\left(\lambda\right) {\dot{\tau}}^2}{G} h^{zz} g_{eff.}^{AB} g_{AC} g_{BD}  - \right. \notag \\
& \left. - G g_{eff}^{AM} g_{eff}^{BN} g_{AC} g_{BD} h_{MN} + G g_{eff}^{AB} h_{A(C} g_{D)B} \right] = 0
\end{align}
We can make further progress by computing
\begin{align}
&g_{eff}^{AM} g_{eff}^{BN} g_{AB} h_{MN}  = e^{-2A} \left( \frac{h^{zz}}{G^4} + \eta^{\alpha \beta} h_{\alpha \beta} \right) \\
&g_{eff}^{AB}h_{AB} = e^{-2A} \left( \frac{h_{zz}}{G^2} + \eta^{\alpha \beta} h_{\alpha \beta} \right)
\end{align}
Using the definition of G and the above equations one can show that
\begin{align}
- \frac{1}{2} \frac{k\left(\lambda\right) {\dot{\tau}}^2}{G} h^{zz} \left( 4 + \frac{1}{G^2} \right) - G g_{eff}^{AM} g_{eff}^{BN} g_{AB} h_{MN} + G g_{eff}^{AB} h_{AB} = e^{-2A} h_{zz} \frac{5 G^2 - 4 G^4 - 1}{2 G^3}
\end{align}
We can now write the equation of motion for $h_{CD}$:
\begin{align}
&\nabla^2 h_{CD} + 2 R_{CPDQ}h^{PQ} - \frac{8}{3} \nabla_P \Phi \nabla_{(C}\Phi h_{D)}^P + \frac{1}{3} x V_f g_{CD} e^{-2A} h_{zz} \frac{5 G^2 - 4 G^4 -1}{2G^3} - \notag \\
& - x V_f \left( - \frac{1}{2} \frac{k\left(\lambda\right) {\dot{\tau}}^2}{G} h^{zz} g_{eff}^{AB}g_{AC} g_{BD} - G g_{eff}^{AM} g_{eff}^{BN} g_{AC} g_{BD} h_{MN} + G g_{eff}^{AB} h_{A(C}g_{D)B} \right) = 0
\end{align}

We now wish to analyse the EOM of $h_{CD}$, that is, study it when $CD = zz$, $CD = z\mu$ and $CD = \mu \nu$. The equations for these cases are respectively
\begin{align}
&\nabla^2 h_{zz} + 2 R_{zPzQ} h^{PQ} - \frac{8}{3} {\dot{\Phi}}^2 e^{-2A} h_{zz} + x V_f h_{zz} \frac{G^2 - 2 G^4 +1}{3 G^3} = 0 \\
& \nabla^2 h_{z\mu} + 2 R_{zP\mu Q} h^{PQ} - \frac{4}{3} {\dot{\Phi}}^2 e^{-2A} h_{z \mu} + x V_f \frac{h_{z\mu}}{G} = 0 \\
& \nabla^2 h_{\mu \nu} + 2 R_{\mu P \nu Q} h^{PQ} + \frac{1}{2} x V_f \frac{G^2 - 1}{G} h_{zz} \eta_{\mu \nu} + \frac{1}{3} x V_f \eta_{\mu \nu} h_{zz} \frac{5 G^2 - 4 G^4 -1}{2 G^3} = 0
\end{align}
If we compute the Riemann tensor components in this background we find
\begin{align}
&R_{zPzQ} h^{PQ} = - e^{-2A} h_{\alpha}^\alpha \ddot{A} \\
&R_{zP\mu Q} h^{PQ} = e^{-2A} h_{z \mu} \ddot{A} \\
& R_{\mu P \nu Q} h^{PQ} = e^{-2A} h_{\mu \nu} {\dot{A}}^2 - e^{-2A} h_{\alpha}^\alpha \eta_{\mu \nu} {\dot{A}}^2 - e^{-2A} h_{zz} \eta_{\mu \nu} \ddot{A}
\end{align}
and for the divergence terms we get
\begin{align}
& \nabla^2 h_{zz} = 2 e^{-2A} h_{\alpha}^\alpha {\dot{A}}^2 - 10 e^{-2A} h_{zz}  {\dot{A}}^2  - e^{-2A} \dot{A} \partial_z h_{zz} - 2 e^{-2A} h_{zz} \ddot{A} + e^{-2A} \partial_z^2{h_{zz}} - 4 e^{-2A} \dot{A} \partial_\alpha h_{z}^\alpha + e^{-2A} \partial_\alpha \partial^\alpha h_{zz} \\
& \nabla^2 h_{z \mu} = - 9 e^{-2A} h_{z\mu} {\dot{A}}^2 - e^{-2A} \dot{A} \partial_z h_{z \mu} - 2 e^{-2A} h_{z \mu} \ddot{A} + e^{-2A} \partial_z^2 h_{z\mu} - 2 e^{-2A} \dot{A} \partial_\alpha h_{\mu}^\alpha + e^{-2A} \partial_\alpha \partial^\alpha h_{z\mu} + 2 e^{-2A} \dot{A} \partial_\mu h_{zz} \\
& \nabla^2 h_{\mu \nu} = e^{-2A} \left( -4 h_{\mu\nu} {\dot{A}}^2 + 2 h_{zz} \eta_{\mu\nu} {\dot{A}}^2 - \dot{A} \partial_z h_{\mu \nu} -2 h_{\mu \nu} \ddot{A} + \partial_z^2 h_{\mu \nu} + \partial_\alpha \partial^\alpha h_{\mu \nu} + 2 \dot{A} \partial_\mu h_{z \nu} + 2 \dot{A} \partial_\nu h_{z \mu}\right).
\end{align}

We are interested in solution with $h_{zz} = 0 = h_{z\mu}$. Then from the equations of $h_{zz}$ and $h_{z\mu}$ we can conclude that
\begin{align}
2 e^{-2A}  \left({\dot{A}}^2 - \ddot{A}\right) h_{\alpha}^\alpha = 0 \Leftrightarrow h_{\alpha}^\alpha = 0 \\
-2 e^{-2 A} \dot{A} \partial_\alpha h^{\alpha}_\mu = 0 \Leftrightarrow \partial^\alpha h_{\alpha \mu} = 0 \\
\nabla^2 h_{\mu \nu} + 2 {\dot{A}}^2 e^{-2A} h_{\mu \nu} = 0
\end{align}
where the last equation is the IHQCD result. So we have show, that in the Einstein frame the EOM of the TT components of the graviton in this family of backgrounds is the same as in IHQCD.

The last result means that in the string frame the equation of motion of the graviton will also be the same as in IHQCD. This follows from the fact that $h^E_{\alpha \beta} = e^{- 4 \Phi / 3} h^S_{\alpha \beta}$ and $A = A_E = A_S - 2 \Phi / 3$. Hence all the phenomenological machinery for the spin J fields in IHQCD can be reproduced here. Having said this,
the proposed equation of motion for the spin J fields in the graviton Regge trajectory is
\begin{align}
&\left( \nabla^2 - 2 e^{-2A_s} \dot{\Phi} \nabla_z - \frac{\Delta ( \Delta - 4 )}{L^2} + J {\dot{A}}^2 e^{-2A_s} + \left( J - 2\right) e^{-2A_s} \left( a \ddot{\Phi} + b \left( \ddot{A} - {\dot{A}}^2 \right) + c {\dot{\Phi}}^2 + d \dot{A} \dot{\Phi} + e {\dot{\tau}}^2  \right)\right) h^{TT}_{\alpha_1 \dots \alpha_J} = 0, \\
&\frac{\Delta ( \Delta - 4 )}{L^2}  = \frac{2}{ls^2} \left(J - 2 \right) \left( 1 + \frac{f}{\sqrt{\lambda}} \right) + \frac{J^2 -4}{\lambda^{4/3}}
\end{align}
This equation is almost the same as the one derived for IHQCD except for the following: i) in the IHQCD case the term $\dot{A} \dot{\Phi}$ does not appear because it can be removed through the use of the equations of motion. Here that is not the case; ii) we have not included the terms $\ddot{\tau}$, $\dot{\tau} \dot{A}$ and $\dot{\tau} \dot{\Phi}$ because the background is symmetric through $\tau \rightarrow - \tau$; iii) I have checked in the Mathematica notebook SpinJ EOMs the spin J fields behave as $h^{TT}_{\alpha_1 \dots \alpha_J} \sim c_1 z^2 + c_2 z^{2-2 J}$ which is the same as  $h^{TT}_{\alpha_1 \dots \alpha_J} \sim c_1 z^{\Delta - J} + c_2 z^{4 - \Delta -J}$ for $\Delta = 2 + J$, i.e. the free theory result.

\section{EOM and gravitational coupling of the $U(1)$ gauge field}

As mentioned previously, for the QCD vacuum we set $A^R_a = 0 = A^L_a$. Here we turn on those fields and derive their action up to quadratic order. The resulting action is already known in the Einstein frame and is the one that Matti and Kiritsis have in their papers. I will do such a derivation in order to understand how to compute the gravitation coupling between a $U(1)$ gauge field and the spin J fields in the graviton's Regge trajectory.

\subsection{The action of the vector $U(1)$ gauge field}

When we turn on the gauge fields, the flavour action becomes
\begin{align}
&S_f = - \frac{1}{2} M^3 N_c \int dz d^4 x V_f \left( \lambda, \tau \right) \bold{Tr} \left[ \sqrt{\text{det} \left(g_{eff. AB} + T^{L}_{AB}\right)} + \sqrt{\text{det} \left(g_{eff. AB} + T^{R}_{AB}\right)} \right], \\
&T^{L}_{AB} = w\left(\lambda\right) F^{L}_{AB} + k\left(\lambda\right)D_{(A}\tau D_{B)}\tau -   k\left(\lambda\right) \partial_A \tau \partial_B \tau , \\
&T^{R}_{AB} = w\left(\lambda\right) F^{R}_{AB} + k\left(\lambda\right)D_{(A}\tau D_{B)}\tau - k\left(\lambda\right) \partial_A \tau \partial_B \tau .
\end{align}
Here we remember that $\bold{Tr}$ is a trace over the gauge indices while the determinant is computed with respect to the space-time indices. We note that
\begin{align}
\text{det}\left(g_{eff. AB} + T^{L/R}_{AB} \right) = \text{det}{g_{eff}}  \text{det} \left( 1 + g_{eff}^{-1} T^{L/R} \right).
\end{align}
To compute $\text{det} \left( 1 + g_{eff}^{-1} T^{L/R} \right)$ we will define $C^{L/R} = g_{eff}^{-1} T^{L/R}$ and use the identity
\begin{align}
\ln \text{det} \left(1+C^{L/R}\right) = \text{tr} \ln\left(1 + C^{L/R}\right) = \text{tr} C^{L/R} - \frac{1}{2} \text{tr} {\left(C^{L/R}\right)}^2 + \cdots
\end{align}
From this it follows that
\begin{align}
&\sqrt{\text{det}  \left(1+C^{L/R}\right)} = 1 + \frac{1}{2} \text{tr} C^{L/R} - \frac{1}{4} \text{tr} {\left(C^{L/R}\right)}^2, \\
& \text{tr} C^{L/R} = \sum_A C^{L/R}_{AA} = \sum _{A,B}  \left(g^{-1}_eff\right)_{AB} T^{L/R}_{AB}, \\
& \text{tr} {\left(C^{L/R}\right)}^2 =  \sum_A \left(C^{L/R} C^{L/R} \right)_{AA} = \sum_{A,B} C^{L/R}_{AB} C^{L/R}_{BA} = \sum_{A,B,C,D}  \left(g^{-1}_{eff}\right)_{AC} T^{L/R}_{CB} \left(g^{-1}_{eff}\right)_{BD} T^{L/R}_{DA}.
\end{align}

Before we proceed with the calculation let us give expressions for $g^{-1}_{eff}$ and $T^{L/R}$
\begin{align}
&\left(g^{-1}_{eff}\right)_{zz} = \frac{e^{-2A}}{G^2}, \\
&\left(g^{-1}_{eff}\right)_{z\mu} = \left(g^{-1}_{eff}\right)_{\mu z} = 0, \\
&\left(g^{-1}_{eff}\right)_{\mu \nu} = \eta^{\mu \nu} e^{-2A}, \\
& T^{L/R}_{zz} = 0, \\
& T^{L/R}_{z \mu} =  w\left(\lambda \right) \left(  \pm \partial_z A_\mu + \partial_z V_\mu \right), \\
& T^{L/R}_{\mu z} = -  w\left(\lambda \right) \left(  \pm \partial_z A_\mu + \partial_z V_\mu \right), \\
& T^{L/R}_{\mu \nu} = 4 k\left(\lambda\right) \tau^2 A_\mu A_\nu + w\left(\lambda \right) \left( \pm A_{\mu \nu} + V_{\mu \nu} \right)
\end{align}
where the vector and axial gauge fields $V_M$ and $A_M$ are linear combinations of the left and right gauge fields
\begin{align}
V_M = \frac{A^L_M + A^R_M}{2}, \, A_M = \frac{A^L_M - A^R_M}{2}
\end{align}
in the gauge $V_z = 0 = A_z$, $V_{\mu \nu} = \partial_\mu V_\nu  - \partial_\nu V_\mu$ and $A_{\mu \nu} = \partial_\mu A_\nu  - \partial_\nu A_\mu$.

Using the expressions for $g^{-1}_{eff}$ and $T^{L/R}$ we get
\begin{align}
\text{tr} C^{L/R} = 4 e^{-2A} k \left( \lambda \right) \tau^2 A_\mu A^\mu
\end{align}
and
\begin{align}
\text{tr} \left( {C^{L/R}}^2\right) = - 2 \left( g^{-1}_{eff} \right)_{zz} \left( g^{-1}_{eff} \right)_{\mu \nu} T^{L/R}_{z \mu} T^{L/R}_{z \nu} +  \left( g^{-1}_{eff} \right)_{\mu \nu} \left( g^{-1}_{eff} \right)_{\alpha \beta} T^{L/R}_{\nu \alpha} T^{L/R}_{\beta \mu} 
\end{align}
The expressions for each term in the last equation are
\begin{align}
&\sum_{\mu, \nu} \left( g^{-1}_{eff} \right)_{zz} \left( g^{-1}_{eff} \right)_{\mu \nu} T^{L/R}_{z \mu} T^{L/R}_{z \nu} = \frac{{w\left(\lambda\right)}^2}{G^2} e^{-4 A} \left( \partial_z A_\mu \partial_z A^\mu \pm 2 \partial_z A_\mu \partial_z V^\mu + \partial_z V_\mu \partial_z V^\mu \right) , \\
& \sum_{\alpha, \beta, \mu, \nu} \left( g^{-1}_{eff} \right)_{\mu \nu} \left( g^{-1}_{eff} \right)_{\alpha \beta} T^{L/R}_{\nu \alpha} T^{L/R}_{\beta \mu} = - e^{-4A} {w\left(\lambda\right)}^2 \left( V_{\mu \nu} V^{\mu \nu} \mp 2 V_{\mu \nu} A^{\mu \nu} + A_{\mu \nu} A^{\mu \nu} \right)
\end{align}

Putting all these results together, after a tedious amount of algebra one gets
\begin{align}
&S_f = - \frac{1}{2} M^3 N_c \int dz d^4 x V_f \left(\lambda, \tau \right) \sqrt{- g_{eff}} \bold{Tr} \left[ 2 + 4 e^{-2A} k\left( \lambda \right) \tau^2 A_\mu A^\mu + \frac{{w\left(\lambda\right)}^2 e^{-4A}}{G^2} \left( \partial_z A_\mu \partial_z A^\mu +  \partial_z V_\mu \partial_z V^\mu \right) + \right. \\
&\left. + \frac{1}{2} e^{-4A} {w\left(\lambda\right)}^2 \left( V_{\mu \nu}V^{\mu \nu} + A_{\mu \nu} A^{\mu \nu} \right) \right]
\end{align}
The first term is the background term. There are two extra terms:
\begin{align}
\label{eq:VM_action}
&S_V = - \frac{1}{2} M^3 N_c \bold{Tr} \int dz d^4x V_{f} \left( \lambda, \tau \right) {w\left(\lambda\right)}^2 G^{-1} e^{A} \left( \partial_z V_\mu \partial_z V^\mu + \frac{1}{2} G^2 V_{\mu \nu} V^{\mu \nu} \right), \\
& S_A = - \frac{1}{2} M^3 N_c \bold{Tr} \int dz d^4 x V_{f} \left( \lambda, \tau \right) {w\left(\lambda\right)}^2 G^{-1} e^{A} \left( \partial_z A_\mu \partial_z A^\mu + \frac{1}{2} G^2 A_{\mu \nu} A^{\mu \nu} + 4 e^{2A} \frac{k\left(\lambda\right) G^2 \tau^2}{{w\left(\lambda\right)}^2} A_\mu A^\mu \right)
\end{align}
which are the actions for the Non-singlet Vector and Axial-Vector mesons presented in Matti and Kiritsis papers.

\subsection{Equation of motion of the EOM in terms of u}
We wont derive the equation of motion of the non-normalizable mode of the $U\left(1\right)$ gauge field because it is done so in~\cite{Arean:2013tja}. Introducing the variable
\begin{equation}
\frac{du}{dz} = G
\end{equation}
we get
\begin{equation}
\frac{d^2 f_Q}{d u^2} - Q^2 f_Q + \frac{d f_Q}{du} \left( \frac{dA}{du} + 2 \frac{w'\left(\Phi\right)}{w\left(\Phi\right)} \frac{d\Phi}{du} + \frac{\frac{\partial V_f}{\partial \tau}}{V_f} \frac{d\tau}{du} +  \frac{\frac{\partial V_f}{\partial \Phi}}{V_f} \frac{d\Phi}{du} \right) = 0.
\end{equation}
Using the fact that
\begin{equation}
\frac{dA}{du} = \frac{e^A}{G q} , \quad  \frac{d\Phi}{du} = \frac{e^A}{G q} \frac{d\Phi}{dA} , \quad \frac{d\tau}{du} = \frac{e^A}{G q} \frac{d\tau}{dA}
\end{equation}
we finally get
\begin{equation}
\frac{d^2 f_Q}{d u^2} - Q^2 f_Q + \frac{d f_Q}{du} \frac{e^A}{G q} \left(1 + 2 \frac{w'\left(\Phi\right)}{w\left(\Phi\right)} \frac{d\Phi}{dA} + \frac{\frac{\partial V_f}{\partial \tau}}{V_f} \frac{d\tau}{dA} +  \frac{\frac{\partial V_f}{\partial \Phi}}{V_f} \frac{d\Phi}{dA} \right) = 0.
\end{equation}
where
\begin{equation}
 \frac{\frac{\partial V_f}{\partial \Phi}}{V_f} = \frac{V_{f0}'\left(\Phi\right)}{V_{f0}}, \quad   \frac{\frac{\partial V_f}{\partial \tau}}{V_f} = - \frac{2\tau\left( -a_1 + a_2 + a_1 a_2 \tau^2\right)}{1 +a_1 \tau^2}
\end{equation}

\subsection{From Einstein frame to string frame}

In our computations we wish to work in the string frame. Hence it is desirable to write $S_f$ in the string frame. As we have seen the square roots of the determinantes present in $S_f$ can be computed as
\begin{align}
\sqrt{-\det \left( g + T^{L/R} \right)} = \sqrt{-g} \left[ 1 + \frac{1}{2} \text{tr} X - \frac{1}{4} \text{tr} X^2 + \frac{1}{8} {\left( \text{tr} X \right)}^2 + \dots \right]
\end{align}
with $X = g^{-1} T^{L/R}$ for some tensor $T^{L/R}$.

The Einstein and string frame are related through $g^s_{MN} = e^{ 4 \Phi / 3}g^{E}_{MN}$. From this it follows that $\sqrt{-g} = e^{- 10 \Phi / 3} \sqrt{-g_s}$ and the matrix $X$ in the previous equation can be written as $X = g_s^{-1} T^{L/R}_s$ with $T^{L/R}_s = e^{4 \Phi / 3} T^{L/R}_E$. This implies that
\begin{align}
\sqrt{-\det \left( g + T^{L/R} \right)} = e^{- 10 \Phi / 3} \sqrt{-\det \left( g_s + T^{L/R}_s \right)}
\end{align}
This implies that $S_f$ in the string frame is obtained by simply substituting the metric by the string frame metric $g_s$, substitute $k\left(\lambda\right)$ and $w\left(\lambda\right)$ by $k_s \left(\lambda\right) = e^{4 \Phi / 3} k\left(\lambda\right) $ and $w_s \left(\lambda\right) = e^{4 \Phi / 3} w\left(\lambda\right) $ and multiply $V_f$ by the factor $e^{-10 \Phi / 3}$. The derivation of $S_V$ and $S_A$ in the string frame will be formally the same and hence the results equal to the Einstein frame ones but with $w_s$, $k_s$ and $e^{-10 \Phi / 3} V_f$ in place of $w$, $k$ and $V_f$ respectively.

\subsection{Coupling between the graviton and the U(1) vector gauge field}
In the last subsection we learned that we can translate directly results in the Einstein frame to results in the string frame by substituting $w$, $k$ and $V_f$ for $w_s$, $k_s$ and $e^{-10 \Phi / 3} V_f$ respectively. In this section we will derive the coupling between the graviton and the U(1) vector gauge field in the Einstein frame. After we have it, it is straightforward writing it in the string frame.

We start by writing the square roots of the determinants as
\begin{align}
	\sqrt{-\det g_{eff}} \left[ 1 + \frac{1}{2} \text{tr} \left( g^{-1}_{eff} T^{L/R}\right) - \frac{1}{4} \text{tr} \left( g^{-1}_{eff} T^{L/R} g^{-1}_{eff} T^{L/R} \right) + \dots \right]
\end{align}
We now wish to perturb this expression with respect to the metric in order to get the desired gravitational coupling. For our purposes we can ignore the perturbation of $\sqrt{-\det g_{eff}}$ and consider and make $A_{\mu} = 0$ since we are only interested in the coupling of $V_\mu$. We wish then to compute $\delta \text{tr} \left( g^{-1}_{eff} T^{L/R}\right)$ and $\delta \text{tr} \left( g^{-1}_{eff} T^{L/R} g^{-1}_{eff} T^{L/R} \right)$. Using the fact that $\delta g^{AB}_{eff} = - g_{eff}^{AM} g_{eff}^{BN} h_{MN}$ one can show
\begin{align}
&\delta \text{tr} \left( g^{-1}_{eff} T^{L/R}\right) = -  g_{eff}^{AM} g_{eff}^{BN} h_{MN} T^{L/R}_{AB}, \\
&\delta \text{tr} \left( g^{-1}_{eff} T^{L/R} g^{-1}_{eff} T^{L/R} \right) = - 2 g^{AM}_{eff.} g^{BN}_{eff.} g^{CD}_{eff.} T^{L/R}_{BC} T^{L/R}_{DA} h_{MN}
\end{align}
Using the expressions for $g^{1}_{eff}$ and $T^{L/R}$ we get
\begin{align}
&\delta \text{tr} \left( g^{-1}_{eff} T^{L/R}\right) = - 4 e^{-4A} k\left(\lambda\right) \tau^2 h^{\alpha \beta} A_\alpha A_\beta = 0 , \\
& \delta \text{tr} \left( g^{-1}_{eff} T^{L/R} g^{-1}_{eff} T^{L/R} \right)  = 2 e^{-6A} {w\left( \lambda \right)}^2 h^{\mu \nu} \left( V_{\mu \sigma} \eta^{\sigma \rho} V_{\nu \rho} + \frac{1}{G^2} \partial_z V_\mu \partial_z V_\nu \right)
\end{align}
Hence the coupling between the vector $U(1)$ gauge field and the graviton is given by
\begin{align}
\frac{1}{2} M^3 N_c N_f \int d^5 X \sqrt{-\det g_{eff}} V_{f} \left(\lambda, \tau \right) e^{-6 A} {w\left(\lambda\right)}^2 h^{\mu \nu} \left( V_{\mu \sigma} \eta^{\sigma \rho} V_{\nu \rho} + \frac{1}{G^2} \partial_z V_\mu \partial_z V_\nu \right)
\end{align}

From this result we wish to generalise to any spin J field. The coupling came from the term $\delta \text{tr} \left( g^{-1}_{eff} T^{L/R} g^{-1}_{eff} T^{L/R} \right)$ wich can be written as
\begin{align}
2 g^{\alpha \mu} g^{\beta \nu} g_{eff}^{AB} T^{L/R}_{A \alpha} T^{L/R}_{B \beta} h_{\mu \nu}
\end{align}
Here $T^{L/R}_{AB} = w\left(\lambda\right) F^V_{AB}, \, F^V_{AB} = \nabla_A V_B - \nabla_B V_A$ since $A_M = 0$ for all space-time indices. The left and right contributions for the coupling are equal to
\begin{align}
2 g^{\alpha \mu} g^{\beta \nu} g_{eff}^{AB} {w\left(\lambda\right)}^2 F^{V}_{A \alpha} F^{V}_{B \beta} h_{\mu \nu}
\end{align}
and the coupling is given by
\begin{align}
\frac{1}{2} M^3 N_c \bold{Tr} \int d^5 X \sqrt{-g} G V_{f} \left(\lambda, \tau \right)  {w\left(\lambda\right)}^2 g^{\alpha \mu} g^{\beta \nu} g_{eff}^{AB} F^V_{A \alpha} F^V_{B \beta} h_{\mu \nu}
\end{align}
In the string frame this coupling takes the form
\begin{align}
\frac{1}{2} M^3 N_c \bold{Tr} \int d^5 X \sqrt{-g_{s}} G e^{-\frac{10}{3}\Phi} V_{f} \left(\lambda, \tau \right)  {w_s\left(\lambda\right)}^2 g^{\alpha \mu}_s g^{\beta \nu}_s g_{eff. s}^{AB} F^V_{A \alpha} F^V_{B \beta} h_{\mu \nu}^s
\end{align}
For a spin J field we propose the coupling
\begin{align}
k_J \int d^5X \sqrt{-g_s} G e^{-\frac{10}{3} \Phi} V_f \left( \lambda, \tau \right) {w_s\left(\lambda \right)}^2 g_s^{\alpha \mu} g_s^{\beta \nu} g_{eff. s}^{AB} F^V_{A\alpha} \nabla^{\alpha_1} \dots \nabla^{\alpha_{J-2}} F^V_{B \beta} h_{\mu \nu \alpha_1 \dots \alpha_{J-2}}
\label{eq:spin_J_graviton_coupling}
\end{align}

\subsection{Regge limit analysis of the spin J coupling}
Now that we have an ansatz for the coupling between the vector $U(1)$ gauge field and the spin J fields in the graviton's Regge trajectory we wish to study it's amplitude in the Regge limit. From now on, to simplify notation, we will assume that all quantities are given in the string frame. In the Regge limit the covariant derivatives can be substituted by partial derivatives because of powers of $\sqrt{s}$. The coupling reduces to
\begin{align}
&g^{\alpha \mu} g^{\beta \nu} g^{\alpha_1 \beta_1} \dots g^{\alpha_{J-2} \beta_{J-2}} g^{AB}_{eff} F^{V}_{A \alpha} \partial_{\beta_1} \dots \partial_{\beta_{J-2}} F^{V}_{B \beta} h_{\mu \nu \alpha_1 \dots \alpha_{J_2}} = \\
& = g^{\alpha \mu} g^{\beta \nu} g^{\alpha_1 \beta_1} \dots g^{\alpha_{J-2} \beta_{J-2}} e^{-2A} \left( \frac{1}{G^2} \partial_z V_\alpha \partial_{\beta_1} \dots \partial_{\beta_{J-2}} \partial_z V_\beta + \eta^{\rho \sigma} V_{\rho \alpha} \partial_{\beta_1} \dots \partial_{\beta_{J-2}} V_{\sigma \beta} \right) h_{\mu \nu \alpha_1 \dots \alpha_{J_2}}
\end{align}
Since $V_\mu = n_\mu f_Q \left( z\right) e^{i k \cdot x}$, $\partial_\mu V = i k_\mu V$ and then
\begin{align}
g^{\alpha \mu} g^{\beta \nu} g^{\alpha_1 \beta_1} \dots g^{\alpha_{J-2} \beta_{J-2}} \left( i k_{\beta_1}\right) \dots  \left( i k_{\beta_{J-2}}\right) e^{-2A} \left( \frac{1}{G^2} \partial_z V_\alpha  \partial_z V_\beta + \eta^{\rho \sigma} V_{\rho \alpha} V_{\sigma \beta} \right) h_{\mu \nu \alpha_1 \dots \alpha_{J_2}}
\end{align}

Before moving forward we introduce our kinematics. We use light-cone coordinates $\left(+,-,\perp \right)$, with the flat space metric given by $ds^2 = - dx^+ dx^- + d x^2_\perp$, where $x_\perp \in \mathbb{R}^2$ is a vector in impact parameter space. 
We take for the large $s$ kinematics of  $12\to34$ scattering the following
\begin{align}
&k_1=\left(\!\sqrt{s},-\frac{Q^2}{\sqrt{s}} ,0\right),\  \ k_3=-\left(\!\sqrt{s},\frac{ q_\perp^2 -Q^2}{\sqrt{s}} , q_\perp \right)\!,\\
&k_2=\left(\frac{M^2}{\sqrt{s}},\sqrt{s} ,0\right),\  \ k_4=-\left(\frac{M^2+ q_\perp^2}{\sqrt{s}},\sqrt{s} ,-q_\perp \right).
\nonumber
\label{eq:kinematics}
\end{align}
where $k_1$ and $k_3$ are  respectively the incoming  and outgoing photon momenta. The proton 
target has mass $M$ and incoming  and outgoing  momenta $k_2$ and $k_4$, respectively.
For the forward Compton scattering amplitude we set $q_\perp = 0$, so that $k_1=-k_3$, and we take the same polarization
for the  incoming and outgoing photon. The possible polarization vectors are
\begin{equation}
  \label{eq:polarization vectors} 
  n(\lambda)=\begin{cases}
    (0,0,\epsilon_\lambda) \,, & \lambda=1,2 \,,\\
    \left( \sqrt{s}/Q, Q/\sqrt{s},0 \right) , & \lambda=3\,,
    \end{cases}
\end{equation}
where $\epsilon_\lambda$ is just the usual transverse polarization vector.

In DIS we will be interested in incoming and outgoing  off-shell photons with the same polarisation. For both photons with perpendicular polarisation ($\lambda = 1, 2$) and both with longitudinal polarisation ($\lambda = 3$) we have respectively:
\begin{align}
&g^{\alpha \mu} g^{\beta \nu} e^{-2 A} \left( \frac{1}{G^2} \partial_z V_\alpha \partial_z V_\beta + \eta^{\rho \sigma} V_{\rho \alpha} V_{\sigma \beta} \right) h_{\mu \nu \alpha_1 \dots \alpha_{J-2}} = \frac{1}{4} s e^{-6A} {f_Q}^2 e^{-i q_\perp \cdot x_\perp} h^{--}_{\alpha_1 \dots \alpha_{J-2}}, \\
&g^{\alpha \mu} g^{\beta \nu} e^{-2 A} \left( \frac{1}{G^2} \partial_z V_\alpha \partial_z V_\beta + \eta^{\rho \sigma} V_{\rho \alpha} V_{\sigma \beta} \right) h_{\mu \nu \alpha_1 \dots \alpha_{J-2}} = \frac{1}{4} s e^{-6A} \frac{{\partial_z f_Q}^2}{Q^2 G^2} e^{-i q_\perp \cdot x_\perp} h^{--}_{\alpha_1 \dots \alpha_{J-2}}
\end{align}


\subsection{Upper part Witten diagram}
We are now ready to compute the upper part of the Witten diagram
\begin{align}
&k_J \int d^5X \sqrt{-g_s} G e^{-\frac{10}{3}\Phi} V_f \left( \lambda, \tau \right) {w_s\left(\lambda \right)}^2 g^{\alpha_1 \beta_1} \dots g^{\alpha_{J-2} \beta_{J-2}} \left( i k_{\beta_1}\right) \dots  \left( i k_{\beta_{J-2}}\right)  \frac{s}{4} e^{-6A} \left(  {f_Q}^2 + \frac{{\partial_z f_Q}^2}{Q^2 G^2}  \right) e^{-i q_\perp \cdot x_\perp} h^{--}_{\alpha_1 \dots \alpha_{J-2}} = \notag \\
& = i^{J-2} k_J \int d^5X \sqrt{-g_s} G e^{-\frac{10}{3}\Phi} V_f \left( \lambda, \tau \right) {w_s\left(\lambda \right)}^2 e^{-2 J A}  s^{J/2} e^{-2A} \left(  {f_Q}^2 + \frac{{\partial_z f_Q}^2}{Q^2 G^2}  \right) e^{-i q_\perp \cdot x_\perp} h_{+ \dots +} =
\end{align}
where we have used the kinematics of the outgoing photon and kept the leadin term power of $\sqrt{s}$.
\subsection{Lower part Witten diagram}
We assume that the proton part of the Witten diagram is treated the same way as in IHQCD. The way we describe the proton's wavefunction is not relevant since it will be contained in a $\bar{z}$ that can be treated as a fitting constant. Like in IHQCD we describe the proton as a scalar field $\Upsilon\left(x, z\right) = e^{i P \cdot x} \upsilon\left(z\right)$ that has the following coupling with a spin J field in the graviton's Regge trajectory
\begin{align}
\bar{k}_J \int d^5 X \sqrt{-g_s} e^{-\Phi} \left( \Upsilon D_{b_1} \dots D_{b_J} \Upsilon \right) h^{b_1 \dots b_J}
\end{align}
Focusing on the TT part of the spin J field, we are left with the single coupling
\begin{align}
\bar{k}_J \int d^5 X \sqrt{-g_s} e^{-\Phi} \left( \Upsilon \partial_{\beta_1} \dots \partial_{\beta_J} \Upsilon \right) h^{\beta_1 \dots \beta_J}
\end{align}

Using the last equation and remembering that $\Upsilon\left(x, z\right) = e^{i P \cdot x} \upsilon\left(z\right)$ in the Regge limit the lower part of the Witten diagram contributes as
\begin{align}
&\bar{k}_J \int d^5 X \sqrt{-g_s} e^{-\Phi}  {\upsilon\left(z\right)}^2 \left( i k_{\beta_1}\right) \dots \left( i k_{\beta_J}\right) e^{i q_\perp \cdot x}  h^{\beta_1 \dots \beta_J} = \bar{k}_J \int d^5 X \sqrt{-g_s} e^{-\Phi}  {\upsilon\left(z\right)}^2 {\left( i k_+\right)}^J e^{i q_\perp \cdot x}  h^{+ \dots +} = \notag \\
& = \bar{k}_J \int d^5 X \sqrt{-g_s} e^{-\Phi} {\left( - 2 e^{-2 A}\right)}^J {\upsilon\left(z\right)}^2 {\left( \frac{i}{2} \sqrt{s}\right)}^J e^{i q_\perp \cdot x}  h_{- \dots -}  = i^J s^{J/2}  \bar{k}_J \int d^5 X \sqrt{-g_s} e^{-\Phi} e^{-2 J A} {\upsilon\left(z\right)}^2 e^{i q_\perp \cdot x}  h_{- \dots -}
\end{align}

\subsection{spin J exchange amplitude}
Now that we know the contributions form the upper and lower part of the diagram we can write the scattering amplitude $\mathcal{A}_J$ contribution due to the exchange of an even spin J field:
\begin{align}
 - k_J \bar{k}_J s^J \int d^5X d^5 \bar{X} \sqrt{-g_s} \sqrt{-\bar{g}_s} G e^{-\frac{10}{3}\Phi} V_f \left( \lambda, \tau \right) {w_s\left(\lambda \right)}^2 e^{-\bar{\Phi}} e^{-2 J \left(A+\bar{A}\right)}  e^{-2A} \left(  {f_Q}^2 + \frac{{\partial_z f_Q}^2}{Q^2 G^2}  \right)  {\bar{\upsilon}}^2 e^{-i q_\perp \cdot \left( x_\perp - \bar{x}_\perp \right)} \Pi _{+\dots+,-\dots-}\left(X, \bar{X} \right)
\end{align}
We now perform the change of variable $w = x - \bar{x}$ and using the identity
\begin{align}
 \int d^2 l_\perp e^{- i q_\perp l_\perp} \int \frac{dw^+ dw^-}{2} \Pi_{+ \dots +, - \dots -} \left(z, \bar{z}, w^+, w^-, l_\perp \right) = - \frac{i}{2^J} {\left( e^{A + \bar{A}} \right)}^{J-1}G_J \left(z, \bar{z}, t\right)
\end{align}
we get
\begin{align}
 & \mathcal{A}_J = \frac{k_J \bar{k}_J}{2^J} s^J \int dz d\bar{z} e^{2 A}e^{4 \bar{A}} G e^{-\frac{10}{3}\Phi} V_f \left( \lambda, \tau \right) {w_s\left(\lambda \right)}^2 e^{-\bar{\Phi}} e^{-J \left(A+\bar{A}\right)}  \left(  {f_Q}^2 + \frac{{\partial_z f_Q}^2}{Q^2 G^2}  \right)  {\bar{\upsilon}}^2 G_J \left(z, \bar{z}, t\right), \\ 
 & G_J \left(z , \bar{z}, t \right) = e^{\Phi + \bar{\Phi} - \frac{1}{2} \left(A + \bar{A}\right)} \sum_n \frac{\psi_n \left(J, z\right) \psi_n^* \left(J, \bar{z}\right)}{t_n\left(J\right) - t}
\end{align}

\subsection{Regge theory}
In order to get the total amplitude we need to sum over even spin J fields with $J \geq 2$. Then we can apply a Sommerfeld-Watson transform
\begin{align}
\frac{1}{2} \sum_{J \geq 2} \left(s^J + {\left(-s\right)}^J\right) = - \frac{\pi}{2} \int \frac{d J}{2 \pi i} \frac{s^J + \left(-s\right)^J}{\sin \pi J}
\end{align}
which requires analytic continuation of the amplitude for the spin J exchange to the complex J-plane. We assume that the J-plane integral can be deformed from the poles at even J, to the poles $J = j_n \left(t\right)$ defined by $t_n \left(J\right) = t$. The scattering domain of negative t contains these poles along the real axis for J < 2. The scattering amplitude for $t = 0$ is then
\begin{align}
&\mathcal{A} \left(s, 0 \right) = \sum_n g_n s^{j_n\left(0\right)} \int dz e^{A(\frac{3}{2} - {j_n\left(0\right)})} G e^{-\frac{7}{3} \Phi}  V_f \left( \lambda, \tau \right) {w_s\left(\lambda \right)}^2  \left(  {f_Q}^2 + \frac{{\partial_z f_Q}^2}{Q^2 G^2}  \right) \psi_n \left(j_n, z\right), \\
& g_n = - \frac{\pi}{2} \frac{k_{j_n\left(0\right)} \bar{k}_{j_n\left(0\right)}}{2^{j_n\left(0\right)}}  \left(i + \cot \frac{\pi j_n\left(0\right)}{2} \right) \frac{d j_n}{dt} \int d\bar{z} e^{\bar{A}\left( \frac{7}{2} - j_n \right)}  {\bar{\upsilon}}^2  \psi\left(j_n, \bar{z}\right) \notag
\end{align}

\subsection{Expressions for $F_2\left(x, Q^2\right)$ and $F_L\left(x, Q^2\right)$ due to pomeron exchange}
$F_2$ is related with the transverse and longitudinal cross-sections through the equation
\begin{equation}
 F_2\left(x, Q^2\right) = \frac{Q^2}{4 \pi^2 \alpha} \left( \sigma_T + \sigma_L \right)
\end{equation}
Using the final result of the previous section and the optical theorem we can write the holographic expression for $F_2$ and $F_L$
\begin{align}
 &F_2\left(x, Q^2\right) = \sum_n \frac{{\rm Im} \, g_n}{4 \pi^2 \alpha} Q^{2 j_n} x^{1-jn} \int dz e^{A(\frac{3}{2} - {j_n\left(0\right)})} G e^{-\frac{7}{3} \Phi}  V_f \left( \lambda, \tau \right) {w_s\left(\lambda \right)}^2  \left(  {f_Q}^2 + \frac{{\partial_z f_Q}^2}{Q^2 G^2}  \right) \psi_n \left(j_n, z\right), \\
& F_L\left(x, Q^2\right) = \sum_n \frac{{\rm Im} \,g_n}{4 \pi^2 \alpha} Q^{2 j_n} x^{1-jn} \int dz e^{A(\frac{3}{2} - {j_n\left(0\right)})} G e^{-\frac{7}{3} \Phi}  V_f \left( \lambda, \tau \right) {w_s\left(\lambda \right)}^2  \frac{{\partial_z f_Q}^2}{Q^2 G^2} \psi_n \left(j_n, z\right)
\end{align}
It turns out that a more convenient variable to perform the above integration is the u variable defined by $du = G dz$. Then the expressions for $F_2$ and $F_L$ are then
\begin{align}
 &F_2\left(x, Q^2\right) = \sum_n \frac{{\rm Im} \, g_n}{4 \pi^2 \alpha} Q^{2 j_n} x^{1-jn} \int du e^{A(\frac{3}{2} - {j_n\left(0\right)})}  e^{-\frac{7}{3} \Phi}  V_f \left( \lambda, \tau \right) {w_s\left(\lambda \right)}^2  \left(  {f_Q}^2 + \frac{{\partial_u f_Q}^2}{Q^2}  \right) \psi_n \left(j_n, u\right), \\
& F_L\left(x, Q^2\right) = \sum_n \frac{{\rm Im} \, g_n}{4 \pi^2 \alpha} Q^{2 j_n} x^{1-jn} \int du e^{A(\frac{3}{2} - {j_n\left(0\right)})}  e^{-\frac{7}{3} \Phi}  V_f \left( \lambda, \tau \right) {w_s\left(\lambda \right)}^2  \frac{{\partial_u f_Q}^2}{Q^2} \psi_n \left(j_n, u\right)
\end{align}
The structure function $F_2$ is related to the total cross-section $\sigma\left(\gamma p \rightarrow X\right)$ through
\begin{equation}
\sigma\left(\gamma p \rightarrow X\right) = 4 \pi^2 \alpha \lim_{Q^2 \rightarrow 0} \frac{F_2\left(x, Q^2\right)}{Q^2}.
\end{equation}
In the limit of $Q^2 \rightarrow 0$
\begin{equation}
\lim_{Q\rightarrow 0} f_Q = 1 , \quad \lim_{Q\rightarrow 0} \frac{\dot{f}_Q}{Q} = 0
\end{equation}
and it follows that
\begin{equation}
\sigma\left(\gamma p \rightarrow X\right) = \sum_n {\rm Im} g_n \,  s^{j_n - 1} \int du e^{-\left(j_n - 3/2 \right)A } e^{-\frac{7}{3} \Phi} V_f w_s^2  \,  \psi_n\left(u\right).
\end{equation}


\section{Vector Meson Trajectory}

In this section we will discuss how to include in our framework the exchange of the vector meson trajectory to compute holographically $F_2$ and $F_L$. Before starting the holographic computation let's review what Donnachie and Landshoff have done. More details in~\cite{donnachie_dosch_landshoff_nachtmann_2002}.

A fit to small-$x$ data is possible with the following ansatz
\begin{equation}
F_2 \left(x, Q^2\right) = \sum_{i = 0, 1, 2} f_i\left(Q^2\right) x^{- \epsilon_i},
\end{equation}
 where the $i=0$ corresponds to hard-pomeron exchange, $i=1$ corresponds to soft-pomeron exchange and $i = 2$ comes from $(f_2, a_2)$ exchange. Only the quantum numbers with $PC=++$ are exchanged.
 The precise definition of the coefficient functions $f_i\left(Q^2\right)$ will not be relevant to our discussion so we will omit them. The above equation can be also generalized to larger values of $x$ but we need to modify the ansatz in order to impose that $F_2\left(x, Q^2\right)$ vanishes as $x \rightarrow 1$ for all values of $Q^2$. Dimensional counting rules suggest the generalization
\begin{equation}
F_2\left(x, Q^2\right) = f_0 \left(Q^2\right) x^{-\epsilon_0} {\left(1-x\right)}^7 +  f_1 \left(Q^2\right) x^{-\epsilon_1} {\left(1-x\right)}^7 +  f_2 \left(Q^2\right) x^{-\epsilon_2} {\left(1-x^2\right)}^3.
\end{equation} 

The hard and soft pomeron are incorporated in our model when we consider the exchange of the graviton's Regge trajectory in the bulk as we have seen in previous papers. The quantum numbers~$I^G\left(J^{PC}\right)$ of $a_2$ and $f_2$ are respectively $1^{-} \left(2^{++}\right)$ and $0^{+}\left(2^{++}\right)$. With the graviton's Regge trajectory in mind, the first step to model these trajectories holographically could be finding the EOM of the fluctuations with these quantum numbers. Then we could generalize in the same way as we did with the graviton. However, in the model we are considering, the only present fluctuations with $J^{PC} = 2^{++}$ are identified with the tensor glueballs and not with these mesons. The way we have around this is to use the fact that the $\rho$ meson, $f_2$ and $a_2$ trajectories are degenerate and apply the procedure to the EOM of the $\rho$ mesons.

Another issue is the coupling in the bulk of the virtual photon with the fields of  $\left(a_2, f_2\right)$ trajectory. One strategy is to compute the cubic coupling of the vector $U\left(1\right)$ gauge field. I have done this and it is zero. The solution I have found around this is to use the that $\left(a_2, f_2\right)$ have the same quantum numbers as the hard and soft pomeron. Based on this we assume the coupling between the $U\left(1\right)$ gauge field dual to the off-shell photon with the meson trajectory is the same as the coupling with the graviton's Regge trajectory.

\subsection{EOM of the $\left(a_2, f_2\right)$ trajectory}

The equation (\ref{eq:VM_action}) can be rewritten as
\begin{equation}
S = - \frac{1}{4} M^3 N_c N_f \int dz d^4 x V_f w^2 e^{5A} G F_{AB}F^{AB} = - \frac{1}{4} M^3 N_c N_f \int dz d^4 x \sqrt{-g} V_f w^2 G F_{AB}F^{AB},
\label{eq:VM_new_action}
\end{equation}
where here the notation $F_{AB}F^{AB}$ means
\begin{equation}
F_{AB}F^{AB} = F_{AB} g_{\rm{eff.}}^{AC} g_{\rm{eff.}}^{BD} F_{CD} =  \frac{2}{G^2}e^{-4A}\left( \frac{1}{2} G^2 V_{\mu \nu} V^{\mu \nu} + \partial_z V_\mu \partial_z V^\mu \right)
\end{equation}
The equation of motion of the fluctuations can now be rewritten as
\begin{equation}
\nabla_A \left(V_f w^2 G F^{AB} \right) = 0
\end{equation}
The above equations are valid in the Einstein frame. However later we will need to work with expressions valid in the string frame. The action (\ref{eq:VM_new_action}) in the string frame takes the form
\begin{equation}
S = - \frac{1}{4} M^3 N_c N_f \int d^5 X \sqrt{-g_s} e^{-\frac{10}{3} \Phi} V_f w_s^2 G F_{AB}g^{AC}_{\rm{eff. (s)}}g^{BD}_{\rm{eff. (s)}}F_{CD}
\label{eq:VM_new_action_string_frame}
\end{equation}
The equation of motion is then
\begin{equation}
\nabla_{A} \left(e^{-\frac{10}{3} \Phi} V_f w_s^2 G g^{AC}_{\rm{eff. (s)}}g^{BD}_{\rm{eff. (s)}}F_{CD}  \right) = 0.
\label{eq:U1_eom_cov_der}
\end{equation}
In order to simplify the notation from now on we assume all the warp factors and effective metrics are in the string frame.

It is not known what is the equation of motion of the spin J fields. In our approach, as a first step, we will do the substitution
\begin{equation}
F_{CD} = \nabla_C h_D - \nabla_D h_C \Rightarrow \nabla_C h_{D a_2 \cdots a_J} - \nabla_D h_{C a_2 \cdots a_J}
\label{eq:spin_j_sub}
\end{equation}
and  hence the equation of motion becomes
\begin{align}
\frac{1}{e^{-\frac{10}{3} \Phi} V_f w_s^2 G}\nabla_A \left(e^{-\frac{10}{3} \Phi} V_f w_s^2 G \right)&g^{AC}_{\rm{eff. (s)}}g^{BD}_{\rm{eff. (s)}}\left(\nabla_C h_{D a_2 \cdots a_J} - \nabla_D h_{C a_2 \cdots a_J}\right) + \notag \\
&+\nabla_A \left[g^{AC}_{\rm{eff. (s)}}g^{BD}_{\rm{eff. (s)}}\left(\nabla_C h_{D a_2 \cdots a_J} - \nabla_D h_{C a_2 \cdots a_J}\right)\right] = 0
\label{eq:spin_J_eom_cov_term}
\end{align}

We now evaluate each of these terms. Because the background fields are only functions of $z$ the first term is
\begin{equation}
\partial_z\left( e^{-\frac{10}{3} \Phi} V_f w_s^2 G \right) g^{zz}_{\rm{eff. (s)}}g^{BD}_{\rm{eff. (s)}}\left(\nabla_z h_{D a_2 \cdots a_J} - \nabla_D h_{z a_2 \cdots a_J}\right)
\label{eq:spin_j_eom_first_term}
\end{equation}
For $B = z$ the above expression is 0, so all we need to do is to deal with the $B=\mu$ case. Later we will only be interested in the TT components of the spin J fields. This means $a_i = \alpha_i, \, i = 1,\cdots, J$.
The auxiliary computations
\begin{align}
& \nabla_z h_{\nu \alpha_2 \cdots \alpha_J} = \partial_z h_{\nu \alpha_2 \cdots \alpha_J} - J \dot{A} h_{\nu \alpha_2 \cdots \alpha_J}, \\
& \nabla_\nu h_{z \alpha_2 \cdots \alpha_J} = - \dot{A} h_{\nu \alpha_2 \cdots \alpha_J},
\label{eq:spin_J_cod_id_1}
\end{align}
allow us to write the first term as
\begin{equation}
\begin{cases}
0, & B = z \\
\frac{\partial_{z}\left( e^{-\frac{10}{3} \Phi} V_f w_s^2 G \right)}{ e^{-\frac{10}{3} \Phi} V_f w_s^2 G} \frac{e^{-4A}}{G^2} \eta^{\mu \nu} \left[ \partial_z h_{\nu \alpha_2 \cdots \alpha_J} - \left(J-1\right) \dot{A} h_{\nu \alpha_2 \cdots \alpha_J} \right], \, &B = \mu
\end{cases}
\end{equation}

Using the Leibniz rule the second term can split into three terms. They are
\begin{align}
&\left(\nabla_A g^{AC}_{\rm{eff.}}\right) g^{BD}_{\rm {eff.}} \left(\nabla_C h_{D \alpha_2 \cdots \alpha_J} - \nabla_D h_{C \alpha_2 \cdots \alpha_J}\right) \\
&g^{AC}_{\rm{eff.}}\left(\nabla_A g^{BD}_{\rm{eff.}}\right) \left(\nabla_C h_{D \alpha_2 \cdots \alpha_J} - \nabla_D h_{C \alpha_2 \cdots \alpha_J}\right) \\
& g^{AC}_{\rm{eff.}}g^{BD}_{\rm{eff.}} \left( \nabla_A \nabla_C h_{D \alpha_2 \cdots \alpha_J} - \nabla_A \nabla_D h_{C \alpha_2 \cdots \alpha_J}\right)
\end{align}
Let's start with the first one. For $B = z$ the first term becomes
\begin{equation}
\left(\nabla_A g^{AC}_{\rm{eff.}}\right) g^{zz}_{\rm {eff.}} \left(\nabla_C h_{z \alpha_2 \cdots \alpha_J} - \nabla_z h_{C \alpha_2 \cdots \alpha_J}\right) = \left(\nabla_A g^{A\mu}_{\rm{eff.}}\right) g^{zz}_{\rm {eff.}} \left(\nabla_\mu h_{z \alpha_2 \cdots \alpha_J} - \nabla_z h_{\mu \alpha_2 \cdots \alpha_J}\right) = 0,
\end{equation}
since $\nabla_A g^{A\mu}_{\rm{eff.}} = 0$.
For $B = \mu$ the first term becoms
\begin{equation}
\left(\nabla_A g^{AC}_{\rm{eff.}}\right) g^{\mu\nu}_{\rm {eff.}} \left(\nabla_C h_{\nu \alpha_2 \cdots \alpha_J} - \nabla_\nu h_{C \alpha_2 \cdots \alpha_J}\right) = \left(\nabla_A g^{Az}_{\rm{eff.}}\right) g^{\mu\nu}_{\rm {eff.}} \left(\nabla_z h_{\nu \alpha_2 \cdots \alpha_J} - \nabla_\nu h_{z \alpha_2 \cdots \alpha_J}\right).
\end{equation}
Using the identities of equation (\ref{eq:spin_J_cod_id_1}) and
\begin{align}
\nabla_A g^{Az}_{\rm{eff.}} = e^{-2A} \left( 4 \frac{\dot{A}}{G^2} - 2 \frac{\dot{G}}{G^3} - 4 \dot{A} \right),
\end{align}
we can show that the first term is
\begin{equation}
\begin{cases}
0, & B = z \\
\left( 4 \frac{\dot{A}}{G^2} - 2 \frac{\dot{G}}{G^3} - 4 \dot{A} \right) e^{-4A} \eta^{\mu \nu} \left[ \partial_z h_{\nu \alpha_2 \cdots \alpha_J} - \left(J-1\right) \dot{A} h_{\nu \alpha_2 \cdots \alpha_J} \right],  &B = \mu
\end{cases}
\end{equation}

The value of $\nabla_A g^{BD}_{\rm{eff.}}$ for $B = z$ is $\partial_A g^{zD}_{\rm{eff.}}$ and hence the second term is zero for $B = z$. For $B = \mu$ we have
\begin{equation}
\nabla_A g^{\mu D}_{\rm{eff.}} = \partial_A g^{\mu D}_{\rm{eff.}} + \Gamma^{\mu}_{AM} g^{MD}_{\rm{eff.}} + \Gamma^{D}_{AM} g^{\mu M}_{\rm{eff.}}
\end{equation}
and using the equations (\ref{eq:spin_J_cod_id_1}) the following identities hold
\begin{align}
& g^{AC}_{\rm{eff.}} \partial_A g^{\mu D}_{\rm{eff.}}  \left(\nabla_C h_{D \alpha_2 \cdots \alpha_J} - \nabla_D h_{C \alpha_2 \cdots \alpha_J}\right) = - 2 \frac{\dot{A}}{G^2} e^{-4A} \eta^{\mu \nu} \left[ \partial_z h_{\nu \alpha_2 \cdots \alpha_J} - \left(J-1\right) \dot{A} h_{\nu \alpha_2 \cdots \alpha_J} \right] \\
& g^{AC}_{\rm{eff.}} \Gamma^{\mu}_{AM} g^{MD}_{\rm{eff.}} \left(\nabla_C h_{D \alpha_2 \cdots \alpha_J} - \nabla_D h_{C \alpha_2 \cdots \alpha_J}\right) = 0 \\
& g^{AC}_{\rm{eff.}} \Gamma^{D}_{AM} g^{\mu M}_{\rm{eff.}} \left(\nabla_C h_{D \alpha_2 \cdots \alpha_J} - \nabla_D h_{C \alpha_2 \cdots \alpha_J}\right) = \dot{A} \left(1 + \frac{1}{G^2}\right)e^{-4A}\eta^{\mu\nu}\left[ \partial_z h_{\nu \alpha_2 \cdots \alpha_J} - \left(J-1\right) \dot{A} h_{\nu \alpha_2 \cdots \alpha_J} \right] 
\end{align}
Putting everything together we can write the second term as
\begin{equation}
\begin{cases}
0, & B = z \\
\dot{A} \left(1 - \frac{1}{G^2} \right) e^{-4A} \eta^{\mu \nu} \left[ \partial_z h_{\nu \alpha_2 \cdots \alpha_J} - \left(J-1\right) \dot{A} h_{\nu \alpha_2 \cdots \alpha_J} \right],  &B = \mu
\end{cases}
\end{equation}

There is only one more term left. For $B = z$ the third term becomes
\begin{equation}
2 g^{\mu \nu}_{\rm{eff.}} g^{zz}_{\rm{eff.}} \nabla_\mu \nabla_{[\nu}h_{z] \alpha_2 \cdots \alpha_J} = \frac{e^{-4A}}{G^2} \eta^{\mu \nu} \left( \nabla_\mu \nabla_\nu h_{z \alpha_2 \cdots \alpha_J} -  \nabla_\mu \nabla_z h_{\nu \alpha_2 \cdots \alpha_J}  \right)
\end{equation}
If we use the following covariant derivative identities
\begin{align}
&\nabla_\mu \nabla_\nu h_{z \alpha_2 \cdots \alpha_J} = -\dot{A} \left( \partial_\mu h_{\nu \alpha_2 \cdots \alpha_J} + \partial_\nu h_{\mu \alpha_2 \cdots \alpha_J} \right) \\
&\nabla_\mu \nabla_z h_{\nu \alpha_2 \cdots \alpha_J} = \partial_z \partial_\mu h_{\nu \alpha_2 \cdots \alpha_J} - \left(J+1\right) \dot{A} \partial_\mu h_{\nu \alpha_2 \cdots \alpha_J}
\end{align}
we conclude that this term is also 0. For $B = \mu$ the third term can be rewritten as
\begin{equation}
2 \frac{e^{-4A}}{G^2} \eta^{\mu \nu} \nabla_{z} \nabla_{[z} h_{\nu] \alpha_2 \cdots \alpha_J} + 2 e^{-4A} \eta^{\alpha \beta} \eta^{\mu \nu} \nabla_{\alpha} \nabla_{[\beta} h_{\nu] \alpha_2 \cdots \alpha_J}
\end{equation}
The following identities
\begin{align}
&\nabla_z \nabla_z h_{\nu \alpha_2 \cdots \alpha_J} = \left[ \partial_z^2 - \dot{A} \left(2 J+1\right) \partial_z - J \ddot{A} + J \left(J+1\right) {\dot{A}}^2 \right] h_{\nu \alpha_2 \cdots \alpha_J} \\
&\nabla_z \nabla_\nu h_{z \alpha_2 \cdots \alpha_J} = - \ddot{A} h_{\nu \alpha_2 \cdots \alpha_J} - \dot{A} \partial_z h_{\nu \alpha_2 \cdots \alpha_J} + \left(J+1\right){\dot{A}}^2 h_{\nu \alpha_2 \cdots \alpha_J} \\
& \nabla_\alpha \nabla_\beta h_{\nu \alpha_2 \cdots \alpha_J} = \partial_\alpha \partial_\beta h_{\nu \alpha_2 \cdots \alpha_J} + \dot{A} \eta_{\alpha \beta} \left( \partial_z h_{\nu \alpha_2 \cdots \alpha_J} - \dot{A} J h_{\nu \alpha_2 \cdots \alpha_J}\right) - {\dot{A}}^2 \eta_{\alpha \nu} h_{\beta \alpha_2 \cdots \alpha_J} - {\dot{A}}^2 \sum_{i=2}^J \eta_{\alpha \alpha_i} h_{\nu \cdots \alpha_{i-1} \beta \alpha_{i+1} \cdots \alpha_J}
\end{align}
give the following expressions for the third term
\begin{equation}
\begin{cases}
0 \, ,B = z \\
e^{-4A} \eta^{\mu \nu} \left[\frac{1}{G^2}\left(\partial_z^2 - 2 J \dot{A} \partial_z - \left(J-1\right) \ddot{A} + \left(J^2-1\right) {\dot{A}}^2 \right)+ \left( \Box + 3\dot{A} \partial_z - 3 {\dot{A}}^2 \left(J-1\right) \right)  \right]h_{\nu \alpha_2 \cdots \alpha_J} \, , B = \mu
\end{cases}
\end{equation}

Plugging the above calculations in equation (\ref{eq:spin_J_eom_cov_term}) we conclude that the EOM of the spin J fields on the meson trajectory should have the term
\begin{align}
&\frac{\partial_z \left(e^{-\frac{10}{3}\Phi} V_f w_s^2 G\right)}{e^{-\frac{10}{3}\Phi} V_f w_s^2 G} \frac{e^{-2A}}{G^2} \left[\partial_z - \left(J-1\right) \dot{A} \right]h_{\alpha_1 \cdots \alpha_J} + e^{-2A}\left( \frac{4 \dot{A}}{G^2} - 2 \frac{\dot{G}}{G^3} - 4 \dot{A} \right) \left[\partial_z - \left(J-1\right) \dot{A} \right]h_{\alpha_1  \cdots \alpha_J} + \notag \\
&+e^{-2A} \left[\frac{1}{G^2}\left(\partial_z^2 - 2 J \dot{A} \partial_z - \left(J-1\right) \ddot{A} + \left(J^2-1\right) {\dot{A}}^2 \right)+ \left( \Box + 3\dot{A} \partial_z - 3 {\dot{A}}^2 \left(J-1\right) \right)  \right]h_{\alpha_1 \cdots \alpha_J}
\end{align}
We can also add terms that i) reduce to the equation of the $U\left(1\right)$ gauge field; ii) vanish in the case of AdS, i.e. $\Phi = \rm{const.}$, $\tau = \rm{const.}$ and $A =  - \log(z)$; have dimension inverse squared length compatible with the previous two constraints. There are 9 possible terms satisfying the last constraing: $\ddot{\Phi}$, $\ddot{A}$, $\ddot{\tau}$, ${\dot{\Phi}}^2$, ${\dot{\tau}}^2$, ${\dot{A}}^2$, $\dot{\Phi} \dot{\tau}$, $\dot{A}\dot{\tau}$ and $\dot{\Phi}\dot{A}$. All these terms must be proportional to $J-1$ in order to satisfy constraint i). Moreover, due to constraint ii) there must be a term proportional to $\ddot{A} - \dot{A}^2$. Because the background is symmetric under $\tau \rightarrow -\tau$ we drop $\ddot{\tau}$, $\dot{\Phi}\dot{\tau}$ and $\dot{A}\dot{\tau}$. This means a term of the form
\begin{equation}
\left(J-1\right) e^{-2A} \left[ a \ddot{\Phi} + b \left(\ddot{A} - {\dot{A}}^2\right) + c {\dot{\Phi}}^2 + d {\dot{\tau}}^2 + e \dot{\Phi} \dot{A} \right] h_{\alpha_1 \cdots \alpha_J}
\end{equation}
must be present.

Finally we can also add a mass term related to the dimension of the dual operator, wich requires a length scale $L$. According to the AdS/CFT correspondence the mass of a vector bulk field $m^2$ is related to the conformal dimension $\Delta$ of the dual operator through
\begin{equation}
m^2 L^2 = \left(\Delta - 1 \right) \left(\Delta - 3 \right)
\end{equation}
This term is absent in the equation of motion of the $U\left(1\right)$ gauge field because this bulk field is dual to the operator $\bar{\psi} \gamma^\mu \psi$ which has $\Delta = 3$ and spin 1. This operator is also a conserved current and hence $\Delta = 3$ and $J = 1$ is a protected point. To extend the mass term for other spin J fields consider the quadratic approximation to the $J\left(\nu\right), \, i \nu = \Delta - 2$ curve
\begin{equation}
J\left(\nu\right) \approx J_0 - \mathcal{D} \nu^2\, , \mathcal{D} = 1 - J_0
\end{equation}
We can invert the above equation to find $\Delta$ as a function of $J$.  The result is given by
\begin{equation}
\left(\Delta -1\right) \left(\Delta - 3 \right) = \frac{1-J}{J_0 - 1}
\end{equation}
In $\mathcal{N} = 4$ SYM in the strong coupling and Regge limit the intercept $J_0$ is related to the coupling constant $\lambda$ through $J_0 = 2 - \frac{2}{\sqrt{\lambda}}$ and hence 
\begin{equation}
\frac{\left(\Delta -1\right) \left(\Delta - 3 \right)}{L^2} = \frac{\left(1-J\right)}{L^2} + \frac{2\left(1-J\right)}{L^2\sqrt{\lambda}}\left(1+\frac{2}{\sqrt{\lambda}}\right).
\end{equation}
Also, from $\mathcal{N} = 4$ SYM , $\sqrt{\lambda} = R^2 / l^2_s$ where $R$ is the AdS radius. Then
\begin{equation}
\frac{\left(\Delta -1\right) \left(\Delta - 3 \right)}{L^2} = -\frac{\left(J-1\right)}{L^2} - \frac{2 l_s^2}{R^2 L^2} \left(J-1\right) \left(1+\frac{2}{\sqrt{\lambda}}\right).
\end{equation}
We then propose the following addition to the equation of motion of the spin J fields in the meson trajectory
\begin{equation}
\frac{\left(\Delta -1\right) \left(\Delta - 3 \right)}{L^2} = -\frac{3\left(J-1\right)}{l_s^2} \left(1+\frac{d}{\sqrt{\lambda}}\right) + \frac{J^2 - 1}{e^{\frac{4}{3}\Phi}}
\end{equation}
The last term was included in order to get the correct UV behaviour of the spin J fields
\begin{align}
h^{TT}_{\alpha_1 \cdots \alpha_J} \sim c_1 z^2 + c_2 z^{2- 2J}
\end{align}

With all the above considerations we will use the following equation
\begin{align}
&\frac{\partial_z \left(e^{-\frac{10}{3}\Phi} V_f w_s^2 G\right)}{e^{-\frac{10}{3}\Phi} V_f w_s^2 G} \frac{e^{-2A}}{G^2} \left[\partial_z - \left(J-1\right) \dot{A} \right]h_{\alpha_1 \cdots \alpha_J} + e^{-2A}\left( \frac{4 \dot{A}}{G^2} - 2 \frac{\dot{G}}{G^3} - 4 \dot{A} \right) \left[\partial_z - \left(J-1\right) \dot{A} \right]h_{\alpha_1  \cdots \alpha_J} + \notag \\
&+e^{-2A} \left[\frac{1}{G^2}\left(\partial_z^2 - 2 J \dot{A} \partial_z - \left(J-1\right) \ddot{A} + \left(J^2-1\right) {\dot{A}}^2 \right)+ \left( \Box + 3\dot{A} \partial_z - 3 {\dot{A}}^2 \left(J-1\right) \right)  \right]h_{\alpha_1 \cdots \alpha_J} - \notag \\
& - \frac{\left(\Delta - 1 \right)\left(\Delta - 3\right)}{L^2} h_{\alpha_1 \cdots \alpha_J}^{TT} + \left(J-1\right) e^{-2A} \left[ a \ddot{\Phi} + b \left(\ddot{A} - {\dot{A}}^2\right) + c {\dot{\Phi}}^2 + d {\dot{\tau}}^2 + e \dot{\Phi} \dot{A} \right] h_{\alpha_1 \cdots \alpha_J} = 0
\label{eq:spin_j_meson_eom}
\end{align}

\subsection{Schrodinger form}
The equation (\ref{eq:spin_j_meson_eom}) can be brought to Schrodinger form by first rewriting it in terms of the u variable. After that we can write the spin J field as
\begin{equation}
h^{TT}_{\alpha_1 \cdots \alpha_J} = \epsilon_{\alpha_1 \cdots \alpha_J} e^{i q \cdot x} \frac{e^{\left(J-1\right)A}}{\sqrt{e^{-\frac{10}{3}\Phi}V_f w_s^2 e^A}} \psi\left(u\right),
\end{equation}
where $\psi$ satisfies the Schrodinger equation
\begin{equation}
- \frac{d^2 \psi}{du^2} + V_J \left(u\right) \psi = t \psi.
\end{equation}
The Schrodinger potential is given by
\begin{align}
&V_J\left(u\right) = V_1\left(u\right) + \left(J^2 - 1\right) e^{2 A - \frac{4}{3} \Phi} - \frac{3}{l_s^2} \left(J-1\right) e^{2A} \left(1 + \frac{f}{\sqrt{\lambda}}\right) - \notag \\
& - \left(J-1\right) \left[a \left( G \dot{G} \dot{\Phi} + G^2 \ddot{\Phi} \right) + b \left( G^2 \ddot{A} + G \dot{G} \dot{A} -G^2 \dot{A}^2\right) +c G^2 \dot{\Phi}^2 + d G^2 \dot{A}\dot{\Phi} + e G^2 \dot{\tau}^2 \right]
\label{eq:meson_schrodinger_potential}
\end{align}
Here the dots mean derivatives with respect to the u variable.
We will denote the eigenvalues of this Schrodinger potential as $t_n\left(J\right)$ and the corresponding eigenfunctions as $\psi_n\left(J, u\right)$.

\subsection{Spin J field propagator}
The propagator obeys equation of the type
\begin{equation}
\mathcal{D} \Pi_{a_1 \cdots a_J, b_1 \cdots b_J} = i g_{a_1(b_1} \cdots g_{|a_J|b_J)} \delta_5 \left(X, \bar{X}\right) \frac{G}{e^{-\frac{10}{3}\Phi}V_f w_s^2},
\end{equation}
where $\mathcal{D}$ is a differential operator defined by equaton (\ref{eq:spin_j_meson_eom}).
In the Regge limit we will be interested in the components $\Pi_{+\cdots+,-\cdots-}$ and for this particular case
\begin{equation}
\mathcal{D} \Pi_{+\cdots+,-\cdots-} = i {\left( - \frac{e^{2A}}{2}\right)}^J  \delta_5 \left(X, \bar{X}\right) \frac{G}{e^{-\frac{10}{3}\Phi}V_f w_s^2}
\end{equation}
Consider now that
\begin{equation}
\int \frac{dw^+ dw^-}{2} \Pi_{+\cdots+, - \cdots -} = -i {\left(-\frac{1}{2}\right)}^J e^{\left(J-1\right)(A+\bar{A})} G_J \left(u, \bar{u}, l_\perp\right)
\end{equation}
Applying the operator $\mathcal{D}$ on the left-hand side of this equation gives
\begin{equation}
\int \frac{dw^+ dw^-}{2} \mathcal{D} \Pi_{+\cdots+, - \cdots -} = \frac{i}{{\left(-2\right)}^J} e^{\left(J-1\right)\left(A+\bar{A}\right)} \delta_3\left(x, \bar{x}\right) \frac{G}{e^{-\frac{10}{3}\Phi}V_f w_s^2}.
\end{equation}
Applying the same operator on the right-hand side gives
\begin{align}
&\mathcal{D} \left[ e^{\left(J-1\right) \left(A+\bar{A}\right)} G_J \left(u, \bar{u}, l_\perp\right) \right] = e^{\left(J-1\right)\left(A+\bar{A}\right)} \mathcal{D}_3 G_J \left(u, \bar{u}, l_\perp\right) \notag \\
&\mathcal{D}_3 = e^{-2A} \partial_u^2 + e^{-2A} \left( \dot{A} - \frac{10}{3} \dot{\Phi} + 2 \frac{w_s' \dot{\Phi}}{w_s} + \frac{\partial_u V_f}{V_f} \right) \partial_u + e^{-2A} \partial^2_{l_\perp} + \notag \\
& \left(J-1\right) e^{-2A} \left[ a \left(G^2\ddot{\Phi} + G \dot{G} \dot{\Phi} \right) + b \left(G^2 \ddot{A} + G \dot{G} \dot{A} - \dot{G}^2 {\dot{A}}^2\right) + c G^2 {\dot{\Phi}}^2 + d G^2 {\dot{\tau}}^2 + e G^2 \dot{\Phi} \dot{A} \right] + \notag \\
& + \frac{3}{l_s^2} \left(J-1\right)\left(1+\frac{d}{\sqrt{\lambda}}\right) - \frac{J^2-1}{\lambda^{4/3}}
\end{align}
The RHS then becomes
\begin{equation}
- \frac{i}{{\left(-2\right)}^J} e^{\left(J-1\right)\left(A+\bar{A}\right)} \mathcal{D}_3 G_J \left(u, \bar{u}, l_\perp\right)
\end{equation}
We conclude that $G_J$ satisfies the following equation
\begin{equation}
\mathcal{D}_3 G_J \left(u, \bar{u}, l_\perp\right) = - \delta_3 \left(x, \bar{x}\right) \frac{G}{e^{-\frac{10}{3} \Phi} V_f w_s^2}
\label{eq:transverse_propagator}
\end{equation}

To solve equation (\ref{eq:transverse_propagator}) consider
\begin{equation}
\mathcal{D}_3 G_J\left(u, l_\perp\right) = 0
\end{equation}
and the following ansatz
\begin{equation}
G_J\left(u, l_\perp\right) = \frac{e^{i q \cdot l_\perp}}{\sqrt{e^{-\frac{10}{3}\Phi}V_f w_s^w e^{A}}} \psi\left(u\right).
\end{equation}
It follows
\begin{equation}
- \frac{d^2 \psi}{d u^2} + V_J \left(u\right) \psi = t \, \psi
\end{equation}
where $V_J$ is the same as in equation (\ref{eq:meson_schrodinger_potential}). Since the eigenfunctions of the Schrodinger potential satisfy $\sum_n \psi_n (u) {\psi_n (\bar{u})}^* = \delta(u - \bar{u})$ once can show
\begin{equation}
\left( - \frac{d^2}{du^2} + V - t \right) \sum_n \frac{\psi_n\left(u\right) {\psi_n\left(\bar{u}\right)}^*}{t_n \left(J\right) - t} = \delta\left(u - \bar{u} \right)
\end{equation}

With the above considerations one can show that the solution to equation (\ref{eq:transverse_propagator}) is
\begin{equation}
G_J \left(u, \bar{u}, t \right) = \left. \frac{1}{\sqrt{e^{-\frac{10}{3}\Phi}V_f w_s^2 e^A}}\right|_u  \left.\frac{1}{\sqrt{e^{-\frac{10}{3}\Phi}V_f w_s^2 e^A}}\right|_{\bar{u}}  \sum_n \frac{\psi_n\left(u\right) \psi_n \left(\bar{u}\right)}{t_n\left(J\right) - t}
\end{equation}

\subsection{Couplings to the spin J fields in the meson trajectory}

Before computing the holographic contribution of the holographic meson trajectory to scattering amplitudes we need to specify the couplings of the external states with the spin J fields of the meson trajectory. We assume such fields come from the open string sector and hence the coupling between a $U(1)$ gauge field  with these fields is given by
\begin{equation}
k_J \int d^5X \sqrt{-g} e^{-\Phi} \left(F_{b_1 a} \nabla_{b_2} \cdots \nabla_{b_{J-1}} F^{a}_{\, \, \, b_J}  \right) h^{b_1 \cdots b_J}
\label{eq:open_string_spin_J_coupling}
\end{equation}
We will also assume that a scalar field $\Upsilon$ coming from the open string sector is dual to the proton target and because of that we assume the substitution
\begin{equation}
\bar{k}_J \int d^5 X \sqrt{-g} e^{-\Phi} h_{b_1 \cdots b_J} \left( \Upsilon \nabla^{b_1} \cdots \nabla^{b_J} \Upsilon \right)
\end{equation}
In the Regge limit only the TT components of the spin J fields will be important and hence the previous couplings reduce to 
\begin{align}
&k_J \int d^5X \sqrt{-g} e^{-\Phi} \left(F_{\beta_1 \alpha} \nabla_{\beta_2} \cdots \nabla_{\beta_{J-1}} F^{\alpha}_{\, \, \, \beta_J}  \right) h^{\beta_1 \cdots \beta_J} \\
&\bar{k}_J \int d^5 X \sqrt{-g} e^{-\Phi}  \left( \Upsilon \nabla_{\beta_1} \cdots \nabla_{\beta_J} \Upsilon \right)h^{\beta_1 \cdots \beta_J}
\label{eq:meson_spin_J_couplings}
\end{align}
Another possibility is to assume that the coupling of the $U\left(1\right)$ gauge field with the spin J fields of the meson trajectory is the same as in equation (\ref{eq:spin_J_graviton_coupling}).

%%%%%%%%%%%%%%%%%%%%%%%%%%%%%%%%%%%%%%
\section{Kinematics}
%%%%%%%%%%%%%%%%%%%%%%%%%%%%%%%%%%%%%%
We will use light-cone coordinates $\left( +, -, \perp \right)$, with metric given by $ds^2 = - dx^{+} dx^{-} + dx_\perp^2$, where $x_\perp \in \mathbb{R}^2$ is a vector in impact parameter space.
For the large $s$ the kinematics of hadron-hadron scattering are the following
\begin{align}
\label{eq:pp_kinematics}
&k_1=\left(\!\sqrt{s},\frac{m_1^2}{\sqrt{s}} ,0\right),\  \ k_3=-\left(\!\sqrt{s},\frac{ q_\perp^2 +m_1^2}{\sqrt{s}} , q_\perp \right)\!,\\
&k_2=\left(\frac{m_2^2}{\sqrt{s}},\sqrt{s} ,0\right),\  \ k_4=-\left(\frac{m_2^2+ q_\perp^2}{\sqrt{s}},\sqrt{s} ,-q_\perp \right).\notag
\end{align}
where $k_1$ and $k_2$ are the incoming hadron momenta and $k_3$ and $k_4$ are the outgoing hadron momenta.

For the case of $\gamma^{*} p$ scattering the expression for $k_2$ and $k_4$ are the same. However the momenta of the incoming off-shell photon $k_1$ and the momenta of the outgoing off-shell photon $k_3$ are given by
\begin{align}
\label{eq:gammap_kinematics}
k_1=\left(\!\sqrt{s},-\frac{Q^2}{\sqrt{s}} ,0\right),\  \ k_3=-\left(\!\sqrt{s},\frac{ q_\perp^2 -Q^2}{\sqrt{s}} , q_\perp \right)\!.\\
\end{align}
The corresponding polarisation vectors of the incoming and outgoing photons are respectively
\begin{align}
&n_1=
    \begin{cases}
      \left(0,0,1,0\right), & \lambda=1 \\
      \left(0,0,0,1\right), & \lambda=2 \\
      \frac{1}{Q} \left( \sqrt{s}, \frac{Q^2}{\sqrt{s}}, 0, 0 \right), & \lambda = 3
    \end{cases} \\
 &n_3=
    \begin{cases}
      \left(0,\frac{2 q_x}{\sqrt{s}},1,0\right), & \lambda=1 \\
      \left(0,\frac{2 q_y}{\sqrt{s}},0,1\right), & \lambda=2 \\
      \frac{1}{Q} \left(\sqrt{s}, \frac{Q^2+q_\perp^2}{\sqrt{s}}, q_\perp \right), & \lambda = 3
    \end{cases}   
\end{align}

\section{$pp$ scattering with holographic pomeron and meson exchange}
We now have all the ingredients to compute holographic scattering amplitudes. We start the hadron-hadron scattering case by considering general spin J exchange. Later we specify for the cases of fields belonging either to the graviton's Regge trajectory or to the vector meson trajectory. The scattering amplitude for spin J exchange is
\begin{equation}
\mathcal{A}_J = {\left(i k'_J\right)}^2 \int d^5X d^5\bar{X} \sqrt{-g} \sqrt{-\bar{g}} e^{- \Phi -\bar{\Phi}} \left(\Upsilon_1 \partial_{-}^J \Upsilon_3\right) \Pi^{-\cdots-, + \cdots +} \left(X, \bar{X}\right) \left(\bar{\Upsilon}_2 \bar{\partial}_{+}^J \bar{\Upsilon}_4\right)
\end{equation}
Lowering the indices of the propagator gives
\begin{equation}
\mathcal{A}_J = {\left(i k'_J\right)}^2 \int d^5X d^5\bar{X} \sqrt{-g} \sqrt{-\bar{g}} e^{- \Phi -\bar{\Phi}} {\left(-2 e^{-2 A }\right)}^J {\left(-2 e^{-2 \bar{A} }\right)}^J \left(\Upsilon_1 \partial_{-}^J \Upsilon_3\right) \Pi_{+\cdots+, - \cdots -} \left(X, \bar{X}\right) \left(\bar{\Upsilon}_2 \bar{\partial}_{+}^J \bar{\Upsilon}_4\right)
\end{equation}
Using the fact that $\Upsilon_i = \upsilon\left(z\right) e^{i k_i \dot x}$ and the kinematics of equation (\ref{eq:pp_kinematics}) we get
\begin{equation}
\mathcal{A}_J = - k_J^2 {\left(-1\right)}^J s^J \int \frac{d^5X}{2}\frac{d^5\bar{X}}{2} e^{5\left(A+\bar{A}\right)} e^{-\Phi - \bar{\Phi}} e^{-2J\left(A+\bar{A}\right)} {|\upsilon_1|}^2 {|\upsilon_2|}^2 e^{-i q_\perp \cdot \left(x_\perp - \bar{x}_\perp\right)} \Pi_{+ \cdots +, - \cdots -} \left(X, \bar{X}\right)
\end{equation}
Making the change of variable $w = x - \bar{x}$  and using the identity
\begin{equation}
\int d^2 l_\perp e^{- i q_\perp \cdot l_\perp}\int \frac{dw^+ dw^-}{2} \Pi_{+\cdots+, - \cdots -} \left(X, \bar{X}\right) = - \frac{i}{\left(-2\right)^J} e^{\left(J-1\right)\left(A+\bar{A}\right)} G_J \left(z, \bar{z}, t\right),
\end{equation}
the scattering amplitude can be rewritten as
\begin{align}
\mathcal{A}_J = i V \frac{k_J^2}{2^J} s^J \int dz d\bar{z} e^{4\left(A+\bar{A}\right)} e^{-J\left(A+\bar{A}\right)} e^{-\Phi - \bar{\Phi}} {|\upsilon_1|}^2 {|\upsilon_2|}^2  G_J \left(z, \bar{z}, t\right).
\end{align}
For the case of a spin J of the graviton's Regge trajectory
\begin{equation}
G_J \left(z, \bar{z}, t \right) = e^{\Phi + \bar{\Phi} - \frac{A+\bar{A}}{2}} \sum_n \frac{\psi_n \left(z\right) {\psi_n \left(\bar{z}\right)}^*}{t_n(J) - t},
\end{equation}
while for the case of a spin J in the vector meson trajectory we have
\begin{equation}
G_J \left(z, \bar{z}, t \right)  = \left.\frac{1}{\sqrt{e^{-\frac{10}{3}\Phi} V_f w_s^2 e^A}}\right|_z  \left.\frac{1}{\sqrt{e^{-\frac{10}{3}\Phi} V_f w_s^2 e^A}}\right|_{\bar{z}} \, \sum_n \frac{\psi_n \left(u\left(z\right)\right){\psi_n \left(u\left(\bar{z}\right)\right)}^*}{t_n(J)-t}
\end{equation}

Let's consider each case separately and start with the graviton case. We sum all even spin J exchanges with $J \geqslant 2$ using the Sommerfeld-Watson transform
\begin{equation}
\frac{1}{2} \sum_{J \geqslant 2} \left( s^J + {\left(-s\right)}^J \right) \rightarrow - \frac{\pi}{2} \int \frac{dJ}{2 \pi i} \frac{s^J + {\left(-s\right)}^J}{\sin\left(\pi J\right)}
\end{equation}
where we are assuming the analytic continuation of the scattering amplitude $\mathcal{A}_J$ to the complex J-plane. Deforming the J-plane integral and catching all the poles $J = j_n\left(t\right)$ defined by $t_n\left(J\right) ) = t$ we get
\begin{align}
\mathcal{A}^{\rm {gluon}} = - \frac{\pi}{2} \sum_n \frac{k_{j_n}^2}{2^{j_n}} s^{j_n} \left[i + \cot \left( \frac{\pi j_n}{2} \right) \right] \frac{d j_n}{dt} \int dz d\bar{z} e^{-\left(j_n - \frac{7}{2}\right) \left(A+\bar{A}\right)} {|\upsilon_1|}^2 {|\upsilon_2|}^2 \psi_n \left(z\right) {\psi_n \left(\bar{z}\right)}^*
\end{align}
In the scattering domain of $t < 0$ these poles are in the real axis for $J < 2$. 

For the case of the $\rho$ meson trajectory we do a similar procedure. We sum all odd spin J exchanges with $J \geqslant 1$ using the Sommerfeld-Watson transform
\begin{equation}
\frac{1}{2} \sum_{J \geqslant 1} \left( s^J - {\left(-s\right)}^J \right) \rightarrow - \frac{\pi}{2} \int \frac{dJ}{2 \pi i} \frac{s^J - {\left(-s\right)}^J}{\sin\left(\pi J\right)}
\end{equation}
and again we deform the J-plane integral from the poles at odd values of J to poles $J = j_n\left(t\right)$ again defined through $t_n\left(J\right) = t$. However this time these poles are along the real axis for $J < 1$.
The contribution of the holographic meson trajectory is then
\begin{equation}
\mathcal{A}^{\rm {meson}} = - \frac{\pi}{2} \sum_n \frac{k_{j_n}^2}{2^{j_n}} s^{j_n} \left[-i + \tan \left( \frac{\pi j_n}{2} \right) \right] \frac{d j_n}{dt} \int dz d\bar{z} \frac{e^{-\left(j_n - 4\right) \left(A+\bar{A}\right)-\Phi - \bar{\Phi}}}{\Xi\left(z\right) \Xi\left(\bar{z}\right)} {|\upsilon_1|}^2 {|\upsilon_2|}^2 \psi_n \left(u\left(z\right)\right) {\psi_n \left(u\left(\bar{z}\right)\right)}^*
\end{equation}

From previous work we know that what people call the hard-pomeron corresponds to the $n=0$ contribution from $\mathcal{A}^{gluon}$ and the soft-pomeron is the $n=1$ constribution from $\mathcal{A}^{gluon}$. These trajectories dominate hadron-hadron scattering at very high center-of-mass energies. For lower energies it is necessary to include a meson trajectory which has an intercept lower than 1. We identify this trajectory as the $n=0$ term from $\mathcal{A}^{\rm {meson}}$. Using the optical theorem the total cross-section is given by
\begin{equation}
\sigma\left(h h \rightarrow X\right) = {\rm Im} \, g_0 s^{j_0 - 1} +  {\rm Im} \, g_1 s^{j_1 - 1} +  {\rm Im} \, g_\rho s^{j_\rho - 1}
\end{equation}

\section{Contribution of the meson exchange to $F^p_2$ and $F^p_L$}

The holographic amplitude due to the exchange of spin J fields for $\gamma^* p$
\begin{equation}
\mathcal{A}_J = - k_J \bar{k}_J \int d^5X d^5 \bar{X} \sqrt{-g} \sqrt{-\bar{g}} e^{-\Phi -\bar{\Phi}} F^{(1)}_{- a} \partial_{-}^{J-2} F^{a (3)}_{\, \, \, -} \Pi^{- \cdots -, + \cdots +} \left(X, \bar{X}\right) \Upsilon^{(2)} \bar{\partial}^J_{+} \Upsilon^{(4)}
\end{equation}
Because $A_z = 0$ and $A_\mu = n_\mu f_Q e^{i k \cdot x}$, $F_{z\mu} = n_\mu \dot{f}_Q e^{i k \cdot x}$ and $F_{\mu\nu} = 2 i k_{[\mu}n_{\nu]} f_Q e^{i k \cdot x}$
the expression for the scattering amplitude due to spin J exchange is
\begin{align}
\mathcal{A}_J = - k_J \bar{k}_J s^J \int d^5X d^5\bar{X} \sqrt{-g} \sqrt{-\bar{g}} e^{- \Phi - \bar{\Phi}} e^{-2A} e^{-2J\left(A+\bar{A}\right)} &\left[ \left( \epsilon_{\lambda_1} \cdot \epsilon_{\lambda_3}\right) f_Q^2 \left(1-\delta_{\lambda_1,3}\right) \left(1-\delta_{\lambda_3,3}\right) +  \right. \notag \\
&\left. + \delta_{\lambda_1,3} \delta_{\lambda_3, 3} \frac{\dot{f_Q}^2}{Q^2}\right] |\upsilon_p|^2 e^{-i q_\perp \cdot l_\perp} \Pi_{+ \cdots +, - \cdots -} \left(X, \bar{X}\right)
\end{align}
Making the change of variable $w = x - \bar{x}$  and using the identity
\begin{equation}
\int d^2 l_\perp e^{- i q_\perp \cdot l_\perp}\int \frac{dw^+ dw^-}{2} \Pi_{+\cdots+, - \cdots -} \left(X, \bar{X}\right) = - \frac{i}{\left(-2\right)^J} e^{\left(J-1\right)\left(A+\bar{A}\right)} G_J \left(z, \bar{z}, t\right),
\end{equation}
the scattering amplitude can be rewritten as
\begin{align}
&\mathcal{A}_J = i V \frac{k_J \bar{k}_J}{2^J} s^J \int dz d\bar{z} e^{-\Phi -\bar{\Phi}} e^{-J\left(A+\bar{A}\right)} e^{4\left(A+\bar{A}\right)} e^{-2A} \times  \notag \\
&\left[ \left( \epsilon_{\lambda_1} \cdot \epsilon_{\lambda_3}\right) f_Q^2 \left(1-\delta_{\lambda_1,3}\right) \left(1-\delta_{\lambda_3,3}\right) + \delta_{\lambda_1,3} \delta_{\lambda_3, 3} \frac{\dot{f_Q}^2}{Q^2}\right]  |\upsilon_p|^2 G_J \left(z, \bar{z}, t\right)
\end{align}
 where for the vector meson trajectory
\begin{equation}
G_J \left(z, \bar{z}, t \right)  = \left.\frac{1}{\sqrt{e^{-\frac{10}{3}\Phi} V_f w_s^2 e^A}}\right|_z  \left.\frac{1}{\sqrt{e^{-\frac{10}{3}\Phi} V_f w_s^2 e^A}}\right|_{\bar{z}} \, \sum_n \frac{\psi_n \left(u\left(z\right)\right){\psi_n \left(u\left(\bar{z}\right)\right)}^*}{t_n(J)-t}.
\end{equation}
Now we need to sum over all even spin J fields with $J \geqslant 2$ and then perform the deformation of the complex-J integral.
Defining
\begin{equation}
g_n\left(t\right) = - \frac{\pi}{2} \left[i + \cot\left(\frac{\pi j_n}{2^{j_n}}\right)\right] \frac{k_{j_n} \bar{k}_{j_n}}{2^{j_n}} \frac{d j_n}{dt} \int d\bar{z} \frac{e^{-\bar{\Phi}}e^{-\left(j_n-4\right)\bar{A}}}{\Xi\left(\bar{z}\right)} |\upsilon_p|^2  {\psi_n\left(u\left(z\right)\right)}^*,
\end{equation}
the scattering amplitude is given by
\begin{equation}
\mathcal{A}^{\lambda_1, \lambda_3} \left(s, t\right) = \sum_n g_n\left(t\right) s^{j_n} \int dz \frac{e^{-\Phi} e^{-\left(j_n-2\right)A}}{\Xi\left(z\right)} \left[ \left( \epsilon_{\lambda_1} \cdot \epsilon_{\lambda_3}\right) f_Q^2 \left(1-\delta_{\lambda_1,3}\right) \left(1-\delta_{\lambda_3,3}\right) + \delta_{\lambda_1,3} \delta_{\lambda_3, 3} \frac{\dot{f_Q}^2}{Q^2}\right] \psi_n \left(u\left(z\right)\right)
\end{equation}

In the literature one finds the proton structure functions $F_2$ and $F_L$ defined in terms of the transverse and longitudinal total cross-sections of the process $\gamma^{*} p$
\begin{align}
&F_2\left(x, Q^2\right) = \frac{Q^2}{4 \pi^2 \alpha} \left(\sigma_T + \sigma_L\right) \\
&F_L\left(x, Q^2\right) = \frac{Q^2}{4 \pi^2 \alpha} \sigma_L
\end{align}
Using the optical theorem and the last relations one finds the contribution of the holographic meson trajectory to the structure functions 
\begin{align}
&F_2\left(x, Q^2\right) = \sum_n \frac{{\rm Im} g_n }{4\pi^2 \alpha} Q^{2 j_n} x^{1-j_n} \int dz \frac{e^{-\left(j_n - 2 \right)A - \Phi}}{\Xi\left(z\right)} \left(f_Q^2 +\frac{\partial_z{f_Q}^2}{Q^2} \right) \psi_n\left(u\left(z\right)\right)\\
&F_L\left(x, Q^2\right) = \sum_n \frac{{\rm Im} g_n }{4\pi^2 \alpha} Q^{2 j_n} x^{1-j_n} \int dz \frac{e^{-\left(j_n - 2 \right)A - \Phi}}{\Xi\left(z\right)} \frac{\partial_z{f_Q}^2}{Q^2} \psi_n\left(u\left(z\right)\right)
\end{align}
Changing to the u variable the above expressions become
\begin{align}
&F_2\left(x, Q^2\right) = \sum_n \frac{{\rm Im} g_n }{4\pi^2 \alpha} Q^{2 j_n} x^{1-j_n} \int du \frac{e^{-\left(j_n - 2 \right)A - \Phi}}{G\Xi\left(u\right)} \left(f_Q^2 +\frac{G^2\partial_u{f_Q}^2}{Q^2} \right) \psi_n\left(u\right)\\
&F_L\left(x, Q^2\right) = \sum_n \frac{{\rm Im} g_n }{4\pi^2 \alpha} Q^{2 j_n} x^{1-j_n} \int du \frac{e^{-\left(j_n - 2 \right)A - \Phi}}{G\Xi\left(u\right)} \frac{G^2\partial_u{f_Q}^2}{Q^2} \psi_n\left(u\right)
\end{align}

Another possibility is instead of using the coupling of equation (\ref{eq:open_string_spin_J_coupling}) we use the one of equation (\ref{eq:spin_J_graviton_coupling}) for the upper part of the Witten diagram. The corresponding expressions are
\begin{align}
&F_2\left(x, Q^2\right) = \sum_n \frac{{\rm Im} g_n }{4\pi^2 \alpha} Q^{2 j_n} x^{1-j_n} \int dz \frac{e^{-\left(j_n - 2 \right)A }}{\Xi\left(z\right)}e^{-\frac{10}{3} \Phi} V_f w_s^2 G \left(f_Q^2 +\frac{\partial_z{f_Q}^2}{G^2Q^2} \right) \psi_n\left(u\left(z\right)\right)\\
&F_L\left(x, Q^2\right) = \sum_n \frac{{\rm Im} g_n }{4\pi^2 \alpha} Q^{2 j_n} x^{1-j_n} \int dz \frac{e^{-\left(j_n - 2 \right)A }}{\Xi\left(z\right)}e^{-\frac{10}{3} \Phi} V_f w_s^2 G \frac{\partial_z{f_Q}^2}{G^2Q^2} \psi_n\left(u\left(z\right)\right)
\end{align}
or in terms of the u variable
\begin{align}
&F_2\left(x, Q^2\right) = \sum_n \frac{{\rm Im} g_n }{4\pi^2 \alpha} Q^{2 j_n} x^{1-j_n} \int du e^{-\left(j_n - 2 \right)A }\sqrt{e^{-\frac{10}{3} \Phi} V_f w_s^2} \left(f_Q^2 +\frac{\partial_u{f_Q}^2}{Q^2} \right) \psi_n\left(u\right)\\
&F_L\left(x, Q^2\right) = \sum_n \frac{{\rm Im} g_n }{4\pi^2 \alpha} Q^{2 j_n} x^{1-j_n} \int du e^{-\left(j_n - 2 \right)A }\sqrt{e^{-\frac{10}{3} \Phi} V_f w_s^2} \frac{\partial_u{f_Q}^2}{Q^2} \psi_n\left(u\right)
\end{align}
These last two expressions are the ones I will use from now on to fit the DIS structure functions $F_2$ and $F_L$.

The structure function $F_2$ is related to the total cross-section $\sigma\left(\gamma p \rightarrow X\right)$ through
\begin{equation}
\sigma\left(\gamma p \rightarrow X\right) = 4 \pi^2 \alpha \lim_{Q^2 \rightarrow 0} \frac{F_2\left(x, Q^2\right)}{Q^2}.
\end{equation}
In the limit of $Q^2 \rightarrow 0$
\begin{equation}
\lim_{Q\rightarrow 0} f_Q = 1 , \quad \lim_{Q\rightarrow 0} \frac{\dot{f}_Q}{Q} = 0
\end{equation}
and it follows that
\begin{equation}
\sigma\left(\gamma p \rightarrow X\right) = \sum_n {\rm Im} g_n \,  s^{j_n - 1} \int du e^{-\left(j_n - 2 \right)A }\sqrt{e^{-\frac{10}{3} \Phi} V_f w_s^2} \,  \psi_n\left(u\right).
\end{equation}


\section{$\gamma^{*} \gamma$ processes}

In this section we derive the holographic expressions for $F_2^\gamma$ and $\sigma\left(\gamma \gamma \rightarrow X\right)$ in the context of Holographic QCD in the Veneziano limit. Taking into account only pomeron exchange the expression for $F_2^\gamma$ is
\begin{equation}
F_2^{\gamma \, {\rm pomeron}}\left(x, Q^2\right) = \sum_n \frac{{\rm Im} \, g_n}{4 \pi^2 \alpha} Q^{2 j_n} x^{1-jn} \int du e^{A(\frac{3}{2} - j_n)}  e^{-\frac{7}{3} \Phi}  V_f \left( \lambda, \tau \right) {w_s\left(\lambda \right)}^2  \left(  {f_Q}^2 + \frac{{\partial_u f_Q}^2}{Q^2}  \right) \psi_n \left(j_n, u\right)
\end{equation}
but the definition of the $g_n$s is
\begin{equation}
g_n = - \frac{\pi}{2} \frac{k^2_{j_n}}{2^{j_n}}  \left(i + \cot \frac{\pi j_n}{2} \right) \frac{d j_n}{dt}  \int du e^{A(\frac{3}{2} - j_n)}  e^{-\frac{7}{3} \Phi}  V_f \left( \lambda, \tau \right) {w_s\left(\lambda \right)}^2  \psi_n \left(j_n, u\right)^*
\end{equation}
The result of the contribution the meson exchange to $F_2^\gamma$ is the same as in $F_2^p$
\begin{equation}
F_2^{\gamma \, {\rm meson}}\left(x, Q^2\right) = \sum_n \frac{{\rm Im} g_n }{4\pi^2 \alpha} Q^{2 j_n} x^{1-j_n} \int du e^{-\left(j_n - 2 \right)A }\sqrt{e^{-\frac{10}{3} \Phi} V_f w_s^2} \left(f_Q^2 +\frac{\partial_u{f_Q}^2}{Q^2} \right) \psi_n\left(u\right)
\end{equation}
with the $g_n$s given by
\begin{equation}
g_n = - \frac{\pi}{2} \frac{k^2_{j_n}}{2^{j_n}}  \left(i + \cot \frac{\pi j_n}{2} \right) \frac{d j_n}{dt}  \int du e^{-\left(j_n - 2 \right)A }\sqrt{e^{-\frac{10}{3} \Phi} V_f w_s^2} \psi_n\left(u\right)
\end{equation}

The total cross-section $\sigma\left(\gamma \gamma \rightarrow X\right)$ is related to the structure function $F^\gamma_2$ through
\begin{equation}
\sigma\left(\gamma \gamma \rightarrow X\right) =  4 \pi^2 \alpha \lim_{Q^2 \rightarrow 0} \frac{F_2^\gamma}{Q^2}
\end{equation}
and hence the contributions of pomeron exchange and meson exchange to this observable are, respectively
\begin{align}
&\sigma^{\rm pomeron}\left(\gamma \gamma \rightarrow X\right) =  \sum_n {\rm Im} \, g_n s^{jn-1} \int du e^{A(\frac{3}{2} - j_n)}  e^{-\frac{7}{3} \Phi}  V_f \left( \lambda, \tau \right) {w_s\left(\lambda \right)}^2  \psi_n \left(j_n, u\right) \\ 
&\sigma^{\rm meson}\left(\gamma \gamma \rightarrow X\right) = \sum_n {\rm Im} g_n s^{j_n - 1} \int du e^{-\left(j_n - 2 \right)A }\sqrt{e^{-\frac{10}{3} \Phi} V_f w_s^2}  \psi_n\left(u\right)
\end{align}
Of course $\sigma\left(\gamma \gamma \rightarrow X\right) = \sigma^{\rm pomeron}\left(\gamma \gamma \rightarrow X\right) + \sigma^{\rm meson}\left(\gamma \gamma \rightarrow X\right)$.




















































\bibliographystyle{elsarticle-num}
\bibliography{bib/HVQCD.bib}

\end{document}