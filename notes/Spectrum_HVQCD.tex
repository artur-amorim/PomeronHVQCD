\documentclass[a4paper,12pt]{article}
\pdfoutput=1 
\usepackage{jheppub}
\usepackage{amssymb}
\usepackage{amsfonts}
\usepackage{amsmath}
\usepackage{color}
\usepackage{ulem}
\usepackage[portuguese, english]{babel}
\usepackage[utf8]{inputenc}
\usepackage[T1]{fontenc}

\newcommand{\overbar}[1]{\mkern 1.5mu\overline{\mkern-1.5mu#1\mkern-1.5mu}\mkern 1.5mu}


\title{Regge theory in a Holographic dual of QCD in the Veneziano Limit.}
\author{Artur Amorim, Miguel Costa, Matti Jarvinen}
\date{}

\begin{document}
\maketitle

%%%%%%%%%%%%%%%%%%%%%%%%%%%%%%%%%%%%%
\section{Holographic dual of QCD in the Veneziano Limit}
%%%%%%%%%%%%%%%%%%%%%%%%%%%%%%%%%%%%%

Quantum Chromodynamics is the widely accepted quantum theory of the strong interactions between quarks and gluons inside the proton and neutron. It consists of gauge fields (gluons) in the adjoint representation of $SU(N_c)$, with $N_c = 3$, coupled to $N_f$ fermions (quarks) in the fundamental representation of $SU(3)$. This theory can be generalised and studied in the $(N_c, N_f)$ plane by considering $SU(N_c)$ Yang-Mills coupled to $N_f$ Dirac fermions in the fundamental representation. This generalisation been studied in great depth in the 't Hooft large-$N_c$ limit  where $N_c \rightarrow \infty$ and $\lambda = g_{\rm{YM}}^2 N_c$ and $N_f$ is kept fixed. This limit is also known as the "quenched" limit since quark contributions are suppressed in powers of $\frac{N_f}{N_c} \rightarrow 0$.
Another interesting large-$N_c$ limit is the Veneziano limit where
\begin{align}
N_c \rightarrow \infty \quad , \quad N_f \rightarrow \infty \quad , \quad \frac{N_f}{N_c} = x \quad {\rm fixed} \quad , \quad \lambda = g_{\rm{YM}}^2 N_c \quad {\rm fixed}
\end{align}

A holographic dual that reproduces the features of QCD in the Veneziano limit was presented in \cite{Jarvinen:2011qe}. It consists of a system of a dilaton and tachyon coupled to five-dimensional gravity. The dilaton action is the same as in the Improved Holographic QCD (IHQCD) model presented in \cite{gursoy_exploring_2008, gursoy_exploring_2008-1}. The exponential of the dilaton $\lambda = e^\Phi$ is dual to the $\mathcal{T}r F^2$ operator of the field theory with its background value equal to the 't Hooft coupling.
The metric ansatz is
\begin{equation}
\label{eq: metric definition}
ds^2 = e^{2 A} \left( d z^2 + \eta_{\mu \nu} dx^\mu dx^\nu \right),
\end{equation}
where the warp factor $A$ is identified with the logarithm of the energy scale in the field theory at the boundary.
In IHQCD the dynamics of the dilaton is set by its potential $V_g(\lambda)$. The small $\lambda$ and large $\lambda$ asymptotic forms of the dilaton potential are constrained in order to reproduce the YM $\beta$-function, to obtain confinement and no bad singularities.
The action of the dilaton coupled to gravity is
\begin{equation}
S_g =  M_p^3 N_c^2 \int d^5X \sqrt{g} \left[ R - \frac{4}{3} \frac{{\left( \partial \lambda \right)}^2}{\lambda^2}+ V_g \left( \lambda\right) \right]\, ,
\label{eq:sg_action}
\end{equation}
where $M_p$ is the five-dimensional Plack scale and $N_c$ is the number of colours.
In this work we will adopt $V_g$ to be
\begin{align}
&V_g(\lambda) = 12 + V_1 \, \lambda + V_2 \, \frac{\lambda^2}{1 + \frac{\rm{sc} \, \lambda}{\lambda_0}} + 3 \rm{V_gIR} \, e^{-\frac{\lambda_0}{\rm {sc} \, \lambda}}  \frac{\lambda^{4/3}}{4 \pi^{8/3}}\sqrt{\log\left(1+\frac{\rm{sc} \, \lambda}{\lambda_0}\right)} \, , \\ \notag
& V_1 = \frac{44}{9 \pi^2} \, , V_2 = \frac{4619}{3888 \pi^4} \, , \lambda_0 = 8 \pi^2 \, .
\end{align}
The values of $V_1$ and $V_2$ are fixed by the pure Yang-Mills $\beta$-function while the parameters $\rm{sc}$ and $\rm{V_gIR}$ will be fitted to the zero-temperature spectrum, in particular by computing meson and tensor glueball mass ratios and comparing them to lattice and experimental results. The spectrum of the theory can be computed by computing the action of the quadratic fluctuations around the background solution followed by the reduction to the four-dimensional dynamics. For example, in the case of pure Yang-Mills (i.e. $x = 0$), the quadratic fluctuations are dual to glueballs with quantum numbers $J^{PC} = 0^{++}, 2^{++}$ and with $J^{PC} = 0^{-+}$ by considering an axion term. For pure Yang-Mills  we have found that $\rm{sc} = 2.50492$ and   $\rm{V_gIR} = 3.47844$ reproduce the lattice ratios of 1.46 and 1.87 of $m_{2^{++}} / m_{0^{++}}$ and $m_{0^{*++}} / m_{0^{++}}$ respectively. In the IR, i.e. $\lambda \rightarrow \infty$, $V_g \sim \lambda^{\frac{4}{3}} \log^{\frac{1}{2}} \lambda$ which gives  linear asymptotic trajectories for glueballs.

The tachyon appears in this model in order to add matter by considering pairs of space-filling $D_4 - \bar{D_4}$ branes. In the boundary theory the operators with lowest dimension involving fermions are $\bar{\psi}^{i}_R \psi^j_L$ with spin 0 and the two spin 1 conserved currents $\bar{\psi}^i_L \sigma^\mu \psi^j_L$ and $\bar{\psi}^i_R \sigma^\mu \psi^j_R$. According to the gauge/gravity duality these operators are dual to bulk complex scalars $T_{ij}$ and two bulk gauge fields $A^{\mu}_{L ,\, ij}$ and $A^{\mu}_{R,\, ij}$ that transform as $(N_f, \bar{N}_f)$ of the flavour symmetry $U\left(N_f\right)_R \times U \left(N_f\right)_L$ while the fields $A^\mu_{L, i j}$ and $A^\mu_{R, i j}$ transform in the adjoint representations of $U\left(N_f\right)_L$ and  $U \left(N_f\right)_R$ respectively. In string theory the three bulk fields can be modelled by considering $N_f$ flavour branes (R) and $N_f$ flavour antibranes(L).  In this configuration the complex scalar fields $T_{ij}$ are the lowest modes of open strings with one end in a D brane and another in the anti-D brane while the fields gauge bulk fields are the lowest open string modes with both ends in a D brane or in a anti-D brane. For our purposes such a system obeys a Dirac-Born-Infeld (DBI) like action (we are ignoring a WZ term)
\begin{align}
\label{eq: DBI action}
&S_{DBI} = - \frac{x}{2} M_p^3 N_c^2 \int d^5 X \, \bold{Str} \left[ V_f \left(\lambda, T^{\dagger} T\right) \sqrt{det\left( g_{ab} + k\left(\lambda\right)D_{(a}T^{\dagger}D_{b)}T + w\left(\lambda\right) F^{L}_{ab}\right)} + \right. \\ \notag
&\left. + V_f \left(\lambda, T T^{\dagger} \right) \sqrt{det\left( g_{ab} + k\left(\lambda\right)D_{(a}TD_{b)}T^{\dagger} + w\left(\lambda\right) F^{R}_{ab}\right)}  \right]
\end{align}
where $\bold{Str}$ is the symmetric trace over the gauge indices while the determinant is taken with respect to the Minkowski indices (since we are going to work up to quadratic order we can treat the symmetric trace as the trace with respect to the gauge indices). $T_{ij}$ is a complex $N_f \times N_f$ matrix and $A^{L,R}_a$ are the world-volume gauge fields of the $U(N_f)_L \times U(N_f)_R$ group. The covariant derivative terms are given by
\begin{equation}
D_a T = \partial_a T - i T A^L_a + i A^R_a T \, , \, D_a T^{\dagger} = \partial_a T^{\dagger} - i  A^L_a T^{\dagger}+ i T^{\dagger} A^R_a 
\end{equation}
and the field strengths by
\begin{align}
F^{L,R} = d A^{L,R} - i A^{L,R} \wedge A^{L,R}
\end{align}
In this work we are assuming that the light quark masses are equal and under this assumption the tachyon is just $T = \tau \mathbf{1}$. Furthermore for the QCD vacuum $A^R_a = 0 = A^L_a$. Using these conditions in the action (\ref{eq: DBI action}) we get
\begin{equation}
S_f = - x M_p^3 N_c^2 \int d^5 X V_f\left(\lambda, T\right) \sqrt{det\left(g_{ab} + k\left(\lambda\right) \partial_a \tau \partial_b \tau \right)}.
\label{eq:sf_equation}
\end{equation}
The background is determined by considering the action $S = S_g + S_f$.

The DBI action that we have described is analogous to the flat space Sen action for the $D - \bar{D}$ system. Since we are in the presence of a curved space-time  and other non-trivial background fields we correct it by including the general potentials $V_f(\lambda, \tau)$, $k(\lambda)$ and $w(\lambda)$. However these potentials must satisfy some properties. The tachyon potential $V_f$ is expected to have a regular series expansion in $\lambda$ and $T$ near the boundary (i.e. $\lambda \rightarrow 0, \, T \rightarrow 0$)
\begin{equation}
V_f \approx V_0 (\lambda) + V_1 (\lambda) T^2 + \mathcal{O}(T^4)
\end{equation} 
and to vanish exponentially in the IR when $T \rightarrow \infty$. In particular in the flat space string theory $V_s \sim \frac{1}{\lambda} e^{- \mu T^2}$. Our ansatz for $V_f$ will be
\begin{align}
& V_f(\lambda, \tau) = V_{f0} \left(\lambda\right) V_\tau \left(\tau\right), \quad V_\tau (\tau) = ( 1 + a_1 \tau^2) e^{- a_2 \tau^2}, \\ \notag
& V_{f0}\left(\lambda\right) = W_0 + W_1 \lambda + W_2 \frac{\lambda^2}{1+ \frac{\rm{sc} \, \lambda}{\lambda_0}} + \frac{3 W_{IR}}{16 \pi^4} {\left(\rm{sc} \, \lambda \right)}^2 e^{- \frac{\lambda_0}{\rm{sc} \, \lambda}} \left( 1 + \frac{\lambda_0 W_1}{\rm{sc} \, \lambda}\right) \\
& W_1 = \frac{24 + \left(11- 2 x\right) W_0}{27 \pi^2}, \, W_2 =  \frac{24 (857 - 46 x) + W_0 (4619 - 1714 x + 92 x^2)}{46656 \pi^4} 
\end{align}
As both $k(\lambda)$ and $w(\lambda)$ are coupling functions under the square root of the DBI action we expect them to have similar qualitative behaviour. On the other hand in order to have the correct UV dimension of the $\bar{q}q$ operator we need to impose
\begin{equation}
k(0) = \frac{a_2 - a_1}{ \frac{3}{2} - x \frac{W_0}{8} }
\end{equation}
The IR asymptotics of the potentials $k$ and $w$ affect the meson spectrum. We are interested in the case where the mesons have an asymptotic linear spectrum and that the meson towers have the same asymptotics. This can be achieved with the following IR asymptotics $k(\lambda) \sim \lambda^{-4/3} {\left(\log \lambda\right)}^{1/2}$ and $w(\lambda) \sim \lambda^{-4/3} \log \lambda$. With this considerations we adopt the following ansatzs for $k$ and $w$
\begin{align}
& \frac{1}{k(\lambda)} =  \frac{ \frac{3}{2} - \frac{W_0 x}{8}}{a_2 - a_1} \left[ 1+ \frac{\rm{sc} \, k_{U1} \lambda}{\lambda_0} + k_{IR} \, e^{-\frac{\lambda_0}{\rm{ksc} \lambda}} \left( 1 + \frac{\lambda_0 k_1}{\rm{ksc} \lambda}\right) \frac{{\left(\frac{\rm{ksc} \lambda}{\lambda_0}\right)}^{4/3}}{\sqrt{\log\left(1 + \frac{\rm{ksc} \lambda}{\lambda_0}\right)}}\right], \\
& \frac{1}{w\left(\lambda\right)} = w_0 \left[1 + \frac{\rm{sc} w_{U1} \lambda}{\lambda_0 \left( 1+ \frac{\rm{sc} \lambda}{\lambda_0}\right)} + w_{IR} e^{- \frac{\lambda_0}{\rm{wsc} \lambda}} \left(1+\frac{\lambda_0 w_1}{\rm{wsc} \lambda}\right)\frac{{\left( \frac{\rm{wsc} \lambda}{\lambda_0}\right)}^{4/3}}{\log\left(1+\frac{\rm{wsc} \lambda}{\lambda_0}\right)} \right]
\end{align}
In order to compute the profiles of the background fields we need to specify the values of the parameters that appear in the definition of the potentials presented above. These parameters will be fitted to the ratios between the low-spin meson and tensor glueball masses to the $\rho^0$ meson mass as predicted by the model.

There is also an additional pseudo-scalar axion field $a$ whose action takes the form
\begin{equation}
\label{eq:axion_action}
S_a = - \frac{M^3 N_c^2}{2} \int d^5 X \sqrt{g} Z(\lambda) [ d a - x (2 V_a(\lambda, \tau) A - \theta d V_a (\lambda, \tau))]^2
\end{equation}
with $\theta$ being the overall phase of the tachyon $T = \tau e^{i \theta} \mathbb{I}_{N_f}$. In the flat-space tachyon condensation $V_a(\lambda, \tau)$ is independent of $\lambda$ and is that same as the tachyon potential that appears in the DBI action. Although in principle it may be different in principle we will take $V_a$ to be $V_f$ defined above without the $V_{0f}$ term. This form guarantees that it becomes a field-independent constant at $T=0$ and that it vanishes exponentially at $T = \infty$.
The $Z(\lambda)$ function is defined by
\begin{equation}
Z(\lambda) = Z_a + c_a {\left(\frac{\lambda}{\lambda_0}\right)}^4
\label{eq:Z_func_def}
\end{equation}
The definition is constrained by Yang-Mills theory and the parameters $Z_a$ and $c_a$ will be determined by fitting the spectrum of singlet axial vector mesons.



%%%%%%%%%%%%%%%%%%%%%%%%%%%%%
\section{Spectrum Evaluation}
%%%%%%%%%%%%%%%%%%%%%%%%%%%%%

The quadratic fluctuations around the background fields can be mapped to the spectrum of mesons and glueballs. The normalisable fluctuations of the metric $g_{ab}$, dilaton $\Phi$ and QCD axion $a$ correspond to glueballs with $J^{PC} = 0^{++}, \, 0^{-+}, \, 2^{++}$ where $J$ is the spin and $P$ and $C$ are the corresponding parity and charge conjugation properties. The meson sector comes from the normalisable fluctuations of the tachyon $T$ and of the gauge fields $A^{L/R}_{a}$. They correspond to mesons with $J^{PC} = 1^{++}, \, 1^{--}, \, 0^{++}, \, 0^{-+}$. 

The fluctuations can be classified according to how they transform under $SU(N_f)$. They can be grouped in flavour singlet and flavour non-singlet nodes. The fluctuations that come from $S_g$ and $S_a$ are only flavour singlet ones while the ones coming from $S_f$ come singlet and non-singlet terms. The singlet terms from $S_f$ will mix with with the singlet terms coming from $S_g$ and $S_a$. In this work we will not consider the flavour singlet states of $J^{PC} = 0^{++}, \, 0^{-+}$, as they involve mixing of the $0^{++}$ glueball with the flavour singlet $\sigma$ meson and mixing between the $0^{-+}$ glueball with the $\eta'$ meson, respectively.

The masses of the different glueballs and mesons can be obtained after expanding the action $S = S_g + S_f$ to quadratic order of the fluctuations of the background fields. It was shown in \cite{Arean:2013tja} that the fluctuations decouple in separate sectors. They are flavour singlet rank-two tensor fluctuations($J^{PC} = 2^{++}$), flavour singlet and non-singlet vector mesons ($J^{PC} = 1^{--}$), flavour singlet and non-singlet axial vector mesons ($J^{PC} = 1^{++}$), flavour singlet and non-singlet scalars ($J^{PC} = 0^{++}$) and flavour singlet and non-singlet pseudoscalars ($J^{PC} = 0^{-+}$). These fluctuations generate towers of $2^{++}$ glueballs, singlet and non-singlet vector mesons, singlet and non-singlet axial vector mesons, non-singlet scalar mesons and mixtures between $0^{++}$ glueballs and $\sigma$ mesons and non-singlet pseudoscalar mesons and mixtures between 
$0^{-+}$ glueballs and $\eta'$ mesons, respectively. 
These towers of mesons and glueballs come as solutions of a Schrodinger problem associated with the equation of motion of the associated fluctuation. The eigenvalues correspond to the square of the mass of the glueball or meson and their holographic wave functions are the associated eigenfunctions.

A detailed derivation of the equations of motion and Schrodinger problems associated with each fluctuation has been done in \cite{Arean:2013tja} and hence we will just summarise the main results relevant for work in the next sections. 
The singlet and non-singlet vector mesons have the same equation of motion
\begin{equation}
\frac{1}{V_f(\lambda, \tau) w(\lambda)^2 e^A G} \partial_z (V_f(\lambda, \tau) w(\lambda)^2 e^A G^{-1} \partial_z \psi_V) + m_V^2 \psi_V = 0 ,
\end{equation}
where $\psi_V$ is their wavefunction. By performing the change of variable defined by
\begin{equation}
\frac{du}{dz} = G(z) \equiv \sqrt{1 + e^{-2A} k(\lambda) {\left( \partial_z \tau \right)}^2}.
\end{equation}
and rescaling $\psi_V = \alpha / \Xi_V$ one can rewrite the equation of motion in the Schrodinger form,
\begin{equation}
- \frac{d^2 \alpha}{d u^2} + V_V \alpha = m_n^2 \alpha,
\end{equation}
 with potential
\begin{equation}
V_V (u) = \frac{1}{\Xi_V (u)} \frac{d^2 \Xi_V (u)}{d u^2}, \quad \Xi_V = w (\lambda) \sqrt{V_f (\lambda, \tau) e^A}
\end{equation}
The singlet and non-singlet axial vector mesons have Schrodinger potentials differing by a term coming from the action $S_a$. The potentials of the non-singlet axial vector mesons and of the singlet axial vector mesons are
\begin{align}
&V_{NSA} (u) = V_V(u) + 4 \frac{\tau^2 e^{2 A}}{w(\lambda)^2} k(\lambda) \\
&V_{SA}(u) = V_{NSA} (u) + 4 \, x \, \frac{e^{2A}Z(\lambda)V_a(\lambda, \tau)^2}{V_f(\lambda,\tau) G w(\lambda)^2}
\end{align}
The non-singlet scalar mesons have the potential
\begin{align}
&V_S(u) = \frac{1}{\Xi_S(u)} \frac{d^2 \Xi_S(u)}{d u^2} + H_S (u), \\
&\Xi_S(u) = \frac{1}{G} \sqrt{V_f(\lambda, \tau) k(\lambda) e^{3A}}, \\
&H_S(u) =  - \frac{e^{2A}}{k(\lambda)} \left( \frac{(\partial_\tau V_\tau)^2}{V_\tau^2} - \frac{\partial_\tau^2 V_\tau}{V_\tau} \right),
\end{align}
where the $H_S$ expression is the same as the one in \cite{Arean:2013tja} for $a_1 = 0$ and $a(\lambda) =  a_2 = 1$.
Finally the equation of motion of the non-singlet pseudoscalar fluctuations is given by
\begin{align}
&V_f(\lambda, \tau) \tau^2 e^{3A} G^{-1} k(\lambda) \partial_z \left[ \frac{1}{V_f(\lambda, \tau) \tau^2 k(\lambda) e^{3A} G} \partial_z \psi_P \right] - \notag \\
& - 4 \tau^2 e^{2A} \frac{k(\lambda)}{w(\lambda)^2} \psi_P + m^2 \psi_P = 0
\end{align}
with associated Schrodinger potential
\begin{align}
&V_P(u) = \frac{1}{\Xi_P(u)} \frac{d^2 \Xi_P(u)}{d u^2} + H_P (u), \\
&\Xi_P(u) = \frac{1}{\tau \sqrt{V_f(\lambda, \tau) k(\lambda) e^{3A}}}, \\
&H_P(u) = \frac{4 \tau^2 e^{2A} k(\lambda)}{w(\lambda)^2}
\end{align}
The singlet tensor fluctuation $J^{PC} = 2^{++}$ has the same equation of motion as in IHQCD, i.e.
\begin{equation}
\partial_z^2 h_{\mu \nu}^{\perp \perp} + 3 \dot{A} \partial_z h_{\mu \nu}^{\perp \perp} + \Box h_{\mu \nu}^{\perp \perp} = 0
\end{equation}
with the equivalent Schrodinger potential
\begin{equation}
V_G (z) = \frac{9}{4} {\dot{A}}^2 + \frac{3}{2} \ddot{A}
\end{equation}

The numerical determination of the spectrum proceeds by first finding the solution of the equations of motion defined by the action $S = S_g + S_f$. Details of the numerical procedure can be found in appendix~\ref{appendix:eom_sol}. For the cases of spin-2, singlet and non-singlet vector and axial vector and non-singlet scalar fluctuations we compute the Schrodinger potential and use a pseudospectral based on Chebyschev polynomials to compute the predicted masses of this model. The number of Chebyschev points used was 1000 and we checked the results were stable by computing the masses with a higher number of points. The reliability of the results was also studied by considering different IR and UV cutoffs on the background fields used to solve the Schrodinger problems. The masses of the pseudoscalars were computed using the shooting method.

\section{Fitting the meson and glueball spectrum}

We now proceed to fix the parameters that appear in the potentials by comparing the predictions of our model with the lattice QCD results for the glueball spectrum and the experimental values of the meson masses quoted by the Particle Data Group \cite{pdg_2020}.  The overall energy unit in the model is also a free parameter. Varying it changes nothing due to symmetry~\cite{Jarvinen:2011qe}. Hence all the masses computed are the same as the experimental and lattice ones in $\rm{GeV}$ units. For this reason we will fit the parameters to the numerical values of ratios of masses instead of numerical values of masses.

In the meson sector we will only consider the ones made of light quarks up and down. This sets the $x$ parameter coming from the flavour sector to be $2/3$. In table~\ref{table:light mesons} we show all the mesons listed in~\cite{pdg_2020} under \textit{light unflavoured mesons} with the values of $J^{PC}$ mentioned before. The exceptions are the flavour singlet scalars and pseudoscalars and the $a_0(980)$. The latter being a quark-antiquark state or a four-quark state is still debatable although the literature favours more the four-quark state hypothesis. For this reason we did not include it in this work. In table~\ref{table:other light mesons} we have the mesons under \textit{other light unflavoured mesons}, which are still not well establishd. In case some of these states are not confirmed this work should be updated.

\begin{table}
\centering
\begin{tabular}{ | c | c | c | c | }
\hline
$J^{PC}$ & I & Meson & Mass Measured (GeV) \\
\hline
$1^{--}$ & 1 & $\rho$ & 0.7755 \\
\hline
$1^{--}$ & 1 &  $\rho(1450)$ & 1.465 \\
\hline
$1^{--}$ & 1 & $\rho(1700)$ & 1.720 \\
\hline
$1^{--}$ & 0 & $\omega(782)$ & 0.78265 \\
\hline
$1^{--}$ & 0 &  $\omega(1420)$ & 1.420 \\
\hline
$1^{--}$ & 0 & $\omega(1650)$ & 1.670 \\
\hline
$1^{++}$ & 1 & $a_1(1260)$ & 1.230\\
\hline
$1^{++}$ & 0 & $f_1(1285)$ & 1.2819\\
\hline
$1^{++}$ & 0 & $f_1(1420)$ & 1.4264\\
\hline
$0^{++}$ & 1 &  $a_0(1450)$ & 1.474 \\
\hline
$0^{-+}$ & 1 &  $\pi_0$ & 0.134977 \\
\hline
$0^{-+}$ & 1 &  $\pi_0(1300)$ & 1.300 \\
\hline
$0^{-+}$ & 1 & $\pi_0(1800)$ & 1.812 \\
\hline
\end{tabular}
\caption{Light unflavoured mesons from~\cite{pdg_2020} used in the spectrum fit. The quantum number $I = 1$ means the meson is a flavour non-singlet state while $I = 0$ means the meson is a flavour singlet state.}
\label{table:light mesons} 
\end{table}

\begin{table}
\centering
\begin{tabular}{ | c | c | c | c | }
\hline
$J^{PC}$ & I & Meson & Mass Measured (GeV) \\
\hline
$1^{--}$ & 1 &  $\rho(2000)$ & 2.000 \\
\hline
$1^{--}$ & 1 & $\rho(2270)$ & 2.265 \\
\hline
$1^{--}$ & 0 &  $\omega(1960)$ & 1.960 \\
\hline
$1^{--}$ & 0 & $\omega(2205)$ & 2.205 \\
\hline
$1^{--}$ & 0 &  $\omega(2290)$ & 2.290 \\
\hline
$1^{--}$ & 0 & $\omega(2330)$ & 2.330 \\
\hline
$1^{++}$ & 1 & $a_1(1930)$ & 1.930\\
\hline
$1^{++}$ & 1 & $a_1(2095)$ & 2.095\\
\hline
$1^{++}$ & 1 & $a_1(2270)$ & 2.270\\
\hline
$1^{++}$ & 0 & $f_1(1970)$ & 1.971\\
\hline
$1^{++}$ & 0 & $f_1(2310$ & 2.310\\
\hline
$0^{++}$ & 1 &  $a_0(2020)$ & 2.025 \\
\hline
$0^{-+}$ & 1 &  $\pi_0(2070)$ & 2.070 \\
\hline
$0^{-+}$ & 1 & $\pi_0(2360)$ & 2.360 \\
\hline
\end{tabular}
\caption{Other light mesons from~\cite{pdg_2020} used in the spectrum fit. The quantum number $I = 1$ means the meson is a flavour non-singlet state while $I = 0$ means the meson is a flavour singlet state.}
\label{table:other light mesons} 
\end{table}

The profile of the background fields are determined by 17 parameters: $\rm{sc}$, $\rm{ksc}$, $\rm{wsc}$, $W_0$, $w_0$, $kU1$, $wU1$, $VgIR$, $WIR$, $kIR$, $wIR$, $W1$, $k1$, $w1$, $a_1$ and $a_2$ are parameters of the potentials appearing in the action and $\tau_0$ is a parameter that characterises the IR asymptotics of the tachyon. For our choice of potentials the asymptotics of the tachyon in the IR is
\begin{equation}
\tau \sim \tau_0 \, z^{\tau_c}, \quad \tau_c = \frac{4}{3} \frac{\left(\frac{3}{2} - x \frac{W_0}{8}\right) kIR a_2}{VgIR (a_2 - a_1)}
\end{equation}
In our fits $a_1$ and $a_2$ are fixed by imposing the constraints $a_2 - a_1 = 1$ and $a_2 = k(0) / 2$ in order to get a reasonable spectrum~\cite{Jarvinen:2015ofa}. This reduces the number of free parameters to 15. We then fit these parameters to ratios between glueball and meson masses in tables~\ref{table:light mesons} and~\ref{table:other light mesons} with the $\rho$ mass by minimising the function
\begin{equation}
J = \sum_{i} \frac{|R_{\rm{pred.} \,i} - R_{\rm{obs.} \, i}|}{R_{\rm{obs.} \, i}} + \lambda e^{- \left(\tau_c - 1\right)} + \lambda e^{-\left( \tau_M^2 - 3.5 \right)},
\end{equation}
excluding the singlet axial vector mesons. This gives a total of 23 data points. The sum term is the absolute relative difference between the predictions of our model and the ones obtained by using experimental and lattice data while the other two terms are constraints we want our background to satisfy. The first condition ensures absence of extra gauge anomalies in the IR while the second one gives more weight to solutions with a tachyon mass in the IR satisfying the Breitenlohner-Freedman bound. We repeated the fit with different values of the $\lambda$ parameter in order to balance the ability of the model to reproduce the observed ratios and still be consistent and stable. We have found that $\lambda = 0.1$ is a good choice and the results that we present below were obtained with this value. Having fixed the background, we fit the parameters $Z_a$ and $c_a$ of equation~\ref{eq:Z_func_def} against four mass ratios between the singlet axial vector mesons with the $\rho$ meson. With this procedure we have obtained the parameter values present in table~\ref{table:best_fit_background_pars} and the corresponding mass ratios of table~\ref{table:best spectrum fit}.
\begin{table}
\centering
\begin{tabular}{|c|c|c|c|c|c|}
\hline
Parameter & value & Parameter & value & Parameter & value \\
\hline
sc & 2.8328 & ksc & 3.16697 & wsc & 1.69263 \\
\hline
$W_0$ & 2.42885 & $w_0$ & 1.06381 & kU1 & 1.32535 \\ 
\hline
wU1 & -0.289843 & VgIR & 1.80415 & WIR & 1.13448 \\ 
\hline
kIR & 1.76473 & wIR & 2.92914 & W1 & 0.234188 \\ 
\hline
k1 & -0.307571 & w1 & 3.03579 & $a_1$ & 0.541312 \\
\hline
 $a_2$ & 1.54131 & $\tau_0$ & 0.92323 & $Z_a$ & -0.0376777 \\
\hline
$c_a$ & 23.3073 & & & & \\
\hline
\end{tabular}
\caption{Best fit parameters of the background potentials to the mass ratios between the mesons listed on tables~\ref{table:light mesons} and~\ref{table:other light mesons} and the $\rho$ meson. }
\label{table:best_fit_background_pars} 
\end{table}
\begin{table}
\centering
\begin{tabular}{ | c | c | c | c | }
\hline
Ratio & $\rm{R_{pred.}}$ & $\rm{R_{obs.}}$ & $|\rm{R_{pred.}} - \rm{R_{obs.}}| / \rm{R_{obs.}}$ \\
\hline
$m_{2^{++}} / m_\rho $  & $1.42174$	& $2.47401$ &	$0.42533$ \\
\hline
$m_{\rho(1450)}/m_\rho$ & $1.66194$ &	$1.88969$ & $0.12052$ \\
\hline
$m_{\rho(1700)} / m_\rho$ & $2.14063$ & $2.21861$ & $0.035149$ \\
\hline
$m_{\rho(2000)} / m_\rho$ & $2.55841$ & $2.57978$ & $0.00828202$ \\
\hline
$m_{\rho(2270)} / m_\rho $ & $2.93938$ & $2.9216$ & $0.00608397$ \\
\hline
$m_{\omega(782)} / m_{\rho}$ & $1$ & $1.00953$ & $0.00944228$ \\
\hline
$m_{\omega(1420)} / m_\rho$ & $1.66194$ &	$1.83164$ & $0.092649$ \\
\hline
$m_{\omega(1650)} / m_{\rho}$ & $2.14063$ & $2.15412$ & $0.00626128$ \\
\hline
$m_{\omega(1960)} / m_\rho$ & $2.55841$ &	$2.52818$ & $0.0119571$ \\
\hline
$m_{\omega(2205)} / m_{\rho}$ & 2.93938 &2.84421 &0.0334604 \\
\hline
$m_{\omega{2290}} / m_{\rho}$ & $3.2884$	& $2.95385$ & $0.113261$\\
\hline
$m_{\omega(2330)} / m_{\rho}$ & $3.60968$	& $3.00544$ &	$0.201048$ \\
\hline
$m_{a_1(1260)} / m_\rho$ & $1.59097$ &	 $1.58656$ & $0.00277946$ \\
\hline
$m_{a_1(1930)} / m_\rho $ & $2.09478$ & $2.48949$ &	$0.158549$ \\
\hline
$m_{a_1(2095)} / m_\rho$ & $2.52329$ &	 $2.70232$ & $0.0662509$ \\
\hline
$m_{a_1(2270)} / m_\rho$ & $2.91585$ & $2.92805$ & $0.00416789$ \\
\hline
$m_{f_1(1285)} / m_\rho$ & $1.65338$ &	$1.65351$ & $-8.06147\times10^{-5}$ \\
\hline
$m_{f_1(1420)} / m_{\rho}$ & $2.12826$ & $1.8399$ &  $0.156729$ \\
\hline
$m_{f_1(1970)} / m_{\rho}$ & $2.5426$ &	$2.54237$ & $8.91731 \times 10^{-5}$ \\
\hline
$m_{f_1(2310)} / m_{\rho}$ & $2.92962$ & $2.97965$ & $-0.0167892$ \\
\hline
$m_\pi / m_\rho$ & $0.174135$ & $0.174105$ & $0.000167864$ \\
\hline
$m_{\pi(1300)} / m_\rho$ & $1.73071$ & $1.67686$ & $0.0321131$ \\
\hline
$m_{\pi(1800} / m_\rho$ & $2.33737$ & $2.33728$ & $4.03495 \times 10^{-5}$ \\
\hline
$m_{\pi(2070)} / m_\rho$ & $2.78489$ & $2.67007$ & $0.0430032$ \\
\hline
$m_{\pi(2360)} / m_\rho$ & $3.17279$ & $3.04414$ & $0.0422604$ \\
\hline
$m_{a_0(1450)} / m_\rho$ & $0.684836$ & $1.9013$ & $0.639806$ \\
\hline
$m_{a_0(2020)} / m_\rho$ & $1.49201$ & $2.61203$ & $0.428791$ \\
\hline
\end{tabular}
\caption{Mass ratios obtained with the parameter values of table~\ref{table:best_fit_background_pars}.}
\label{table:best spectrum fit} 
\end{table}

As mentioned above after fixing the background we still have freedom to change the units of the model. This can be done by the scaling symmetry
\begin{equation}
A \rightarrow A - \log \Lambda \,, \quad z \rightarrow \Lambda z
\end{equation}
We have choosen $\Lambda$ such that the numerical mass of the $\rho$ meson is the one in GeV units. As expected the mass ratios have not changed.

\section{$\gamma\gamma$, $\gamma p$ and $pp$ total cross-sections in holographic QCD}
In this section we present the necessary ingredients to compute the total cross-sections of $\gamma\gamma$, $\gamma p$ and $pp$ scattering in holographic models of QCD in the Veneziano limit. First we will discuss the kinematics of each process. Then we will present generic holographic expressions of the forward scattering amplitude, in the Regge limit, via the exchange of higher spin J fields. We conclude by deriving the holographic expression of the total cross-sections by taking the imaginary part of the amplitudes and using the optical theorem.

\subsection{Kinematics}
For all processes we will use light-cone coordinates $\left(+,-,\perp \right)$, with flat space metric $ds^2 = - dx^+ dx^- + d x^2_\perp$, where $x_\perp \in \mathbb{R}^2$.
The large $s$ kinematics of  $\gamma^{*} \left(Q_1^2\right) p \rightarrow \gamma^{*} \left(Q_3^2\right) p $ scattering for the incoming and outgoing off-shell photons are the following
\begin{equation}
k_1=\left(\!\sqrt{s},-\frac{Q_1^2}{\sqrt{s}} ,0\right),
\ \ \ \ \ 
-k_3=\left(\sqrt{s} , \frac{q_\perp^2 - Q_3^2}{\sqrt{s}}, q_\perp \right),
\label{eq:gp_off_shell_photons_kinematics}
\end{equation}
while the incoming and outgoing protons with mass $M$ have momenta
\begin{equation}
k_2=\left(\frac{M^2}{\sqrt{s}} , \sqrt{s}, 0\right),
\ \ \ \ \ 
-k_4=\left(\frac{q_\perp^2 + M^2}{\sqrt{s}}, \sqrt{s},  -q_\perp \right).
\label{eq:gp_off_shell_photons_kinematics}
\end{equation}
The momentum transfer $q_\perp$ is a $\mathbb{R}^2$ vector and is related to the Mandelstam variable $t$ through $t = - q_\perp^2$.
For the forward scattering amplitude the 
momentum transfer $q_\perp = 0$ so that the outgoing off-shell photon has $k_3=-k_1$ and the outgoing
proton $k_4=-k_2$. The  incoming and outgoing photon polarizations are the same.
The possible polarization vectors are
\begin{equation}
  \label{eq:polarization vectors} 
n(\lambda) = \begin{cases}
    \big(0,0,\epsilon_\lambda\big) \,, & \ \ \ \lambda=1,2 \,,\\
   \big( \sqrt{s}/Q, Q/\sqrt{s},0  \big)\, , & \ \ \ \lambda=3\,,
    \end{cases}
\end{equation}
where $\epsilon_\lambda$ is just the usual transverse polarization vector.
In the context of holographic QCD the relevant Witten diagram is the present in figure~\ref{fig:Witten_diagram_gp}.
\begin{figure}[!t]
  \center
  \includegraphics[height=4cm]{images/WittenDiagram_gp} 
  \caption{Tree level Witten diagram representing spin $J$    exchange in  $\gamma^* p\to\gamma^* p$ scattering. 
The $n_1$ and $n_3$ labels denote the incoming/outgoing photon polarizations, for forward scattering $n_1=n_3$.
}
  \label{fig:Witten_diagram_gp}
\end{figure}

The incoming photons in the $\gamma^{*}\left(Q_1^2\right) \gamma^{*}\left(Q_2^2\right) \rightarrow \gamma^{*}\left(Q_3^2\right) \gamma^{*}\left(Q_4^2\right)$  have the four momenta
\begin{equation} 
k_1 = \left( \sqrt{s}, - \frac{Q_1^2}{\sqrt{s}}, 0 \right) \,,  \qquad  
k_2 = \left( - \frac{Q_2^2}{\sqrt{s}}, \sqrt{s},  0 \right) \,,
\end{equation}
while the outgoing photons have
\begin{equation}
k_3 = - \left( \sqrt{s},  \frac{q_\perp^2 - Q_3^2}{\sqrt{s}}, q_\perp \right) \, \qquad  k_4 = - \left(  \frac{q_\perp^2-Q_4^2}{\sqrt{s}}, \sqrt{s},  -q_\perp \right),
\end{equation}
where $Q^2_i = k_i^2 >0$  $(i=1, \dots, 4)$ is the off-shellness. As in the case of $\gamma^* p$ scattering for the forward scattering amplitude the 
momentum transfer is null.
The possible polarization vectors are, respectively,
\begin{align}
    &n_{1,3} =
    \begin{cases}
      \left(0,0,1,0\right), & \lambda=1 \\
      \left(0,0,0,1\right), & \lambda=2 \\
      \frac{1}{Q} \left( \sqrt{s}, \frac{Q^2}{\sqrt{s}}, 0, 0 \right), & \lambda = 3
    \end{cases}\,,\\
    &n_{2,4} =
    \begin{cases}
      \left(0,0,1,0\right), & \lambda=1 \\
      \left(0,0,0,1\right), & \lambda=2 \\
      \frac{1}{Q} \left( \frac{Q^2}{\sqrt{s}}, \sqrt{s}, 0, 0 \right), & \lambda = 3
    \end{cases} \,,
    \label{eq:inPolarization}
\end{align}
since in the forward amplitude the incoming and outgoing off-shell photons have the same polarizations.
Notice that the transverse photons~$\left(\lambda=1,2\right)$ are normalized such that~$n^2 = 1$, while the longitudinal photons~$\left(\lambda=3\right)$ are normalized such that~$n^2 = -1$.
For this process the Witten diagram that we compute is the one of figure~\ref{fig:Witten_diagram_gamma_gamma}.
\begin{figure}[!h]
  \center
  \includegraphics[height=4cm]{images/WittenDiagram_gg.pdf} 
  \caption{Tree level Witten diagram representing spin $J$    exchange in a $\gamma^* \gamma^* \to \gamma^* \gamma^*$ scattering process between bulk photons.}
  \label{fig:Witten_diagram_gamma_gamma}
\end{figure}

Finally, the large $s$ kinematics of $pp$ scattering are the following
\begin{align}
\label{eq:pp_kinematics}
&k_1=\left(\!\sqrt{s},\frac{M^2}{\sqrt{s}} ,0\right),\  \ k_3=-\left(\!\sqrt{s},\frac{ q_\perp^2 +M^2}{\sqrt{s}} , q_\perp \right)\!,\\
&k_2=\left(\frac{M^2}{\sqrt{s}},\sqrt{s} ,0\right),\  \ k_4=-\left(\frac{M^2+ q_\perp^2}{\sqrt{s}},\sqrt{s} ,-q_\perp \right).\notag
\end{align}
where $k_1$ and $k_2$ are the incoming proton momenta and $k_3$ and $k_4$ are the outgoing proton momenta. As in the other processes we will only compute the forward scattering amplitude for which $q_\perp = 0$. In figure~\ref{fig:Witten_diagram_pp} shows the Witten diagram for $pp$ scattering. 
\begin{figure}[!h]
  \center
  \includegraphics[height=4cm]{images/WittenDiagram_pp.pdf} 
  \caption{Tree level Witten diagram representing spin $J$    exchange in a $pp \to pp$ scattering process between bulk protons.}
  \label{fig:Witten_diagram_pp}
\end{figure}

\subsection{Holographic scattering amplitudes}
Before we start the computation of the forward scattering amplitudes we need to define the external states as well the interaction between them and the spin J fields that are exchanged in the Witten diagrams of figures~\ref{fig:Witten_diagram_gp},~\ref{fig:Witten_diagram_gamma_gamma} and~\ref{fig:Witten_diagram_pp}.

An external photon is a source of the conserved current $\bar{\psi} \gamma^\mu \psi$, where the quark field $\psi$ comes from the open string sector. According to the gauge/gravity duality this field is dual to the nonnormalizable mode of a vector field in the bulk. In the context of this model the natural candidate is the linear combination of the $A^L$ and $A^R$ gauge fields
\begin{align}
V_M = \frac{A^L_M + A^R_M}{2} \,.
\end{align}
In the string frame the action of this field is
\begin{equation}
S = - \frac{1}{4} M^3 N_c N_f \int d^5 X \sqrt{-g_s} e^{-\frac{10}{3} \Phi} V_f w_s^2 G F_{AB}g^{AC}_{\rm{eff. (s)}}g^{BD}_{\rm{eff. (s)}}F_{CD} \, ,
\label{eq:VM_new_action_string_frame}
\end{equation}
where $g_s$ is the determinant of the metric in the string frame, $g^{AB}_{\rm{eff. (s)}}$ is the inverse of $g_{\rm{eff.(s)} AB} = g_{\rm{(s)} AB} +  k_s\left(\lambda\right) \partial_M \tau \partial_N \tau $, $k_s \left(\lambda\right) = \lambda^{4 / 3} k\left(\lambda\right) $ and $w_s \left(\lambda\right) = \lambda^{4 / 3} w\left(\lambda\right)$. Working on the gauge $V_z = 0$ and $\partial_\mu V^{\mu} = 0$ the vector field components describing a boundary plane wave solution with polarization $n_\mu$ take the form
\begin{equation}
V_\mu \left(x, z\right) = n_\mu f_Q \left(z\right) e^{i k \cdot x} \, , \quad k^2 = Q^2 \, ,
\end{equation}
where $f_Q$ satisfies the equation
\begin{equation}
\frac{1}{V_f(\lambda, \tau) w(\lambda)^2 e^A G} \partial_z (V_f(\lambda, \tau) w(\lambda)^2 e^A G^{-1} \partial_z f_Q) - Q^2 f_Q = 0 \, \,
\end{equation}
subject to the boundary conditions $f_Q\left(0\right) = 1$ and $\partial_z f_Q \left( z \to \infty \right) = 0$.

For the proton external state we consider that it is dual to the normalizable mode of a bulk scalar field $\Upsilon\left(x, z\right) = e^{i P \cdot x} \upsilon\left(z\right)$ that represents an unpolarised proton. We will see that the contribution of the proton wavefunction to the scattering amplitudes is inside an integral that will be absorbed in the coupling constants and hence the precise details will not be important.

The last ingredient of our model are the higher spin fields $h_{a_1 \cdots a_J}$ that will mediate the interaction between the external states in the considered scattering states. In this work we will consider bulk spin J fields that are dual to the spin J twist two operators made of the gluon field as well bulk spin J fields dual to the spin J twist two operators made of the quark bilinears. This extends the previous works~\cite{Ballon-Bayona:2015wra, ballon_bayona_unity_2017, Amorim:2018yod} where only the bulk fields dual to the gluon operators were considered. As discussed in appendix~\ref{appendix:b} we will consider that the coupling between the $U(1)$ gauge field and these spin J fields is given by
\begin{align}
k_J \int d^5X \sqrt{-g_s} G e^{-\frac{10}{3} \Phi} V_f \left( \lambda, \tau \right) {w_s\left(\lambda \right)}^2  g_{eff. s}^{AB} F^V_{A C} \nabla_{A_1} \dots \nabla_{A_{J-2}} F^V_{B D} h^{C D A_1 \dots A_{J-2}} \, ,
\label{eq:gauge_field_spin_J_coupling}
\end{align}
while for the scalar field $\Upsilon$ dual to the proton state the coupling is
\begin{align}
\bar{k}_J \int d^5 X \sqrt{-g_s} e^{-\Phi} \left( \Upsilon D_{b_1} \dots D_{b_J} \Upsilon \right) h^{b_1 \dots b_J} \, .
\label{eq:proton_spin_J_coupling}
\end{align}

The higher spin J field $h_{A_1 \cdots A_J}$ is totally symmetric, traceless and satisfies the transversality property $\nabla^{A_1} h_{A_1 \cdots A_J} = 0$. This implies that it is not important in which external fields the covariant derivatives in~\ref{eq:gauge_field_spin_J_coupling} and~\ref{eq:proton_spin_J_coupling} act. 
Below we assume that the spin J field has a propagator, without specifying its form. In the next sections we focus on the dynamics of this field in detail for the case of pomeron and meson trajectories.

\subsection{Regge limit analysis of the spin J coupling}
Now that we have an ansatz for the coupling between the vector $U(1)$ gauge field and the spin J fields in the graviton's Regge trajectory we wish to study it's amplitude in the Regge limit. From now on, to simplify notation, we will assume that all quantities are given in the string frame. In the Regge limit the covariant derivatives can be substituted by partial derivatives because of powers of $\sqrt{s}$. The coupling reduces to
\begin{align}
&g^{\alpha \mu} g^{\beta \nu} g^{\alpha_1 \beta_1} \dots g^{\alpha_{J-2} \beta_{J-2}} g^{AB}_{eff} F^{V}_{A \alpha} \partial_{\beta_1} \dots \partial_{\beta_{J-2}} F^{V}_{B \beta} h_{\mu \nu \alpha_1 \dots \alpha_{J_2}} = \\
& = g^{\alpha \mu} g^{\beta \nu} g^{\alpha_1 \beta_1} \dots g^{\alpha_{J-2} \beta_{J-2}} e^{-2A} \left( \frac{1}{G^2} \partial_z V_\alpha \partial_{\beta_1} \dots \partial_{\beta_{J-2}} \partial_z V_\beta + \eta^{\rho \sigma} V_{\rho \alpha} \partial_{\beta_1} \dots \partial_{\beta_{J-2}} V_{\sigma \beta} \right) h_{\mu \nu \alpha_1 \dots \alpha_{J_2}}
\end{align}
Since $V_\mu = n_\mu f_Q \left( z\right) e^{i k \cdot x}$, $\partial_\mu V = i k_\mu V$ and then
\begin{align}
g^{\alpha \mu} g^{\beta \nu} g^{\alpha_1 \beta_1} \dots g^{\alpha_{J-2} \beta_{J-2}} \left( i k_{\beta_1}\right) \dots  \left( i k_{\beta_{J-2}}\right) e^{-2A} \left( \frac{1}{G^2} \partial_z V_\alpha  \partial_z V_\beta + \eta^{\rho \sigma} V_{\rho \alpha} V_{\sigma \beta} \right) h_{\mu \nu \alpha_1 \dots \alpha_{J_2}}
\end{align}

In DIS we will be interested in incoming and outgoing  off-shell photons with the same polarisation. For both photons with perpendicular polarisation ($\lambda = 1, 2$) and both with longitudinal polarisation ($\lambda = 3$) we have respectively:
\begin{align}
&g^{\alpha \mu} g^{\beta \nu} e^{-2 A} \left( \frac{1}{G^2} \partial_z V_\alpha \partial_z V_\beta + \eta^{\rho \sigma} V_{\rho \alpha} V_{\sigma \beta} \right) h_{\mu \nu \alpha_1 \dots \alpha_{J-2}} = \frac{1}{4} s e^{-6A} {f_Q}^2 e^{-i q_\perp \cdot x_\perp} h^{--}_{\alpha_1 \dots \alpha_{J-2}}, \\
&g^{\alpha \mu} g^{\beta \nu} e^{-2 A} \left( \frac{1}{G^2} \partial_z V_\alpha \partial_z V_\beta + \eta^{\rho \sigma} V_{\rho \alpha} V_{\sigma \beta} \right) h_{\mu \nu \alpha_1 \dots \alpha_{J-2}} = \frac{1}{4} s e^{-6A} \frac{{\partial_z f_Q}^2}{Q^2 G^2} e^{-i q_\perp \cdot x_\perp} h^{--}_{\alpha_1 \dots \alpha_{J-2}}
\end{align}


\subsection{Upper part Witten diagram}
We are now ready to compute the upper part of the Witten diagram
\begin{align}
&k_J \int d^5X \sqrt{-g_s} G e^{-\frac{10}{3}\Phi} V_f \left( \lambda, \tau \right) {w_s\left(\lambda \right)}^2 g^{\alpha_1 \beta_1} \dots g^{\alpha_{J-2} \beta_{J-2}} \left( i k_{\beta_1}\right) \dots  \left( i k_{\beta_{J-2}}\right)  \frac{s}{4} e^{-6A} \left(  {f_Q}^2 + \frac{{\partial_z f_Q}^2}{Q^2 G^2}  \right) e^{-i q_\perp \cdot x_\perp} h^{--}_{\alpha_1 \dots \alpha_{J-2}} = \notag \\
& = i^{J-2} k_J \int d^5X \sqrt{-g_s} G e^{-\frac{10}{3}\Phi} V_f \left( \lambda, \tau \right) {w_s\left(\lambda \right)}^2 e^{-2 J A}  s^{J/2} e^{-2A} \left(  {f_Q}^2 + \frac{{\partial_z f_Q}^2}{Q^2 G^2}  \right) e^{-i q_\perp \cdot x_\perp} h_{+ \dots +} =
\end{align}
where we have used the kinematics of the outgoing photon and kept the leadin term power of $\sqrt{s}$.

\subsection{Lower part Witten diagram}
Using the last equation and remembering that $\Upsilon\left(x, z\right) = e^{i P \cdot x} \upsilon\left(z\right)$ in the Regge limit the lower part of the Witten diagram contributes as
\begin{align}
&\bar{k}_J \int d^5 X \sqrt{-g_s} e^{-\Phi}  {\upsilon\left(z\right)}^2 \left( i k_{\beta_1}\right) \dots \left( i k_{\beta_J}\right) e^{i q_\perp \cdot x}  h^{\beta_1 \dots \beta_J} = \bar{k}_J \int d^5 X \sqrt{-g_s} e^{-\Phi}  {\upsilon\left(z\right)}^2 {\left( i k_+\right)}^J e^{i q_\perp \cdot x}  h^{+ \dots +} = \notag \\
& = \bar{k}_J \int d^5 X \sqrt{-g_s} e^{-\Phi} {\left( - 2 e^{-2 A}\right)}^J {\upsilon\left(z\right)}^2 {\left( \frac{i}{2} \sqrt{s}\right)}^J e^{i q_\perp \cdot x}  h_{- \dots -}  = i^J s^{J/2}  \bar{k}_J \int d^5 X \sqrt{-g_s} e^{-\Phi} e^{-2 J A} {\upsilon\left(z\right)}^2 e^{i q_\perp \cdot x}  h_{- \dots -}
\end{align}

\subsection{spin J exchange amplitude}
Now that we know the contributions form the upper and lower part of the diagram we can write the scattering amplitude $\mathcal{A}_J$ contribution due to the exchange of an even spin J field:
\begin{align}
 - k_J \bar{k}_J s^J \int d^5X d^5 \bar{X} \sqrt{-g_s} \sqrt{-\bar{g}_s} G e^{-\frac{10}{3}\Phi} V_f \left( \lambda, \tau \right) {w_s\left(\lambda \right)}^2 e^{-\bar{\Phi}} e^{-2 J \left(A+\bar{A}\right)}  e^{-2A} \left(  {f_Q}^2 + \frac{{\partial_z f_Q}^2}{Q^2 G^2}  \right)  {\bar{\upsilon}}^2 e^{-i q_\perp \cdot \left( x_\perp - \bar{x}_\perp \right)} \Pi _{+\dots+,-\dots-}\left(X, \bar{X} \right)
\end{align}
We now perform the change of variable $w = x - \bar{x}$ and using the identity
\begin{align}
 \int d^2 l_\perp e^{- i q_\perp l_\perp} \int \frac{dw^+ dw^-}{2} \Pi_{+ \dots +, - \dots -} \left(z, \bar{z}, w^+, w^-, l_\perp \right) = - \frac{i}{2^J} {\left( e^{A + \bar{A}} \right)}^{J-1}G_J \left(z, \bar{z}, t\right)
\end{align}
we get
\begin{align}
 & \mathcal{A}_J = \frac{k_J \bar{k}_J}{2^J} s^J \int dz d\bar{z} e^{2 A}e^{4 \bar{A}} G e^{-\frac{10}{3}\Phi} V_f \left( \lambda, \tau \right) {w_s\left(\lambda \right)}^2 e^{-\bar{\Phi}} e^{-J \left(A+\bar{A}\right)}  \left(  {f_Q}^2 + \frac{{\partial_z f_Q}^2}{Q^2 G^2}  \right)  {\bar{\upsilon}}^2 G_J \left(z, \bar{z}, t\right), \\ 
 & G_J \left(z , \bar{z}, t \right) = e^{\Phi + \bar{\Phi} - \frac{1}{2} \left(A + \bar{A}\right)} \sum_n \frac{\psi_n \left(J, z\right) \psi_n^* \left(J, \bar{z}\right)}{t_n\left(J\right) - t}
\end{align}

\subsection{Regge theory}
In order to get the total amplitude we need to sum over even spin J fields with $J \geq 2$. Then we can apply a Sommerfeld-Watson transform
\begin{align}
\frac{1}{2} \sum_{J \geq 2} \left(s^J + {\left(-s\right)}^J\right) = - \frac{\pi}{2} \int \frac{d J}{2 \pi i} \frac{s^J + \left(-s\right)^J}{\sin \pi J}
\end{align}
which requires analytic continuation of the amplitude for the spin J exchange to the complex J-plane. We assume that the J-plane integral can be deformed from the poles at even J, to the poles $J = j_n \left(t\right)$ defined by $t_n \left(J\right) = t$. The scattering domain of negative t contains these poles along the real axis for J < 2. The scattering amplitude for $t = 0$ is then
\begin{align}
&\mathcal{A} \left(s, 0 \right) = \sum_n g_n s^{j_n\left(0\right)} \int dz e^{A(\frac{3}{2} - {j_n\left(0\right)})} G e^{-\frac{7}{3} \Phi}  V_f \left( \lambda, \tau \right) {w_s\left(\lambda \right)}^2  \left(  {f_Q}^2 + \frac{{\partial_z f_Q}^2}{Q^2 G^2}  \right) \psi_n \left(j_n, z\right), \\
& g_n = - \frac{\pi}{2} \frac{k_{j_n\left(0\right)} \bar{k}_{j_n\left(0\right)}}{2^{j_n\left(0\right)}}  \left(i + \cot \frac{\pi j_n\left(0\right)}{2} \right) \frac{d j_n}{dt} \int d\bar{z} e^{\bar{A}\left( \frac{7}{2} - j_n \right)}  {\bar{\upsilon}}^2  \psi\left(j_n, \bar{z}\right) \notag
\end{align}

\subsection{Expressions for $F_2\left(x, Q^2\right)$ and $F_L\left(x, Q^2\right)$ due to pomeron exchange}
$F_2$ is related with the transverse and longitudinal cross-sections through the equation
\begin{equation}
 F_2\left(x, Q^2\right) = \frac{Q^2}{4 \pi^2 \alpha} \left( \sigma_T + \sigma_L \right)
\end{equation}
Using the final result of the previous section and the optical theorem we can write the holographic expression for $F_2$ and $F_L$
\begin{align}
 &F_2\left(x, Q^2\right) = \sum_n \frac{{\rm Im} \, g_n}{4 \pi^2 \alpha} Q^{2 j_n} x^{1-jn} \int dz e^{A(\frac{3}{2} - {j_n\left(0\right)})} G e^{-\frac{7}{3} \Phi}  V_f \left( \lambda, \tau \right) {w_s\left(\lambda \right)}^2  \left(  {f_Q}^2 + \frac{{\partial_z f_Q}^2}{Q^2 G^2}  \right) \psi_n \left(j_n, z\right), \\
& F_L\left(x, Q^2\right) = \sum_n \frac{{\rm Im} \,g_n}{4 \pi^2 \alpha} Q^{2 j_n} x^{1-jn} \int dz e^{A(\frac{3}{2} - {j_n\left(0\right)})} G e^{-\frac{7}{3} \Phi}  V_f \left( \lambda, \tau \right) {w_s\left(\lambda \right)}^2  \frac{{\partial_z f_Q}^2}{Q^2 G^2} \psi_n \left(j_n, z\right)
\end{align}
It turns out that a more convenient variable to perform the above integration is the u variable defined by $du = G dz$. Then the expressions for $F_2$ and $F_L$ are then
\begin{align}
 &F_2\left(x, Q^2\right) = \sum_n \frac{{\rm Im} \, g_n}{4 \pi^2 \alpha} Q^{2 j_n} x^{1-jn} \int du e^{A(\frac{3}{2} - {j_n\left(0\right)})}  e^{-\frac{7}{3} \Phi}  V_f \left( \lambda, \tau \right) {w_s\left(\lambda \right)}^2  \left(  {f_Q}^2 + \frac{{\partial_u f_Q}^2}{Q^2}  \right) \psi_n \left(j_n, u\right), \\
& F_L\left(x, Q^2\right) = \sum_n \frac{{\rm Im} \, g_n}{4 \pi^2 \alpha} Q^{2 j_n} x^{1-jn} \int du e^{A(\frac{3}{2} - {j_n\left(0\right)})}  e^{-\frac{7}{3} \Phi}  V_f \left( \lambda, \tau \right) {w_s\left(\lambda \right)}^2  \frac{{\partial_u f_Q}^2}{Q^2} \psi_n \left(j_n, u\right)
\end{align}
The structure function $F_2$ is related to the total cross-section $\sigma\left(\gamma p \rightarrow X\right)$ through
\begin{equation}
\sigma\left(\gamma p \rightarrow X\right) = 4 \pi^2 \alpha \lim_{Q^2 \rightarrow 0} \frac{F_2\left(x, Q^2\right)}{Q^2}.
\end{equation}
In the limit of $Q^2 \rightarrow 0$
\begin{equation}
\lim_{Q\rightarrow 0} f_Q = 1 , \quad \lim_{Q\rightarrow 0} \frac{\dot{f}_Q}{Q} = 0
\end{equation}
and it follows that
\begin{equation}
\sigma\left(\gamma p \rightarrow X\right) = \sum_n {\rm Im} g_n \,  s^{j_n - 1} \int du e^{-\left(j_n - 3/2 \right)A } e^{-\frac{7}{3} \Phi} V_f w_s^2  \,  \psi_n\left(u\right).
\end{equation}



\section{Spin J field EOM}

In this section we will propose a phenomenological equation of motion for the spin J fields. First we will derive the equation of motion of the graviton in the Einstein frame which, as we shall see, is the same as in the Improved Holographic QCD model. This result is consistent with equation (A.108) from 1309.2286 which is the same result found in IHQCD.

To find the equation of motion of the graviton we ned to perturb the actions (\ref{eq: HVQCD action}) with respect to the metric. The $S_g$ term contributes with
\begin{align}
&\delta S_g = M^3 N_c^2 \int d^5X \sqrt{-g} \left[ R_{ab} - \frac{1}{2} g_{ab} \left( R -\frac{4}{3} \nabla_a \Phi \nabla^a \Phi + V_g \left(\lambda\right) \right)- \frac{4}{3} \nabla_a \Phi \nabla_b \Phi \right],  \notag \\
& \delta S_f = - M^3 N_c N_f \int d^5 X V_f \left( \lambda, \tau \right) \delta \sqrt{- \text{det} \left( g_{MN} + k\left(\lambda\right) \partial_M \tau \partial_N \tau \right)}.
\end{align}
To make further progress we define $g_{eff. MN} = g_{MN} +  k\left(\lambda\right) \partial_M \tau \partial_N \tau $ and its matrix inverse $g_{eff.}^{MA}g_{eff. AN} = \delta^M_N$. The following identity will be useful
\begin{align}
\left( g_{eff}^{AB} + \delta  g_{eff}^{AB}\right) \left( g_{eff. BC} + \delta g_{BC}\right) = \delta^A_C \Leftrightarrow \delta g_{eff.}^{AD} = - g_{eff}^{AB} g_{eff.}^{DC} \delta g_{BC}.
\end{align}
From Jacobi's formula it follows
\begin{align}
\delta \sqrt{-\text{det} \left(g_{eff. MN}\right)} = -\frac{1}{2} \sqrt{-\text{det} \left(g_{eff. MN}\right)} g_{eff. MN} \delta g_{eff.}^{MN} = \frac{1}{2} \sqrt{-\text{det} \left(g_{eff. MN}\right)} g_{eff.}^{AB} \delta g_{AB}
\end{align}
Furthermore $\delta g_{AB} = - g_{AC} g_{BD} \delta g^{CD}$ and the contribution from $S_f$ is
\begin{align}
\delta S_f = \frac{1}{2} M^3 N_c N_f \int d^5X V_f\left(\lambda, \tau\right) \sqrt{-\text{det} g_{eff. MN}} g_{eff.}^{AB}g_{AC}g_{BD} \delta g^{CD}.
\end{align}

The Einstein equations are then
\begin{align}
R_{CD} - \frac{1}{2} g_{CD} \left( R - \frac{4}{3} {\left( \nabla \Phi \right)}^2 + V_g \left(\lambda\right) \right) - \frac{4}{3} \nabla_C \Phi \nabla_D \Phi + \frac{1}{2} x_f V_f \frac{\sqrt{-g_{eff.}}}{\sqrt{-g}} g_{eff.}^{AB} g_{AC} g_{BD} = 0.
\end{align}
Taking the trace of this equation we find that the Ricci scalar is given by
\begin{equation}
R = \frac{4}{3} {\left( \nabla \Phi \right)}^2 - \frac{5}{3} V_g \left( \lambda \right) + \frac{1}{3} x V_f \frac{\sqrt{-g_{eff.}}}{\sqrt{-g}} g_{eff.}^{AB} g_{AB},
\end{equation}
and substituting it in the Einstein equations we have
\begin{align}
R_{CD} - \frac{1}{2} g_{CD} \left(-\frac{2}{3} V_g \left(\lambda\right) + \frac{1}{3} x V_f \frac{\sqrt{-g_{eff.}}}{\sqrt{-g}} g_{eff.}^{AB} g_{AB} \right) - \frac{4}{3} \nabla_C \Phi \nabla_D \Phi + \frac{1}{2} x_f V_f \frac{\sqrt{-g_{eff.}}}{\sqrt{-g}} g_{eff.}^{AB} g_{AC} g_{BD} = 0.
\end{align}
Moreover $g_{eff.} = g + k\left( \lambda \right) d \tau \otimes d \tau$ meaning that $\sqrt{-g_{eff.}} = \sqrt{-g} \sqrt{\text{det} \left( 1+ k \left( \lambda \right) g^{-1} d \tau \otimes d \tau \right)}$ resultin in equation
\begin{align}
&R_{CD} - \frac{1}{2} g_{CD} \left( - \frac{2}{3} V_g \left( \lambda \right)  + \frac{1}{3} x V_f \sqrt{\text{det} \left( 1+ k \left( \lambda \right) g^{-1} d \tau \otimes d \tau \right)} g_{eff.}^{AB} g_{AB} \right) - \frac{4}{3} \nabla_C \Phi \nabla_D \Phi + \notag \\\ 
&+ \frac{1}{2} x V_f \sqrt{\text{det} \left( 1+ k \left( \lambda \right) g^{-1} d \tau \otimes d \tau \right)} g_{eff.}^{AB} g_{AC} g_{BD} = 0
\end{align}

The linearized Einstein equations are
\begin{align}
& \delta R_{CD} - \frac{1}{2} h_{CD} \left( - \frac{2}{3} V_g \left(\lambda\right) + \frac{1}{3} x V_f \sqrt{1 + k\left(\lambda\right) e^{-2A} \dot{\tau}^2} g_{eff.}^{AB} g_{AB} \right) - \frac{1}{2} g_{CD} \frac{1}{3} x V_f \times \notag \\
& \times \left( \delta  \sqrt{1 + k\left(\lambda\right) g^{-1} d \tau \otimes d \tau} g_{eff.}^{AB}g_{AB} +  \sqrt{1 + k\left(\lambda\right) e^{-2A} \dot{\tau}^2} \delta g_{eff.}^{AB} g_{AB} +  \sqrt{1 + k\left(\lambda\right) e^{-2A} \dot{\tau}^2} g^{AB}_{eff.} h_{AB} \right) + \notag \\
& + \frac{1}{2} x V_f \left(  \delta  \sqrt{1 + k\left(\lambda\right) g^{-1} d \tau \otimes d \tau} g_{eff.}^{AB} g_{AC} g_{BD} +  \sqrt{1 + k\left(\lambda\right) e^{-2A} \dot{\tau}^2} \delta g_{eff.}^{AB} g_{AC} g_{BD} + \right. \notag \\
&\left. +  \sqrt{1 + k\left(\lambda\right) e^{-2A} \dot{\tau}^2} g_{eff.}^{AB} h_{AC} g_{BD} +  \sqrt{1 + k\left(\lambda\right) e^{-2A} \dot{\tau}^2} g_{eff.}^{AB} g_{AC} h_{BD}\right) = 0.
\end{align}
In order to make progress we first compute
\begin{align}
&g_{eff.}^{AB} g_{AB} = 4 + \frac{1}{1 + k\left( \lambda \right) e^{-2A} {\dot{\tau}}^2} \\
& \delta g_{eff}^{AB} = - g_{eff}^{AM} g_{eff.}^{BN} h_{MN} \\
& \delta  \sqrt{1 + k\left(\lambda\right) g^{-1} d \tau \otimes d \tau} = - \frac{1}{2} \frac{k\left( \lambda \right) {\dot{\tau}}^2}{\sqrt{1+e^{-2A} k\left(\lambda\right) {\dot{\tau}}^2}} h^{zz} \\
& g_{eff}^{AB}h_{AC}g_{BD} + g_{eff.}^{AB} g_{AC} h_{BD} = 2 g_{eff.}^{AB} h_{A\left(C\right.}g_{\left.D\right)B}
\end{align}
and define $G = \sqrt{1+e^{-2A} k\left(\lambda\right) {\dot{\tau}}^2}$. The linearized Einstein equations are then
\begin{align}
&\delta R_{CD} - \frac{1}{2} h_{CD} \left[ -\frac{2}{3} V_g + \frac{1}{3} x V_f G \left(4 + \frac{1}{G^2} \right)\right] - \frac{1}{6} g_{CD} x V_f \left[ -\frac{1}{2} \frac{k\left(\lambda\right) {\dot{\tau}}^2}{G^2} h^{zz} \left(4 + \frac{1}{G^2}\right) - \right. \notag \\
& \left. - G g_{eff}^{AM} g_{eff}^{BN} g_{AB} h_{MN} + G g_{eff}^{AB} h_{AB}  \right]  + \frac{1}{2} x V_f \left[ - \frac{1}{2} \frac{k\left(\lambda\right) {\dot{\tau}}^2}{G} h^{zz} g_{eff}^{AB} g_{AC} g_{BD}  - \right. \notag \\
&\left. - G g_{eff}^{AM} g_{eff}^{BN} h_{MN} g_{AC} g_{BD} + 2 G g_{eff}^{AB} h_{A(C}g_{D)B} \right] = 0
\end{align}
The perturbed Ricci tensor is given by
\begin{align}
&\delta R_{CD} = R_{P(C}h_{D)}^P - R_{CPDQ}h^{PQ} - \frac{1}{2} \nabla^2 h_{CD} \\
&R_{P(C}h_{D)}^P = \frac{1}{2} h_{CD} \left[ - \frac{2}{3} V_g + \frac{1}{3} x V_f G \left( 4 + \frac{1}{G^2}\right) \right] + \frac{4}{3} \nabla_P \Phi \nabla_{(C}\Phi h_{D)}^P - \frac{1}{2} x V_f G g_{eff}^{AB} h_{A(C}g_{D)B}
\end{align}
where in the last equation we have used Einstein equations.
Putting everything together we get
\begin{align}
&\nabla^2 h_{CD} + 2 R_{CPDQ} h^{PQ} - \frac{8}{3} \nabla_P \Phi \nabla_{(C} \Phi h_{D)}^P + \frac{1}{3} x V_f g_{CD} \left[ - \frac{1}{2} \frac{k\left(\lambda\right) {\dot{\tau}}^2}{G} h^{zz} \left( 4 + \frac{1}{G^2} \right) - \right.  \notag \\
&\left. - G g_{eff}^{AM} g_{eff}^{BN} g_{AB} h_{MN} + G g_{eff}^{AB} h_{AB}  \right] - x V_f \left[ - \frac{1}{2} \frac{k\left(\lambda\right) {\dot{\tau}}^2}{G} h^{zz} g_{eff.}^{AB} g_{AC} g_{BD}  - \right. \notag \\
& \left. - G g_{eff}^{AM} g_{eff}^{BN} g_{AC} g_{BD} h_{MN} + G g_{eff}^{AB} h_{A(C} g_{D)B} \right] = 0
\end{align}
We can make further progress by computing
\begin{align}
&g_{eff}^{AM} g_{eff}^{BN} g_{AB} h_{MN}  = e^{-2A} \left( \frac{h^{zz}}{G^4} + \eta^{\alpha \beta} h_{\alpha \beta} \right) \\
&g_{eff}^{AB}h_{AB} = e^{-2A} \left( \frac{h_{zz}}{G^2} + \eta^{\alpha \beta} h_{\alpha \beta} \right)
\end{align}
Using the definition of G and the above equations one can show that
\begin{align}
- \frac{1}{2} \frac{k\left(\lambda\right) {\dot{\tau}}^2}{G} h^{zz} \left( 4 + \frac{1}{G^2} \right) - G g_{eff}^{AM} g_{eff}^{BN} g_{AB} h_{MN} + G g_{eff}^{AB} h_{AB} = e^{-2A} h_{zz} \frac{5 G^2 - 4 G^4 - 1}{2 G^3}
\end{align}
We can now write the equation of motion for $h_{CD}$:
\begin{align}
&\nabla^2 h_{CD} + 2 R_{CPDQ}h^{PQ} - \frac{8}{3} \nabla_P \Phi \nabla_{(C}\Phi h_{D)}^P + \frac{1}{3} x V_f g_{CD} e^{-2A} h_{zz} \frac{5 G^2 - 4 G^4 -1}{2G^3} - \notag \\
& - x V_f \left( - \frac{1}{2} \frac{k\left(\lambda\right) {\dot{\tau}}^2}{G} h^{zz} g_{eff}^{AB}g_{AC} g_{BD} - G g_{eff}^{AM} g_{eff}^{BN} g_{AC} g_{BD} h_{MN} + G g_{eff}^{AB} h_{A(C}g_{D)B} \right) = 0
\end{align}

We now wish to analyse the EOM of $h_{CD}$, that is, study it when $CD = zz$, $CD = z\mu$ and $CD = \mu \nu$. The equations for these cases are respectively
\begin{align}
&\nabla^2 h_{zz} + 2 R_{zPzQ} h^{PQ} - \frac{8}{3} {\dot{\Phi}}^2 e^{-2A} h_{zz} + x V_f h_{zz} \frac{G^2 - 2 G^4 +1}{3 G^3} = 0 \\
& \nabla^2 h_{z\mu} + 2 R_{zP\mu Q} h^{PQ} - \frac{4}{3} {\dot{\Phi}}^2 e^{-2A} h_{z \mu} + x V_f \frac{h_{z\mu}}{G} = 0 \\
& \nabla^2 h_{\mu \nu} + 2 R_{\mu P \nu Q} h^{PQ} + \frac{1}{2} x V_f \frac{G^2 - 1}{G} h_{zz} \eta_{\mu \nu} + \frac{1}{3} x V_f \eta_{\mu \nu} h_{zz} \frac{5 G^2 - 4 G^4 -1}{2 G^3} = 0
\end{align}
If we compute the Riemann tensor components in this background we find
\begin{align}
&R_{zPzQ} h^{PQ} = - e^{-2A} h_{\alpha}^\alpha \ddot{A} \\
&R_{zP\mu Q} h^{PQ} = e^{-2A} h_{z \mu} \ddot{A} \\
& R_{\mu P \nu Q} h^{PQ} = e^{-2A} h_{\mu \nu} {\dot{A}}^2 - e^{-2A} h_{\alpha}^\alpha \eta_{\mu \nu} {\dot{A}}^2 - e^{-2A} h_{zz} \eta_{\mu \nu} \ddot{A}
\end{align}
and for the divergence terms we get
\begin{align}
& \nabla^2 h_{zz} = 2 e^{-2A} h_{\alpha}^\alpha {\dot{A}}^2 - 10 e^{-2A} h_{zz}  {\dot{A}}^2  - e^{-2A} \dot{A} \partial_z h_{zz} - 2 e^{-2A} h_{zz} \ddot{A} + e^{-2A} \partial_z^2{h_{zz}} - 4 e^{-2A} \dot{A} \partial_\alpha h_{z}^\alpha + e^{-2A} \partial_\alpha \partial^\alpha h_{zz} \\
& \nabla^2 h_{z \mu} = - 9 e^{-2A} h_{z\mu} {\dot{A}}^2 - e^{-2A} \dot{A} \partial_z h_{z \mu} - 2 e^{-2A} h_{z \mu} \ddot{A} + e^{-2A} \partial_z^2 h_{z\mu} - 2 e^{-2A} \dot{A} \partial_\alpha h_{\mu}^\alpha + e^{-2A} \partial_\alpha \partial^\alpha h_{z\mu} + 2 e^{-2A} \dot{A} \partial_\mu h_{zz} \\
& \nabla^2 h_{\mu \nu} = e^{-2A} \left( -4 h_{\mu\nu} {\dot{A}}^2 + 2 h_{zz} \eta_{\mu\nu} {\dot{A}}^2 - \dot{A} \partial_z h_{\mu \nu} -2 h_{\mu \nu} \ddot{A} + \partial_z^2 h_{\mu \nu} + \partial_\alpha \partial^\alpha h_{\mu \nu} + 2 \dot{A} \partial_\mu h_{z \nu} + 2 \dot{A} \partial_\nu h_{z \mu}\right).
\end{align}

We are interested in solution with $h_{zz} = 0 = h_{z\mu}$. Then from the equations of $h_{zz}$ and $h_{z\mu}$ we can conclude that
\begin{align}
2 e^{-2A}  \left({\dot{A}}^2 - \ddot{A}\right) h_{\alpha}^\alpha = 0 \Leftrightarrow h_{\alpha}^\alpha = 0 \\
-2 e^{-2 A} \dot{A} \partial_\alpha h^{\alpha}_\mu = 0 \Leftrightarrow \partial^\alpha h_{\alpha \mu} = 0 \\
\nabla^2 h_{\mu \nu} + 2 {\dot{A}}^2 e^{-2A} h_{\mu \nu} = 0
\end{align}
where the last equation is the IHQCD result. So we have show, that in the Einstein frame the EOM of the TT components of the graviton in this family of backgrounds is the same as in IHQCD.

The last result means that in the string frame the equation of motion of the graviton will also be the same as in IHQCD. This follows from the fact that $h^E_{\alpha \beta} = e^{- 4 \Phi / 3} h^S_{\alpha \beta}$ and $A = A_E = A_S - 2 \Phi / 3$. Hence all the phenomenological machinery for the spin J fields in IHQCD can be reproduced here. Having said this,
the proposed equation of motion for the spin J fields in the graviton Regge trajectory is
\begin{align}
&\left( \nabla^2 - 2 e^{-2A_s} \dot{\Phi} \nabla_z - \frac{\Delta ( \Delta - 4 )}{L^2} + J {\dot{A}}^2 e^{-2A_s} + \left( J - 2\right) e^{-2A_s} \left( a \ddot{\Phi} + b \left( \ddot{A} - {\dot{A}}^2 \right) + c {\dot{\Phi}}^2 + d \dot{A} \dot{\Phi} + e {\dot{\tau}}^2  \right)\right) h^{TT}_{\alpha_1 \dots \alpha_J} = 0, \\
&\frac{\Delta ( \Delta - 4 )}{L^2}  = \frac{2}{ls^2} \left(J - 2 \right) \left( 1 + \frac{f}{\sqrt{\lambda}} \right) + \frac{J^2 -4}{\lambda^{4/3}}
\end{align}
This equation is almost the same as the one derived for IHQCD except for the following: i) in the IHQCD case the term $\dot{A} \dot{\Phi}$ does not appear because it can be removed through the use of the equations of motion. Here that is not the case; ii) we have not included the terms $\ddot{\tau}$, $\dot{\tau} \dot{A}$ and $\dot{\tau} \dot{\Phi}$ because the background is symmetric through $\tau \rightarrow - \tau$; iii) I have checked in the Mathematica notebook SpinJ EOMs the spin J fields behave as $h^{TT}_{\alpha_1 \dots \alpha_J} \sim c_1 z^2 + c_2 z^{2-2 J}$ which is the same as  $h^{TT}_{\alpha_1 \dots \alpha_J} \sim c_1 z^{\Delta - J} + c_2 z^{4 - \Delta -J}$ for $\Delta = 2 + J$, i.e. the free theory result.


\section{Vector Meson Trajectory}

In this section we will discuss how to include in our framework the exchange of the vector meson trajectory to compute holographically $F_2$ and $F_L$. Before starting the holographic computation let's review what Donnachie and Landshoff have done. More details in~\cite{donnachie_dosch_landshoff_nachtmann_2002}.

A fit to small-$x$ data is possible with the following ansatz
\begin{equation}
F_2 \left(x, Q^2\right) = \sum_{i = 0, 1, 2} f_i\left(Q^2\right) x^{- \epsilon_i},
\end{equation}
 where the $i=0$ corresponds to hard-pomeron exchange, $i=1$ corresponds to soft-pomeron exchange and $i = 2$ comes from $(f_2, a_2)$ exchange. Only the quantum numbers with $PC=++$ are exchanged.
 The precise definition of the coefficient functions $f_i\left(Q^2\right)$ will not be relevant to our discussion so we will omit them. The above equation can be also generalized to larger values of $x$ but we need to modify the ansatz in order to impose that $F_2\left(x, Q^2\right)$ vanishes as $x \rightarrow 1$ for all values of $Q^2$. Dimensional counting rules suggest the generalization
\begin{equation}
F_2\left(x, Q^2\right) = f_0 \left(Q^2\right) x^{-\epsilon_0} {\left(1-x\right)}^7 +  f_1 \left(Q^2\right) x^{-\epsilon_1} {\left(1-x\right)}^7 +  f_2 \left(Q^2\right) x^{-\epsilon_2} {\left(1-x^2\right)}^3.
\end{equation} 

The hard and soft pomeron are incorporated in our model when we consider the exchange of the graviton's Regge trajectory in the bulk as we have seen in previous papers. The quantum numbers~$I^G\left(J^{PC}\right)$ of $a_2$ and $f_2$ are respectively $1^{-} \left(2^{++}\right)$ and $0^{+}\left(2^{++}\right)$. With the graviton's Regge trajectory in mind, the first step to model these trajectories holographically could be finding the EOM of the fluctuations with these quantum numbers. Then we could generalize in the same way as we did with the graviton. However, in the model we are considering, the only present fluctuations with $J^{PC} = 2^{++}$ are identified with the tensor glueballs and not with these mesons. The way we have around this is to use the fact that the $\rho$ meson, $f_2$ and $a_2$ trajectories are degenerate and apply the procedure to the EOM of the $\rho$ mesons.

Another issue is the coupling in the bulk of the virtual photon with the fields of  $\left(a_2, f_2\right)$ trajectory. One strategy is to compute the cubic coupling of the vector $U\left(1\right)$ gauge field. I have done this and it is zero. The solution I have found around this is to use the that $\left(a_2, f_2\right)$ have the same quantum numbers as the hard and soft pomeron. Based on this we assume the coupling between the $U\left(1\right)$ gauge field dual to the off-shell photon with the meson trajectory is the same as the coupling with the graviton's Regge trajectory.

\subsection{EOM of the $\left(a_2, f_2\right)$ trajectory}

The equation (\ref{eq:VM_action}) can be rewritten as
\begin{equation}
S = - \frac{1}{4} M^3 N_c N_f \int dz d^4 x V_f w^2 e^{5A} G F_{AB}F^{AB} = - \frac{1}{4} M^3 N_c N_f \int dz d^4 x \sqrt{-g} V_f w^2 G F_{AB}F^{AB},
\label{eq:VM_new_action}
\end{equation}
where here the notation $F_{AB}F^{AB}$ means
\begin{equation}
F_{AB}F^{AB} = F_{AB} g_{\rm{eff.}}^{AC} g_{\rm{eff.}}^{BD} F_{CD} =  \frac{2}{G^2}e^{-4A}\left( \frac{1}{2} G^2 V_{\mu \nu} V^{\mu \nu} + \partial_z V_\mu \partial_z V^\mu \right)
\end{equation}
The equation of motion of the fluctuations can now be rewritten as
\begin{equation}
\nabla_A \left(V_f w^2 G F^{AB} \right) = 0
\end{equation}
The above equations are valid in the Einstein frame. However later we will need to work with expressions valid in the string frame. The action (\ref{eq:VM_new_action}) in the string frame takes the form
\begin{equation}
S = - \frac{1}{4} M^3 N_c N_f \int d^5 X \sqrt{-g_s} e^{-\frac{10}{3} \Phi} V_f w_s^2 G F_{AB}g^{AC}_{\rm{eff. (s)}}g^{BD}_{\rm{eff. (s)}}F_{CD}
\label{eq:VM_new_action_string_frame}
\end{equation}
The equation of motion is then
\begin{equation}
\nabla_{A} \left(e^{-\frac{10}{3} \Phi} V_f w_s^2 G g^{AC}_{\rm{eff. (s)}}g^{BD}_{\rm{eff. (s)}}F_{CD}  \right) = 0.
\label{eq:U1_eom_cov_der}
\end{equation}
In order to simplify the notation from now on we assume all the warp factors and effective metrics are in the string frame.

It is not known what is the equation of motion of the spin J fields. In our approach, as a first step, we will do the substitution
\begin{equation}
F_{CD} = \nabla_C h_D - \nabla_D h_C \Rightarrow \nabla_C h_{D a_2 \cdots a_J} - \nabla_D h_{C a_2 \cdots a_J}
\label{eq:spin_j_sub}
\end{equation}
and  hence the equation of motion becomes
\begin{align}
\frac{1}{e^{-\frac{10}{3} \Phi} V_f w_s^2 G}\nabla_A \left(e^{-\frac{10}{3} \Phi} V_f w_s^2 G \right)&g^{AC}_{\rm{eff. (s)}}g^{BD}_{\rm{eff. (s)}}\left(\nabla_C h_{D a_2 \cdots a_J} - \nabla_D h_{C a_2 \cdots a_J}\right) + \notag \\
&+\nabla_A \left[g^{AC}_{\rm{eff. (s)}}g^{BD}_{\rm{eff. (s)}}\left(\nabla_C h_{D a_2 \cdots a_J} - \nabla_D h_{C a_2 \cdots a_J}\right)\right] = 0
\label{eq:spin_J_eom_cov_term}
\end{align}

We now evaluate each of these terms. Because the background fields are only functions of $z$ the first term is
\begin{equation}
\partial_z\left( e^{-\frac{10}{3} \Phi} V_f w_s^2 G \right) g^{zz}_{\rm{eff. (s)}}g^{BD}_{\rm{eff. (s)}}\left(\nabla_z h_{D a_2 \cdots a_J} - \nabla_D h_{z a_2 \cdots a_J}\right)
\label{eq:spin_j_eom_first_term}
\end{equation}
For $B = z$ the above expression is 0, so all we need to do is to deal with the $B=\mu$ case. Later we will only be interested in the TT components of the spin J fields. This means $a_i = \alpha_i, \, i = 1,\cdots, J$.
The auxiliary computations
\begin{align}
& \nabla_z h_{\nu \alpha_2 \cdots \alpha_J} = \partial_z h_{\nu \alpha_2 \cdots \alpha_J} - J \dot{A} h_{\nu \alpha_2 \cdots \alpha_J}, \\
& \nabla_\nu h_{z \alpha_2 \cdots \alpha_J} = - \dot{A} h_{\nu \alpha_2 \cdots \alpha_J},
\label{eq:spin_J_cod_id_1}
\end{align}
allow us to write the first term as
\begin{equation}
\begin{cases}
0, & B = z \\
\frac{\partial_{z}\left( e^{-\frac{10}{3} \Phi} V_f w_s^2 G \right)}{ e^{-\frac{10}{3} \Phi} V_f w_s^2 G} \frac{e^{-4A}}{G^2} \eta^{\mu \nu} \left[ \partial_z h_{\nu \alpha_2 \cdots \alpha_J} - \left(J-1\right) \dot{A} h_{\nu \alpha_2 \cdots \alpha_J} \right], \, &B = \mu
\end{cases}
\end{equation}

Using the Leibniz rule the second term can split into three terms. They are
\begin{align}
&\left(\nabla_A g^{AC}_{\rm{eff.}}\right) g^{BD}_{\rm {eff.}} \left(\nabla_C h_{D \alpha_2 \cdots \alpha_J} - \nabla_D h_{C \alpha_2 \cdots \alpha_J}\right) \\
&g^{AC}_{\rm{eff.}}\left(\nabla_A g^{BD}_{\rm{eff.}}\right) \left(\nabla_C h_{D \alpha_2 \cdots \alpha_J} - \nabla_D h_{C \alpha_2 \cdots \alpha_J}\right) \\
& g^{AC}_{\rm{eff.}}g^{BD}_{\rm{eff.}} \left( \nabla_A \nabla_C h_{D \alpha_2 \cdots \alpha_J} - \nabla_A \nabla_D h_{C \alpha_2 \cdots \alpha_J}\right)
\end{align}
Let's start with the first one. For $B = z$ the first term becomes
\begin{equation}
\left(\nabla_A g^{AC}_{\rm{eff.}}\right) g^{zz}_{\rm {eff.}} \left(\nabla_C h_{z \alpha_2 \cdots \alpha_J} - \nabla_z h_{C \alpha_2 \cdots \alpha_J}\right) = \left(\nabla_A g^{A\mu}_{\rm{eff.}}\right) g^{zz}_{\rm {eff.}} \left(\nabla_\mu h_{z \alpha_2 \cdots \alpha_J} - \nabla_z h_{\mu \alpha_2 \cdots \alpha_J}\right) = 0,
\end{equation}
since $\nabla_A g^{A\mu}_{\rm{eff.}} = 0$.
For $B = \mu$ the first term becoms
\begin{equation}
\left(\nabla_A g^{AC}_{\rm{eff.}}\right) g^{\mu\nu}_{\rm {eff.}} \left(\nabla_C h_{\nu \alpha_2 \cdots \alpha_J} - \nabla_\nu h_{C \alpha_2 \cdots \alpha_J}\right) = \left(\nabla_A g^{Az}_{\rm{eff.}}\right) g^{\mu\nu}_{\rm {eff.}} \left(\nabla_z h_{\nu \alpha_2 \cdots \alpha_J} - \nabla_\nu h_{z \alpha_2 \cdots \alpha_J}\right).
\end{equation}
Using the identities of equation (\ref{eq:spin_J_cod_id_1}) and
\begin{align}
\nabla_A g^{Az}_{\rm{eff.}} = e^{-2A} \left( 4 \frac{\dot{A}}{G^2} - 2 \frac{\dot{G}}{G^3} - 4 \dot{A} \right),
\end{align}
we can show that the first term is
\begin{equation}
\begin{cases}
0, & B = z \\
\left( 4 \frac{\dot{A}}{G^2} - 2 \frac{\dot{G}}{G^3} - 4 \dot{A} \right) e^{-4A} \eta^{\mu \nu} \left[ \partial_z h_{\nu \alpha_2 \cdots \alpha_J} - \left(J-1\right) \dot{A} h_{\nu \alpha_2 \cdots \alpha_J} \right],  &B = \mu
\end{cases}
\end{equation}

The value of $\nabla_A g^{BD}_{\rm{eff.}}$ for $B = z$ is $\partial_A g^{zD}_{\rm{eff.}}$ and hence the second term is zero for $B = z$. For $B = \mu$ we have
\begin{equation}
\nabla_A g^{\mu D}_{\rm{eff.}} = \partial_A g^{\mu D}_{\rm{eff.}} + \Gamma^{\mu}_{AM} g^{MD}_{\rm{eff.}} + \Gamma^{D}_{AM} g^{\mu M}_{\rm{eff.}}
\end{equation}
and using the equations (\ref{eq:spin_J_cod_id_1}) the following identities hold
\begin{align}
& g^{AC}_{\rm{eff.}} \partial_A g^{\mu D}_{\rm{eff.}}  \left(\nabla_C h_{D \alpha_2 \cdots \alpha_J} - \nabla_D h_{C \alpha_2 \cdots \alpha_J}\right) = - 2 \frac{\dot{A}}{G^2} e^{-4A} \eta^{\mu \nu} \left[ \partial_z h_{\nu \alpha_2 \cdots \alpha_J} - \left(J-1\right) \dot{A} h_{\nu \alpha_2 \cdots \alpha_J} \right] \\
& g^{AC}_{\rm{eff.}} \Gamma^{\mu}_{AM} g^{MD}_{\rm{eff.}} \left(\nabla_C h_{D \alpha_2 \cdots \alpha_J} - \nabla_D h_{C \alpha_2 \cdots \alpha_J}\right) = 0 \\
& g^{AC}_{\rm{eff.}} \Gamma^{D}_{AM} g^{\mu M}_{\rm{eff.}} \left(\nabla_C h_{D \alpha_2 \cdots \alpha_J} - \nabla_D h_{C \alpha_2 \cdots \alpha_J}\right) = \dot{A} \left(1 + \frac{1}{G^2}\right)e^{-4A}\eta^{\mu\nu}\left[ \partial_z h_{\nu \alpha_2 \cdots \alpha_J} - \left(J-1\right) \dot{A} h_{\nu \alpha_2 \cdots \alpha_J} \right] 
\end{align}
Putting everything together we can write the second term as
\begin{equation}
\begin{cases}
0, & B = z \\
\dot{A} \left(1 - \frac{1}{G^2} \right) e^{-4A} \eta^{\mu \nu} \left[ \partial_z h_{\nu \alpha_2 \cdots \alpha_J} - \left(J-1\right) \dot{A} h_{\nu \alpha_2 \cdots \alpha_J} \right],  &B = \mu
\end{cases}
\end{equation}

There is only one more term left. For $B = z$ the third term becomes
\begin{equation}
2 g^{\mu \nu}_{\rm{eff.}} g^{zz}_{\rm{eff.}} \nabla_\mu \nabla_{[\nu}h_{z] \alpha_2 \cdots \alpha_J} = \frac{e^{-4A}}{G^2} \eta^{\mu \nu} \left( \nabla_\mu \nabla_\nu h_{z \alpha_2 \cdots \alpha_J} -  \nabla_\mu \nabla_z h_{\nu \alpha_2 \cdots \alpha_J}  \right)
\end{equation}
If we use the following covariant derivative identities
\begin{align}
&\nabla_\mu \nabla_\nu h_{z \alpha_2 \cdots \alpha_J} = -\dot{A} \left( \partial_\mu h_{\nu \alpha_2 \cdots \alpha_J} + \partial_\nu h_{\mu \alpha_2 \cdots \alpha_J} \right) \\
&\nabla_\mu \nabla_z h_{\nu \alpha_2 \cdots \alpha_J} = \partial_z \partial_\mu h_{\nu \alpha_2 \cdots \alpha_J} - \left(J+1\right) \dot{A} \partial_\mu h_{\nu \alpha_2 \cdots \alpha_J}
\end{align}
we conclude that this term is also 0. For $B = \mu$ the third term can be rewritten as
\begin{equation}
2 \frac{e^{-4A}}{G^2} \eta^{\mu \nu} \nabla_{z} \nabla_{[z} h_{\nu] \alpha_2 \cdots \alpha_J} + 2 e^{-4A} \eta^{\alpha \beta} \eta^{\mu \nu} \nabla_{\alpha} \nabla_{[\beta} h_{\nu] \alpha_2 \cdots \alpha_J}
\end{equation}
The following identities
\begin{align}
&\nabla_z \nabla_z h_{\nu \alpha_2 \cdots \alpha_J} = \left[ \partial_z^2 - \dot{A} \left(2 J+1\right) \partial_z - J \ddot{A} + J \left(J+1\right) {\dot{A}}^2 \right] h_{\nu \alpha_2 \cdots \alpha_J} \\
&\nabla_z \nabla_\nu h_{z \alpha_2 \cdots \alpha_J} = - \ddot{A} h_{\nu \alpha_2 \cdots \alpha_J} - \dot{A} \partial_z h_{\nu \alpha_2 \cdots \alpha_J} + \left(J+1\right){\dot{A}}^2 h_{\nu \alpha_2 \cdots \alpha_J} \\
& \nabla_\alpha \nabla_\beta h_{\nu \alpha_2 \cdots \alpha_J} = \partial_\alpha \partial_\beta h_{\nu \alpha_2 \cdots \alpha_J} + \dot{A} \eta_{\alpha \beta} \left( \partial_z h_{\nu \alpha_2 \cdots \alpha_J} - \dot{A} J h_{\nu \alpha_2 \cdots \alpha_J}\right) - {\dot{A}}^2 \eta_{\alpha \nu} h_{\beta \alpha_2 \cdots \alpha_J} - {\dot{A}}^2 \sum_{i=2}^J \eta_{\alpha \alpha_i} h_{\nu \cdots \alpha_{i-1} \beta \alpha_{i+1} \cdots \alpha_J}
\end{align}
give the following expressions for the third term
\begin{equation}
\begin{cases}
0 \, ,B = z \\
e^{-4A} \eta^{\mu \nu} \left[\frac{1}{G^2}\left(\partial_z^2 - 2 J \dot{A} \partial_z - \left(J-1\right) \ddot{A} + \left(J^2-1\right) {\dot{A}}^2 \right)+ \left( \Box + 3\dot{A} \partial_z - 3 {\dot{A}}^2 \left(J-1\right) \right)  \right]h_{\nu \alpha_2 \cdots \alpha_J} \, , B = \mu
\end{cases}
\end{equation}

Plugging the above calculations in equation (\ref{eq:spin_J_eom_cov_term}) we conclude that the EOM of the spin J fields on the meson trajectory should have the term
\begin{align}
&\frac{\partial_z \left(e^{-\frac{10}{3}\Phi} V_f w_s^2 G\right)}{e^{-\frac{10}{3}\Phi} V_f w_s^2 G} \frac{e^{-2A}}{G^2} \left[\partial_z - \left(J-1\right) \dot{A} \right]h_{\alpha_1 \cdots \alpha_J} + e^{-2A}\left( \frac{4 \dot{A}}{G^2} - 2 \frac{\dot{G}}{G^3} - 4 \dot{A} \right) \left[\partial_z - \left(J-1\right) \dot{A} \right]h_{\alpha_1  \cdots \alpha_J} + \notag \\
&+e^{-2A} \left[\frac{1}{G^2}\left(\partial_z^2 - 2 J \dot{A} \partial_z - \left(J-1\right) \ddot{A} + \left(J^2-1\right) {\dot{A}}^2 \right)+ \left( \Box + 3\dot{A} \partial_z - 3 {\dot{A}}^2 \left(J-1\right) \right)  \right]h_{\alpha_1 \cdots \alpha_J}
\end{align}
We can also add terms that i) reduce to the equation of the $U\left(1\right)$ gauge field; ii) vanish in the case of AdS, i.e. $\Phi = \rm{const.}$, $\tau = \rm{const.}$ and $A =  - \log(z)$; have dimension inverse squared length compatible with the previous two constraints. There are 9 possible terms satisfying the last constraing: $\ddot{\Phi}$, $\ddot{A}$, $\ddot{\tau}$, ${\dot{\Phi}}^2$, ${\dot{\tau}}^2$, ${\dot{A}}^2$, $\dot{\Phi} \dot{\tau}$, $\dot{A}\dot{\tau}$ and $\dot{\Phi}\dot{A}$. All these terms must be proportional to $J-1$ in order to satisfy constraint i). Moreover, due to constraint ii) there must be a term proportional to $\ddot{A} - \dot{A}^2$. Because the background is symmetric under $\tau \rightarrow -\tau$ we drop $\ddot{\tau}$, $\dot{\Phi}\dot{\tau}$ and $\dot{A}\dot{\tau}$. This means a term of the form
\begin{equation}
\left(J-1\right) e^{-2A} \left[ a \ddot{\Phi} + b \left(\ddot{A} - {\dot{A}}^2\right) + c {\dot{\Phi}}^2 + d {\dot{\tau}}^2 + e \dot{\Phi} \dot{A} \right] h_{\alpha_1 \cdots \alpha_J}
\end{equation}
must be present.

Finally we can also add a mass term related to the dimension of the dual operator, wich requires a length scale $L$. According to the AdS/CFT correspondence the mass of a vector bulk field $m^2$ is related to the conformal dimension $\Delta$ of the dual operator through
\begin{equation}
m^2 L^2 = \left(\Delta - 1 \right) \left(\Delta - 3 \right)
\end{equation}
This term is absent in the equation of motion of the $U\left(1\right)$ gauge field because this bulk field is dual to the operator $\bar{\psi} \gamma^\mu \psi$ which has $\Delta = 3$ and spin 1. This operator is also a conserved current and hence $\Delta = 3$ and $J = 1$ is a protected point. To extend the mass term for other spin J fields consider the quadratic approximation to the $J\left(\nu\right), \, i \nu = \Delta - 2$ curve
\begin{equation}
J\left(\nu\right) \approx J_0 - \mathcal{D} \nu^2\, , \mathcal{D} = 1 - J_0
\end{equation}
We can invert the above equation to find $\Delta$ as a function of $J$.  The result is given by
\begin{equation}
\left(\Delta -1\right) \left(\Delta - 3 \right) = \frac{1-J}{J_0 - 1}
\end{equation}
In $\mathcal{N} = 4$ SYM in the strong coupling and Regge limit the intercept $J_0$ is related to the coupling constant $\lambda$ through $J_0 = 2 - \frac{2}{\sqrt{\lambda}}$ and hence 
\begin{equation}
\frac{\left(\Delta -1\right) \left(\Delta - 3 \right)}{L^2} = \frac{\left(1-J\right)}{L^2} + \frac{2\left(1-J\right)}{L^2\sqrt{\lambda}}\left(1+\frac{2}{\sqrt{\lambda}}\right).
\end{equation}
Also, from $\mathcal{N} = 4$ SYM , $\sqrt{\lambda} = R^2 / l^2_s$ where $R$ is the AdS radius. Then
\begin{equation}
\frac{\left(\Delta -1\right) \left(\Delta - 3 \right)}{L^2} = -\frac{\left(J-1\right)}{L^2} - \frac{2 l_s^2}{R^2 L^2} \left(J-1\right) \left(1+\frac{2}{\sqrt{\lambda}}\right).
\end{equation}
We then propose the following addition to the equation of motion of the spin J fields in the meson trajectory
\begin{equation}
\frac{\left(\Delta -1\right) \left(\Delta - 3 \right)}{L^2} = -\frac{3\left(J-1\right)}{l_s^2} \left(1+\frac{d}{\sqrt{\lambda}}\right) + \frac{J^2 - 1}{e^{\frac{4}{3}\Phi}}
\end{equation}
The last term was included in order to get the correct UV behaviour of the spin J fields
\begin{align}
h^{TT}_{\alpha_1 \cdots \alpha_J} \sim c_1 z^2 + c_2 z^{2- 2J}
\end{align}

With all the above considerations we will use the following equation
\begin{align}
&\frac{\partial_z \left(e^{-\frac{10}{3}\Phi} V_f w_s^2 G\right)}{e^{-\frac{10}{3}\Phi} V_f w_s^2 G} \frac{e^{-2A}}{G^2} \left[\partial_z - \left(J-1\right) \dot{A} \right]h_{\alpha_1 \cdots \alpha_J} + e^{-2A}\left( \frac{4 \dot{A}}{G^2} - 2 \frac{\dot{G}}{G^3} - 4 \dot{A} \right) \left[\partial_z - \left(J-1\right) \dot{A} \right]h_{\alpha_1  \cdots \alpha_J} + \notag \\
&+e^{-2A} \left[\frac{1}{G^2}\left(\partial_z^2 - 2 J \dot{A} \partial_z - \left(J-1\right) \ddot{A} + \left(J^2-1\right) {\dot{A}}^2 \right)+ \left( \Box + 3\dot{A} \partial_z - 3 {\dot{A}}^2 \left(J-1\right) \right)  \right]h_{\alpha_1 \cdots \alpha_J} - \notag \\
& - \frac{\left(\Delta - 1 \right)\left(\Delta - 3\right)}{L^2} h_{\alpha_1 \cdots \alpha_J}^{TT} + \left(J-1\right) e^{-2A} \left[ a \ddot{\Phi} + b \left(\ddot{A} - {\dot{A}}^2\right) + c {\dot{\Phi}}^2 + d {\dot{\tau}}^2 + e \dot{\Phi} \dot{A} \right] h_{\alpha_1 \cdots \alpha_J} = 0
\label{eq:spin_j_meson_eom}
\end{align}

\subsection{Schrodinger form}
The equation (\ref{eq:spin_j_meson_eom}) can be brought to Schrodinger form by first rewriting it in terms of the u variable. After that we can write the spin J field as
\begin{equation}
h^{TT}_{\alpha_1 \cdots \alpha_J} = \epsilon_{\alpha_1 \cdots \alpha_J} e^{i q \cdot x} \frac{e^{\left(J-1\right)A}}{\sqrt{e^{-\frac{10}{3}\Phi}V_f w_s^2 e^A}} \psi\left(u\right),
\end{equation}
where $\psi$ satisfies the Schrodinger equation
\begin{equation}
- \frac{d^2 \psi}{du^2} + V_J \left(u\right) \psi = t \psi.
\end{equation}
The Schrodinger potential is given by
\begin{align}
&V_J\left(u\right) = V_1\left(u\right) + \left(J^2 - 1\right) e^{2 A - \frac{4}{3} \Phi} - \frac{3}{l_s^2} \left(J-1\right) e^{2A} \left(1 + \frac{f}{\sqrt{\lambda}}\right) - \notag \\
& - \left(J-1\right) \left[a \left( G \dot{G} \dot{\Phi} + G^2 \ddot{\Phi} \right) + b \left( G^2 \ddot{A} + G \dot{G} \dot{A} -G^2 \dot{A}^2\right) +c G^2 \dot{\Phi}^2 + d G^2 \dot{A}\dot{\Phi} + e G^2 \dot{\tau}^2 \right]
\label{eq:meson_schrodinger_potential}
\end{align}
Here the dots mean derivatives with respect to the u variable.
We will denote the eigenvalues of this Schrodinger potential as $t_n\left(J\right)$ and the corresponding eigenfunctions as $\psi_n\left(J, u\right)$.

\subsection{Spin J field propagator}
The propagator obeys equation of the type
\begin{equation}
\mathcal{D} \Pi_{a_1 \cdots a_J, b_1 \cdots b_J} = i g_{a_1(b_1} \cdots g_{|a_J|b_J)} \delta_5 \left(X, \bar{X}\right) \frac{G}{e^{-\frac{10}{3}\Phi}V_f w_s^2},
\end{equation}
where $\mathcal{D}$ is a differential operator defined by equaton (\ref{eq:spin_j_meson_eom}).
In the Regge limit we will be interested in the components $\Pi_{+\cdots+,-\cdots-}$ and for this particular case
\begin{equation}
\mathcal{D} \Pi_{+\cdots+,-\cdots-} = i {\left( - \frac{e^{2A}}{2}\right)}^J  \delta_5 \left(X, \bar{X}\right) \frac{G}{e^{-\frac{10}{3}\Phi}V_f w_s^2}
\end{equation}
Consider now that
\begin{equation}
\int \frac{dw^+ dw^-}{2} \Pi_{+\cdots+, - \cdots -} = -i {\left(-\frac{1}{2}\right)}^J e^{\left(J-1\right)(A+\bar{A})} G_J \left(u, \bar{u}, l_\perp\right)
\end{equation}
Applying the operator $\mathcal{D}$ on the left-hand side of this equation gives
\begin{equation}
\int \frac{dw^+ dw^-}{2} \mathcal{D} \Pi_{+\cdots+, - \cdots -} = \frac{i}{{\left(-2\right)}^J} e^{\left(J-1\right)\left(A+\bar{A}\right)} \delta_3\left(x, \bar{x}\right) \frac{G}{e^{-\frac{10}{3}\Phi}V_f w_s^2}.
\end{equation}
Applying the same operator on the right-hand side gives
\begin{align}
&\mathcal{D} \left[ e^{\left(J-1\right) \left(A+\bar{A}\right)} G_J \left(u, \bar{u}, l_\perp\right) \right] = e^{\left(J-1\right)\left(A+\bar{A}\right)} \mathcal{D}_3 G_J \left(u, \bar{u}, l_\perp\right) \notag \\
&\mathcal{D}_3 = e^{-2A} \partial_u^2 + e^{-2A} \left( \dot{A} - \frac{10}{3} \dot{\Phi} + 2 \frac{w_s' \dot{\Phi}}{w_s} + \frac{\partial_u V_f}{V_f} \right) \partial_u + e^{-2A} \partial^2_{l_\perp} + \notag \\
& \left(J-1\right) e^{-2A} \left[ a \left(G^2\ddot{\Phi} + G \dot{G} \dot{\Phi} \right) + b \left(G^2 \ddot{A} + G \dot{G} \dot{A} - \dot{G}^2 {\dot{A}}^2\right) + c G^2 {\dot{\Phi}}^2 + d G^2 {\dot{\tau}}^2 + e G^2 \dot{\Phi} \dot{A} \right] + \notag \\
& + \frac{3}{l_s^2} \left(J-1\right)\left(1+\frac{d}{\sqrt{\lambda}}\right) - \frac{J^2-1}{\lambda^{4/3}}
\end{align}
The RHS then becomes
\begin{equation}
- \frac{i}{{\left(-2\right)}^J} e^{\left(J-1\right)\left(A+\bar{A}\right)} \mathcal{D}_3 G_J \left(u, \bar{u}, l_\perp\right)
\end{equation}
We conclude that $G_J$ satisfies the following equation
\begin{equation}
\mathcal{D}_3 G_J \left(u, \bar{u}, l_\perp\right) = - \delta_3 \left(x, \bar{x}\right) \frac{G}{e^{-\frac{10}{3} \Phi} V_f w_s^2}
\label{eq:transverse_propagator}
\end{equation}

To solve equation (\ref{eq:transverse_propagator}) consider
\begin{equation}
\mathcal{D}_3 G_J\left(u, l_\perp\right) = 0
\end{equation}
and the following ansatz
\begin{equation}
G_J\left(u, l_\perp\right) = \frac{e^{i q \cdot l_\perp}}{\sqrt{e^{-\frac{10}{3}\Phi}V_f w_s^w e^{A}}} \psi\left(u\right).
\end{equation}
It follows
\begin{equation}
- \frac{d^2 \psi}{d u^2} + V_J \left(u\right) \psi = t \, \psi
\end{equation}
where $V_J$ is the same as in equation (\ref{eq:meson_schrodinger_potential}). Since the eigenfunctions of the Schrodinger potential satisfy $\sum_n \psi_n (u) {\psi_n (\bar{u})}^* = \delta(u - \bar{u})$ once can show
\begin{equation}
\left( - \frac{d^2}{du^2} + V - t \right) \sum_n \frac{\psi_n\left(u\right) {\psi_n\left(\bar{u}\right)}^*}{t_n \left(J\right) - t} = \delta\left(u - \bar{u} \right)
\end{equation}

With the above considerations one can show that the solution to equation (\ref{eq:transverse_propagator}) is
\begin{equation}
G_J \left(u, \bar{u}, t \right) = \left. \frac{1}{\sqrt{e^{-\frac{10}{3}\Phi}V_f w_s^2 e^A}}\right|_u  \left.\frac{1}{\sqrt{e^{-\frac{10}{3}\Phi}V_f w_s^2 e^A}}\right|_{\bar{u}}  \sum_n \frac{\psi_n\left(u\right) \psi_n \left(\bar{u}\right)}{t_n\left(J\right) - t}
\end{equation}

\subsection{Couplings to the spin J fields in the meson trajectory}

Before computing the holographic contribution of the holographic meson trajectory to scattering amplitudes we need to specify the couplings of the external states with the spin J fields of the meson trajectory. We assume such fields come from the open string sector and hence the coupling between a $U(1)$ gauge field  with these fields is given by
\begin{equation}
k_J \int d^5X \sqrt{-g} e^{-\Phi} \left(F_{b_1 a} \nabla_{b_2} \cdots \nabla_{b_{J-1}} F^{a}_{\, \, \, b_J}  \right) h^{b_1 \cdots b_J}
\label{eq:open_string_spin_J_coupling}
\end{equation}
We will also assume that a scalar field $\Upsilon$ coming from the open string sector is dual to the proton target and because of that we assume the substitution
\begin{equation}
\bar{k}_J \int d^5 X \sqrt{-g} e^{-\Phi} h_{b_1 \cdots b_J} \left( \Upsilon \nabla^{b_1} \cdots \nabla^{b_J} \Upsilon \right)
\end{equation}
In the Regge limit only the TT components of the spin J fields will be important and hence the previous couplings reduce to 
\begin{align}
&k_J \int d^5X \sqrt{-g} e^{-\Phi} \left(F_{\beta_1 \alpha} \nabla_{\beta_2} \cdots \nabla_{\beta_{J-1}} F^{\alpha}_{\, \, \, \beta_J}  \right) h^{\beta_1 \cdots \beta_J} \\
&\bar{k}_J \int d^5 X \sqrt{-g} e^{-\Phi}  \left( \Upsilon \nabla_{\beta_1} \cdots \nabla_{\beta_J} \Upsilon \right)h^{\beta_1 \cdots \beta_J}
\label{eq:meson_spin_J_couplings}
\end{align}
Another possibility is to assume that the coupling of the $U\left(1\right)$ gauge field with the spin J fields of the meson trajectory is the same as in equation (\ref{eq:spin_J_graviton_coupling}).

%%%%%%%%%%%%%%%%%%%%%%%%%%%%%%%%%%%%%%
\section{Kinematics}
%%%%%%%%%%%%%%%%%%%%%%%%%%%%%%%%%%%%%%
We will use light-cone coordinates $\left( +, -, \perp \right)$, with metric given by $ds^2 = - dx^{+} dx^{-} + dx_\perp^2$, where $x_\perp \in \mathbb{R}^2$ is a vector in impact parameter space.
For the large $s$ the kinematics of hadron-hadron scattering are the following
\begin{align}
\label{eq:pp_kinematics}
&k_1=\left(\!\sqrt{s},\frac{m_1^2}{\sqrt{s}} ,0\right),\  \ k_3=-\left(\!\sqrt{s},\frac{ q_\perp^2 +m_1^2}{\sqrt{s}} , q_\perp \right)\!,\\
&k_2=\left(\frac{m_2^2}{\sqrt{s}},\sqrt{s} ,0\right),\  \ k_4=-\left(\frac{m_2^2+ q_\perp^2}{\sqrt{s}},\sqrt{s} ,-q_\perp \right).\notag
\end{align}
where $k_1$ and $k_2$ are the incoming hadron momenta and $k_3$ and $k_4$ are the outgoing hadron momenta.

For the case of $\gamma^{*} p$ scattering the expression for $k_2$ and $k_4$ are the same. However the momenta of the incoming off-shell photon $k_1$ and the momenta of the outgoing off-shell photon $k_3$ are given by
\begin{align}
\label{eq:gammap_kinematics}
k_1=\left(\!\sqrt{s},-\frac{Q^2}{\sqrt{s}} ,0\right),\  \ k_3=-\left(\!\sqrt{s},\frac{ q_\perp^2 -Q^2}{\sqrt{s}} , q_\perp \right)\!.\\
\end{align}
The corresponding polarisation vectors of the incoming and outgoing photons are respectively
\begin{align}
&n_1=
    \begin{cases}
      \left(0,0,1,0\right), & \lambda=1 \\
      \left(0,0,0,1\right), & \lambda=2 \\
      \frac{1}{Q} \left( \sqrt{s}, \frac{Q^2}{\sqrt{s}}, 0, 0 \right), & \lambda = 3
    \end{cases} \\
 &n_3=
    \begin{cases}
      \left(0,\frac{2 q_x}{\sqrt{s}},1,0\right), & \lambda=1 \\
      \left(0,\frac{2 q_y}{\sqrt{s}},0,1\right), & \lambda=2 \\
      \frac{1}{Q} \left(\sqrt{s}, \frac{Q^2+q_\perp^2}{\sqrt{s}}, q_\perp \right), & \lambda = 3
    \end{cases}   
\end{align}

\section{$pp$ scattering with holographic pomeron and meson exchange}
We now have all the ingredients to compute holographic scattering amplitudes. We start the hadron-hadron scattering case by considering general spin J exchange. Later we specify for the cases of fields belonging either to the graviton's Regge trajectory or to the vector meson trajectory. The scattering amplitude for spin J exchange is
\begin{equation}
\mathcal{A}_J = {\left(i k'_J\right)}^2 \int d^5X d^5\bar{X} \sqrt{-g} \sqrt{-\bar{g}} e^{- \Phi -\bar{\Phi}} \left(\Upsilon_1 \partial_{-}^J \Upsilon_3\right) \Pi^{-\cdots-, + \cdots +} \left(X, \bar{X}\right) \left(\bar{\Upsilon}_2 \bar{\partial}_{+}^J \bar{\Upsilon}_4\right)
\end{equation}
Lowering the indices of the propagator gives
\begin{equation}
\mathcal{A}_J = {\left(i k'_J\right)}^2 \int d^5X d^5\bar{X} \sqrt{-g} \sqrt{-\bar{g}} e^{- \Phi -\bar{\Phi}} {\left(-2 e^{-2 A }\right)}^J {\left(-2 e^{-2 \bar{A} }\right)}^J \left(\Upsilon_1 \partial_{-}^J \Upsilon_3\right) \Pi_{+\cdots+, - \cdots -} \left(X, \bar{X}\right) \left(\bar{\Upsilon}_2 \bar{\partial}_{+}^J \bar{\Upsilon}_4\right)
\end{equation}
Using the fact that $\Upsilon_i = \upsilon\left(z\right) e^{i k_i \dot x}$ and the kinematics of equation (\ref{eq:pp_kinematics}) we get
\begin{equation}
\mathcal{A}_J = - k_J^2 {\left(-1\right)}^J s^J \int \frac{d^5X}{2}\frac{d^5\bar{X}}{2} e^{5\left(A+\bar{A}\right)} e^{-\Phi - \bar{\Phi}} e^{-2J\left(A+\bar{A}\right)} {|\upsilon_1|}^2 {|\upsilon_2|}^2 e^{-i q_\perp \cdot \left(x_\perp - \bar{x}_\perp\right)} \Pi_{+ \cdots +, - \cdots -} \left(X, \bar{X}\right)
\end{equation}
Making the change of variable $w = x - \bar{x}$  and using the identity
\begin{equation}
\int d^2 l_\perp e^{- i q_\perp \cdot l_\perp}\int \frac{dw^+ dw^-}{2} \Pi_{+\cdots+, - \cdots -} \left(X, \bar{X}\right) = - \frac{i}{\left(-2\right)^J} e^{\left(J-1\right)\left(A+\bar{A}\right)} G_J \left(z, \bar{z}, t\right),
\end{equation}
the scattering amplitude can be rewritten as
\begin{align}
\mathcal{A}_J = i V \frac{k_J^2}{2^J} s^J \int dz d\bar{z} e^{4\left(A+\bar{A}\right)} e^{-J\left(A+\bar{A}\right)} e^{-\Phi - \bar{\Phi}} {|\upsilon_1|}^2 {|\upsilon_2|}^2  G_J \left(z, \bar{z}, t\right).
\end{align}
For the case of a spin J of the graviton's Regge trajectory
\begin{equation}
G_J \left(z, \bar{z}, t \right) = e^{\Phi + \bar{\Phi} - \frac{A+\bar{A}}{2}} \sum_n \frac{\psi_n \left(z\right) {\psi_n \left(\bar{z}\right)}^*}{t_n(J) - t},
\end{equation}
while for the case of a spin J in the vector meson trajectory we have
\begin{equation}
G_J \left(z, \bar{z}, t \right)  = \left.\frac{1}{\sqrt{e^{-\frac{10}{3}\Phi} V_f w_s^2 e^A}}\right|_z  \left.\frac{1}{\sqrt{e^{-\frac{10}{3}\Phi} V_f w_s^2 e^A}}\right|_{\bar{z}} \, \sum_n \frac{\psi_n \left(u\left(z\right)\right){\psi_n \left(u\left(\bar{z}\right)\right)}^*}{t_n(J)-t}
\end{equation}

Let's consider each case separately and start with the graviton case. We sum all even spin J exchanges with $J \geqslant 2$ using the Sommerfeld-Watson transform
\begin{equation}
\frac{1}{2} \sum_{J \geqslant 2} \left( s^J + {\left(-s\right)}^J \right) \rightarrow - \frac{\pi}{2} \int \frac{dJ}{2 \pi i} \frac{s^J + {\left(-s\right)}^J}{\sin\left(\pi J\right)}
\end{equation}
where we are assuming the analytic continuation of the scattering amplitude $\mathcal{A}_J$ to the complex J-plane. Deforming the J-plane integral and catching all the poles $J = j_n\left(t\right)$ defined by $t_n\left(J\right) ) = t$ we get
\begin{align}
\mathcal{A}^{\rm {gluon}} = - \frac{\pi}{2} \sum_n \frac{k_{j_n}^2}{2^{j_n}} s^{j_n} \left[i + \cot \left( \frac{\pi j_n}{2} \right) \right] \frac{d j_n}{dt} \int dz d\bar{z} e^{-\left(j_n - \frac{7}{2}\right) \left(A+\bar{A}\right)} {|\upsilon_1|}^2 {|\upsilon_2|}^2 \psi_n \left(z\right) {\psi_n \left(\bar{z}\right)}^*
\end{align}
In the scattering domain of $t < 0$ these poles are in the real axis for $J < 2$. 

For the case of the $\rho$ meson trajectory we do a similar procedure. We sum all odd spin J exchanges with $J \geqslant 1$ using the Sommerfeld-Watson transform
\begin{equation}
\frac{1}{2} \sum_{J \geqslant 1} \left( s^J - {\left(-s\right)}^J \right) \rightarrow - \frac{\pi}{2} \int \frac{dJ}{2 \pi i} \frac{s^J - {\left(-s\right)}^J}{\sin\left(\pi J\right)}
\end{equation}
and again we deform the J-plane integral from the poles at odd values of J to poles $J = j_n\left(t\right)$ again defined through $t_n\left(J\right) = t$. However this time these poles are along the real axis for $J < 1$.
The contribution of the holographic meson trajectory is then
\begin{equation}
\mathcal{A}^{\rm {meson}} = - \frac{\pi}{2} \sum_n \frac{k_{j_n}^2}{2^{j_n}} s^{j_n} \left[-i + \tan \left( \frac{\pi j_n}{2} \right) \right] \frac{d j_n}{dt} \int dz d\bar{z} \frac{e^{-\left(j_n - 4\right) \left(A+\bar{A}\right)-\Phi - \bar{\Phi}}}{\Xi\left(z\right) \Xi\left(\bar{z}\right)} {|\upsilon_1|}^2 {|\upsilon_2|}^2 \psi_n \left(u\left(z\right)\right) {\psi_n \left(u\left(\bar{z}\right)\right)}^*
\end{equation}

From previous work we know that what people call the hard-pomeron corresponds to the $n=0$ contribution from $\mathcal{A}^{gluon}$ and the soft-pomeron is the $n=1$ constribution from $\mathcal{A}^{gluon}$. These trajectories dominate hadron-hadron scattering at very high center-of-mass energies. For lower energies it is necessary to include a meson trajectory which has an intercept lower than 1. We identify this trajectory as the $n=0$ term from $\mathcal{A}^{\rm {meson}}$. Using the optical theorem the total cross-section is given by
\begin{equation}
\sigma\left(h h \rightarrow X\right) = {\rm Im} \, g_0 s^{j_0 - 1} +  {\rm Im} \, g_1 s^{j_1 - 1} +  {\rm Im} \, g_\rho s^{j_\rho - 1}
\end{equation}

\section{Contribution of the meson exchange to $F^p_2$ and $F^p_L$}

The holographic amplitude due to the exchange of spin J fields for $\gamma^* p$
\begin{equation}
\mathcal{A}_J = - k_J \bar{k}_J \int d^5X d^5 \bar{X} \sqrt{-g} \sqrt{-\bar{g}} e^{-\Phi -\bar{\Phi}} F^{(1)}_{- a} \partial_{-}^{J-2} F^{a (3)}_{\, \, \, -} \Pi^{- \cdots -, + \cdots +} \left(X, \bar{X}\right) \Upsilon^{(2)} \bar{\partial}^J_{+} \Upsilon^{(4)}
\end{equation}
Because $A_z = 0$ and $A_\mu = n_\mu f_Q e^{i k \cdot x}$, $F_{z\mu} = n_\mu \dot{f}_Q e^{i k \cdot x}$ and $F_{\mu\nu} = 2 i k_{[\mu}n_{\nu]} f_Q e^{i k \cdot x}$
the expression for the scattering amplitude due to spin J exchange is
\begin{align}
\mathcal{A}_J = - k_J \bar{k}_J s^J \int d^5X d^5\bar{X} \sqrt{-g} \sqrt{-\bar{g}} e^{- \Phi - \bar{\Phi}} e^{-2A} e^{-2J\left(A+\bar{A}\right)} &\left[ \left( \epsilon_{\lambda_1} \cdot \epsilon_{\lambda_3}\right) f_Q^2 \left(1-\delta_{\lambda_1,3}\right) \left(1-\delta_{\lambda_3,3}\right) +  \right. \notag \\
&\left. + \delta_{\lambda_1,3} \delta_{\lambda_3, 3} \frac{\dot{f_Q}^2}{Q^2}\right] |\upsilon_p|^2 e^{-i q_\perp \cdot l_\perp} \Pi_{+ \cdots +, - \cdots -} \left(X, \bar{X}\right)
\end{align}
Making the change of variable $w = x - \bar{x}$  and using the identity
\begin{equation}
\int d^2 l_\perp e^{- i q_\perp \cdot l_\perp}\int \frac{dw^+ dw^-}{2} \Pi_{+\cdots+, - \cdots -} \left(X, \bar{X}\right) = - \frac{i}{\left(-2\right)^J} e^{\left(J-1\right)\left(A+\bar{A}\right)} G_J \left(z, \bar{z}, t\right),
\end{equation}
the scattering amplitude can be rewritten as
\begin{align}
&\mathcal{A}_J = i V \frac{k_J \bar{k}_J}{2^J} s^J \int dz d\bar{z} e^{-\Phi -\bar{\Phi}} e^{-J\left(A+\bar{A}\right)} e^{4\left(A+\bar{A}\right)} e^{-2A} \times  \notag \\
&\left[ \left( \epsilon_{\lambda_1} \cdot \epsilon_{\lambda_3}\right) f_Q^2 \left(1-\delta_{\lambda_1,3}\right) \left(1-\delta_{\lambda_3,3}\right) + \delta_{\lambda_1,3} \delta_{\lambda_3, 3} \frac{\dot{f_Q}^2}{Q^2}\right]  |\upsilon_p|^2 G_J \left(z, \bar{z}, t\right)
\end{align}
 where for the vector meson trajectory
\begin{equation}
G_J \left(z, \bar{z}, t \right)  = \left.\frac{1}{\sqrt{e^{-\frac{10}{3}\Phi} V_f w_s^2 e^A}}\right|_z  \left.\frac{1}{\sqrt{e^{-\frac{10}{3}\Phi} V_f w_s^2 e^A}}\right|_{\bar{z}} \, \sum_n \frac{\psi_n \left(u\left(z\right)\right){\psi_n \left(u\left(\bar{z}\right)\right)}^*}{t_n(J)-t}.
\end{equation}
Now we need to sum over all even spin J fields with $J \geqslant 2$ and then perform the deformation of the complex-J integral.
Defining
\begin{equation}
g_n\left(t\right) = - \frac{\pi}{2} \left[i + \cot\left(\frac{\pi j_n}{2^{j_n}}\right)\right] \frac{k_{j_n} \bar{k}_{j_n}}{2^{j_n}} \frac{d j_n}{dt} \int d\bar{z} \frac{e^{-\bar{\Phi}}e^{-\left(j_n-4\right)\bar{A}}}{\Xi\left(\bar{z}\right)} |\upsilon_p|^2  {\psi_n\left(u\left(z\right)\right)}^*,
\end{equation}
the scattering amplitude is given by
\begin{equation}
\mathcal{A}^{\lambda_1, \lambda_3} \left(s, t\right) = \sum_n g_n\left(t\right) s^{j_n} \int dz \frac{e^{-\Phi} e^{-\left(j_n-2\right)A}}{\Xi\left(z\right)} \left[ \left( \epsilon_{\lambda_1} \cdot \epsilon_{\lambda_3}\right) f_Q^2 \left(1-\delta_{\lambda_1,3}\right) \left(1-\delta_{\lambda_3,3}\right) + \delta_{\lambda_1,3} \delta_{\lambda_3, 3} \frac{\dot{f_Q}^2}{Q^2}\right] \psi_n \left(u\left(z\right)\right)
\end{equation}

In the literature one finds the proton structure functions $F_2$ and $F_L$ defined in terms of the transverse and longitudinal total cross-sections of the process $\gamma^{*} p$
\begin{align}
&F_2\left(x, Q^2\right) = \frac{Q^2}{4 \pi^2 \alpha} \left(\sigma_T + \sigma_L\right) \\
&F_L\left(x, Q^2\right) = \frac{Q^2}{4 \pi^2 \alpha} \sigma_L
\end{align}
Using the optical theorem and the last relations one finds the contribution of the holographic meson trajectory to the structure functions 
\begin{align}
&F_2\left(x, Q^2\right) = \sum_n \frac{{\rm Im} g_n }{4\pi^2 \alpha} Q^{2 j_n} x^{1-j_n} \int dz \frac{e^{-\left(j_n - 2 \right)A - \Phi}}{\Xi\left(z\right)} \left(f_Q^2 +\frac{\partial_z{f_Q}^2}{Q^2} \right) \psi_n\left(u\left(z\right)\right)\\
&F_L\left(x, Q^2\right) = \sum_n \frac{{\rm Im} g_n }{4\pi^2 \alpha} Q^{2 j_n} x^{1-j_n} \int dz \frac{e^{-\left(j_n - 2 \right)A - \Phi}}{\Xi\left(z\right)} \frac{\partial_z{f_Q}^2}{Q^2} \psi_n\left(u\left(z\right)\right)
\end{align}
Changing to the u variable the above expressions become
\begin{align}
&F_2\left(x, Q^2\right) = \sum_n \frac{{\rm Im} g_n }{4\pi^2 \alpha} Q^{2 j_n} x^{1-j_n} \int du \frac{e^{-\left(j_n - 2 \right)A - \Phi}}{G\Xi\left(u\right)} \left(f_Q^2 +\frac{G^2\partial_u{f_Q}^2}{Q^2} \right) \psi_n\left(u\right)\\
&F_L\left(x, Q^2\right) = \sum_n \frac{{\rm Im} g_n }{4\pi^2 \alpha} Q^{2 j_n} x^{1-j_n} \int du \frac{e^{-\left(j_n - 2 \right)A - \Phi}}{G\Xi\left(u\right)} \frac{G^2\partial_u{f_Q}^2}{Q^2} \psi_n\left(u\right)
\end{align}

Another possibility is instead of using the coupling of equation (\ref{eq:open_string_spin_J_coupling}) we use the one of equation (\ref{eq:spin_J_graviton_coupling}) for the upper part of the Witten diagram. The corresponding expressions are
\begin{align}
&F_2\left(x, Q^2\right) = \sum_n \frac{{\rm Im} g_n }{4\pi^2 \alpha} Q^{2 j_n} x^{1-j_n} \int dz \frac{e^{-\left(j_n - 2 \right)A }}{\Xi\left(z\right)}e^{-\frac{10}{3} \Phi} V_f w_s^2 G \left(f_Q^2 +\frac{\partial_z{f_Q}^2}{G^2Q^2} \right) \psi_n\left(u\left(z\right)\right)\\
&F_L\left(x, Q^2\right) = \sum_n \frac{{\rm Im} g_n }{4\pi^2 \alpha} Q^{2 j_n} x^{1-j_n} \int dz \frac{e^{-\left(j_n - 2 \right)A }}{\Xi\left(z\right)}e^{-\frac{10}{3} \Phi} V_f w_s^2 G \frac{\partial_z{f_Q}^2}{G^2Q^2} \psi_n\left(u\left(z\right)\right)
\end{align}
or in terms of the u variable
\begin{align}
&F_2\left(x, Q^2\right) = \sum_n \frac{{\rm Im} g_n }{4\pi^2 \alpha} Q^{2 j_n} x^{1-j_n} \int du e^{-\left(j_n - 2 \right)A }\sqrt{e^{-\frac{10}{3} \Phi} V_f w_s^2} \left(f_Q^2 +\frac{\partial_u{f_Q}^2}{Q^2} \right) \psi_n\left(u\right)\\
&F_L\left(x, Q^2\right) = \sum_n \frac{{\rm Im} g_n }{4\pi^2 \alpha} Q^{2 j_n} x^{1-j_n} \int du e^{-\left(j_n - 2 \right)A }\sqrt{e^{-\frac{10}{3} \Phi} V_f w_s^2} \frac{\partial_u{f_Q}^2}{Q^2} \psi_n\left(u\right)
\end{align}
These last two expressions are the ones I will use from now on to fit the DIS structure functions $F_2$ and $F_L$.

The structure function $F_2$ is related to the total cross-section $\sigma\left(\gamma p \rightarrow X\right)$ through
\begin{equation}
\sigma\left(\gamma p \rightarrow X\right) = 4 \pi^2 \alpha \lim_{Q^2 \rightarrow 0} \frac{F_2\left(x, Q^2\right)}{Q^2}.
\end{equation}
In the limit of $Q^2 \rightarrow 0$
\begin{equation}
\lim_{Q\rightarrow 0} f_Q = 1 , \quad \lim_{Q\rightarrow 0} \frac{\dot{f}_Q}{Q} = 0
\end{equation}
and it follows that
\begin{equation}
\sigma\left(\gamma p \rightarrow X\right) = \sum_n {\rm Im} g_n \,  s^{j_n - 1} \int du e^{-\left(j_n - 2 \right)A }\sqrt{e^{-\frac{10}{3} \Phi} V_f w_s^2} \,  \psi_n\left(u\right).
\end{equation}


\section{$\gamma^{*} \gamma$ processes}

In this section we derive the holographic expressions for $F_2^\gamma$ and $\sigma\left(\gamma \gamma \rightarrow X\right)$ in the context of Holographic QCD in the Veneziano limit. Taking into account only pomeron exchange the expression for $F_2^\gamma$ is
\begin{equation}
F_2^{\gamma \, {\rm pomeron}}\left(x, Q^2\right) = \sum_n \frac{{\rm Im} \, g_n}{4 \pi^2 \alpha} Q^{2 j_n} x^{1-jn} \int du e^{A(\frac{3}{2} - j_n)}  e^{-\frac{7}{3} \Phi}  V_f \left( \lambda, \tau \right) {w_s\left(\lambda \right)}^2  \left(  {f_Q}^2 + \frac{{\partial_u f_Q}^2}{Q^2}  \right) \psi_n \left(j_n, u\right)
\end{equation}
but the definition of the $g_n$s is
\begin{equation}
g_n = - \frac{\pi}{2} \frac{k^2_{j_n}}{2^{j_n}}  \left(i + \cot \frac{\pi j_n}{2} \right) \frac{d j_n}{dt}  \int du e^{A(\frac{3}{2} - j_n)}  e^{-\frac{7}{3} \Phi}  V_f \left( \lambda, \tau \right) {w_s\left(\lambda \right)}^2  \psi_n \left(j_n, u\right)^*
\end{equation}
The result of the contribution the meson exchange to $F_2^\gamma$ is the same as in $F_2^p$
\begin{equation}
F_2^{\gamma \, {\rm meson}}\left(x, Q^2\right) = \sum_n \frac{{\rm Im} g_n }{4\pi^2 \alpha} Q^{2 j_n} x^{1-j_n} \int du e^{-\left(j_n - 2 \right)A }\sqrt{e^{-\frac{10}{3} \Phi} V_f w_s^2} \left(f_Q^2 +\frac{\partial_u{f_Q}^2}{Q^2} \right) \psi_n\left(u\right)
\end{equation}
with the $g_n$s given by
\begin{equation}
g_n = - \frac{\pi}{2} \frac{k^2_{j_n}}{2^{j_n}}  \left(i + \cot \frac{\pi j_n}{2} \right) \frac{d j_n}{dt}  \int du e^{-\left(j_n - 2 \right)A }\sqrt{e^{-\frac{10}{3} \Phi} V_f w_s^2} \psi_n\left(u\right)
\end{equation}

The total cross-section $\sigma\left(\gamma \gamma \rightarrow X\right)$ is related to the structure function $F^\gamma_2$ through
\begin{equation}
\sigma\left(\gamma \gamma \rightarrow X\right) =  4 \pi^2 \alpha \lim_{Q^2 \rightarrow 0} \frac{F_2^\gamma}{Q^2}
\end{equation}
and hence the contributions of pomeron exchange and meson exchange to this observable are, respectively
\begin{align}
&\sigma^{\rm pomeron}\left(\gamma \gamma \rightarrow X\right) =  \sum_n {\rm Im} \, g_n s^{jn-1} \int du e^{A(\frac{3}{2} - j_n)}  e^{-\frac{7}{3} \Phi}  V_f \left( \lambda, \tau \right) {w_s\left(\lambda \right)}^2  \psi_n \left(j_n, u\right) \\ 
&\sigma^{\rm meson}\left(\gamma \gamma \rightarrow X\right) = \sum_n {\rm Im} g_n s^{j_n - 1} \int du e^{-\left(j_n - 2 \right)A }\sqrt{e^{-\frac{10}{3} \Phi} V_f w_s^2}  \psi_n\left(u\right)
\end{align}
Of course $\sigma\left(\gamma \gamma \rightarrow X\right) = \sigma^{\rm pomeron}\left(\gamma \gamma \rightarrow X\right) + \sigma^{\rm meson}\left(\gamma \gamma \rightarrow X\right)$.




\appendix

\section{Solving the Equatios of Motion}
\label{appendix:eom_sol}

In this appendix we give details on how the equations of motion are solved. It turns out to be convenient to write the resulting equations in terms of $A$ instead of the radial coordinate $z$. To do this one introduces the variable
\begin{equation}
q\left(A\right) = \frac{dz}{dA} e^{A}
\label{eq: q definition}
\end{equation}
From now on all the background fields will be functions of $A$ and the differential equations we present are all with respect to A being the independent variable.

\subsubsection*{Yang-Mills Equations of Motion}

If we make $x = 0$ we get the action of the IHQCD model and hence we are dealing with a pure Yang-Mills theory in the large $N_c$ limit. The equations of motion are then
\begin{align}
&\frac{dq}{dA} = \frac{1}{3} \left( 12 q - q^3 V_g \left(\Phi\right) \right) \, , \label{eq:q_YM_eom} \\
&\frac{d\Phi}{dA} = - \frac{\sqrt{3}}{2} \sqrt{12 - q^2 V_g\left(\Phi\right)} \label{eq:Phi_YM_eom}
\end{align}
We will solve this coupled differential equations from the IR to the UV. We set $A_{IR} = -150$ and $A_{UVYM} = 50$ as the lower and upper bound on A. In the IR the warp factor $A$ and the dilaton $\Phi$ have the asymptotic forms
\begin{align}
&A_{IR} \left(z\right) = \frac{23}{24} - \frac{173}{3456 z^2} - z^2 + 0.25 \log(6 z^2) - \frac{\log(Vg_{IR})}{2} \\
&\Phi_{IR} \left(z \right) = - \frac{23}{16} - \frac{151}{2304 z^2} + \frac{3}{2} z^2 - \log(\frac{sc}{\lambda_0})
\label{eq: YM fields IR asymptotics}
\end{align}
This allows us to determine $z_{IRYM}$ such that $A_{IR}\left(z_{IRYM}\right) = -150$ and use its numerical value to compute $\Phi\left(A_{IR}\right) = \Phi_{IR} \left( z_{IRYM}\right)$ and $q\left(A_{IR}\right) = e^{A_{IR}} / \frac{dA_{IR}}{d z} $ and hence defining boundary conditions to solve equations~\ref{eq:q_YM_eom} and~\ref{eq:Phi_YM_eom}. Note that $sc$ and $V_{gIR}$ determine uniquely both the initial conditions as well the evolution of the background fields in pure YM.

After we solve the YM equations we can determine $z\left(A\right)$ by using equation (\ref{eq: q definition}). We solve this numerically by taking $z(A_{IRYM})$ = $z_{IRYM}$ as initial condition and at the end we perform the shift $z\left(A\right) \to z\left(A\right) - z\left(A_{UVYM} \right) $ in order to have the UV singularity at $z = 0$. For the spectrum we will consider only  $A$ between $-10$ and $10$.

We finish this section by specifying which algorithms we have used to find the numerical YM background. To compute $z_{IRYM}$ we have used the root finding Van Wijngaarden-Dekker-Brent method described in~\cite{10.5555/1403886}. The differential equations~\ref{eq:q_YM_eom} and~\ref{eq:Phi_YM_eom} were solved using $\text{integrate\_const}$ with a Runge-Kutta-Dormand-Prince stepper from the Boost $C\texttt{++}$ library.

\subsubsection*{Solving HVQCD equations of motion}

The solution of HVQCD equations is done in four stages. In the first two we are deep in the IR and the tachyon is decoupled from the other background fields and $q\left(A\right)$ and $\Phi\left(A\right)$ obey equations (\ref{eq: YM equations}). In the third stage the tachyon will couple to the other background fields until it reaches the UV where it decouples again and we are in the fourth and last stage of the solution of the EOMs.

As in the YM case we start by first specifying the IR boundary conditions. We first define $A_{IR} = -150$, $A_{UVYM} = 50$, $A_{UVc} = 100$ and $A_{UVf} = 1000$.
Again  we compute $z_{IRYM}$ as in the YM case and we use it to compute $q\left(A_{IR}\right)$ and $\Phi\left(A_{IR}\right)$ as in YM and $\tau\left(A_{IR}\right)$ using the tachyon IR asymptotics
\begin{equation}
\tau \sim \tau_0 \, z^{\tau_c}, \quad \tau_c = \frac{4}{3} \frac{\left(\frac{3}{2} - x \frac{W_0}{8}\right) kIR a_2}{VgIR (a_2 - a_1)}
\end{equation}
where $\tau_0$ is a paremter that is gonna be fitted by the spectrum. This is the last parameter to be introduced.
The constants $\tau_{cut} = 1000$ and $V_{f cut} = 10^{-8}$ are defined and they signal the end of the first and second stages of the construction of the numerical background, respectively.

We start the construction of the background by computing the YM profile of $q\left(A\right)$ and $\Phi\left(A\right)$ from $A_{IR}$ to $A_{UVYM}$. If $\tau_{IR} > \tau_{cut}$ the tachyon profile will be given by the solution of the differential equation
\begin{equation}
	\frac{d\tau}{dA} = - 4 \frac{q^2 V_{f0}\left(\Phi\right) \tau}{8 V_{f0}\left( \Phi\right) k\left(\Phi\right) + 2 k\left(\Phi\right) \frac{d V_{f0}}{d\Phi} \frac{d\Phi}{dA} + V_{f0} \left(\Phi\right) \frac{dk}{d\Phi} \frac{d\Phi}{dA} }
\end{equation}
until $\tau < \tau_{cut}$, marking the end of the first stage. The value of $A$ such that this condition is met is called $A_{UV1}$ The values of $q$ and $\Phi$ used are the ones given by YM.

In the second stage we solve the tachyon differential equation
\begin{align}
&\frac{d^2\tau}{d A^2} = - 2 \frac{q^2 \tau}{k\left(\Phi\right)} - 4 \frac{d \tau}{dA} + \frac{d \log q}{dA} \frac{d\tau}{dA} - 2 \tau {\left(\frac{d\tau}{dA}\right)}^2 - 4 k\left(\Phi\right) \frac{{\left( \frac{d\tau}{dA}\right)}^3}{q^2} - \\ \notag
& - \frac{d \log V_{f0}}{d\Phi} \frac{d\tau}{dA} \frac{d\Phi}{dA} - \frac{d \log k}{d\Phi} \frac{d\tau}{dA} \frac{d\Phi}{dA} - k\left(\Phi\right) \frac{d \log V_{f0}}{d\Phi} {\left(\frac{d\tau}{dA}\right)}^3 \frac{\frac{d\Phi}{dA}}{q^2} - \frac{d k}{d\Phi} \frac{d\Phi}{dA} \frac{{\left(\frac{d\tau}{dA}\right)}^3}{2 q^2}
\label{eq: tau YM 2}
\end{align}
from the value of $A_{UV1}$ to $A_{UVYM}$. Using the profiles of $q$, $\Phi$ and $\tau$ we compute $A_{UV2}$ such that $A_{UV1} < A_{UV2} < A_{UVYM}$ and $V_{f0}\left(\Phi\right) e^{-\tau^2} = V_{fcut} V_g \left(\Phi\right)$ are satisfied. This condition marks when the tachyon starts to couple with $q$ and $\Phi$ from the IR to the UV.\footnote{Matti's code rejects such backgrounds while my code just takes them as valid backgrounds. This, I think, explains why Matti could not get the same results for the spectrum as me in that fit. This also means that in the numerical background I have obtained the tachyon never couples to $q$ and $\Phi$. }

In the stage where the tachyon is coupled the backgroud fields' dynamics obeys
\begin{align}
& \frac{dq}{dA} = \frac{4}{9} q {\left(\frac{d\Phi}{dA}\right)}^2 + q V_f \left(\Phi, \tau\right) k\left(\Phi\right) \frac{{\left(\frac{d\tau}{dA}\right)}^2}{6\sqrt{1+ k\left(\Phi\right) {\left(\frac{\frac{d\tau}{dA}}{q}\right)}^2}} \\
& \frac{d^2 \Phi}{d^2A} = - \frac{3}{8} q^2 \frac{d V_g}{d\Phi} + \frac{9}{\frac{d\Phi}{dA}} + \frac{3}{4} \frac{q^2}{\frac{d\Phi}{dA}} \left( \frac{V_f \left(\Phi, \tau\right)}{\sqrt{1+ k \frac{{\left(\frac{d\tau}{dA}\right)}^2}{q^2}}} - V_g \left(\Phi\right) \right) - \\ \notag
& - 5 \frac{d\Phi}{d A} + V_f \left(\Phi, \tau\right) k \left(\Phi\right) \frac{{\left(\frac{d\tau}{dA}\right)}^2 \frac{d\Phi}{dA}}{6 \sqrt{1+ k\left(\Phi\right) {\left(\frac{\frac{d\tau}{dA}}{q}\right)}^2 }} + \frac{4}{9} {\left(\frac{d\Phi}{dA}\right)}^3 + \\ \notag
& + \frac{3}{8} q^2 \frac{\frac{d V_f}{d\Phi}}{\sqrt{1+ k\left(\Phi\right) {\left(\frac{\frac{d\tau}{dA}}{q}\right)}^2}} + \frac{3}{8} k\left(\Phi\right) {\left( \frac{d\tau}{dA}\right)}^2 \frac{\frac{dV_f}{d\Phi}}{\sqrt{1+ k\left(\Phi\right) {\left(\frac{\frac{d\tau}{dA}}{q}\right)}^2}} + \\ \notag
& + \frac{3}{16} V_f\left(\Phi, \tau\right) {\left( \frac{d\tau}{dA}\right)}^2 \frac{\frac{dk}{d \Phi}}{\sqrt{1+ k\left(\Phi\right) {\left(\frac{\frac{d\tau}{dA}}{q}\right)}^2}} \\ \notag
& \frac{d^2 \tau}{d A^2} = -2 q^2 \frac{\tau}{k\left(\Phi\right)} - 4 \frac{d\tau}{d A} -2 \tau {\left(\frac{d\tau}{dA}\right)}^2 - 4\frac{k\left(\Phi\right)}{q^2} {\left(\frac{d\tau}{dA}\right)}^3 + k\left(\Phi\right) V_f \left(\Phi, \tau \right) \frac{{\left(\frac{d\tau}{dA}\right)}^3}{6 \sqrt{1+ k\left(\Phi\right) {\left(\frac{\frac{d\tau}{dA}}{q}\right)}^2}} - \\
& - \frac{d \log(V_{f0})}{dA} \frac{d\Phi}{dA} \frac{d\tau}{dA} - k \left(\Phi\right) \frac{d\log(V_{f0}}{dA} \frac{d\Phi}{dA}\frac{{\left(\frac{d\tau}{d A}\right)}^3}{q^2} + \frac{4}{9} {\left(\frac{d\Phi}{dA}\right)}^2 \frac{d\tau}{dA} - \frac{d \log(k)}{d\Phi} \frac{d\Phi}{dA} \frac{d\tau}{dA} - \\ \notag
& - \frac{1}{2} \frac{d k}{d\Phi} \frac{d\Phi}{dA} \frac{{\left( \frac{d\tau}{dA}\right)}^3}{q^2}
\end{align}
This is a stiff system of differential equations and for this reason we had used the function integrate\_adaptive with the rosenbrock4 stepper from Boost. This system is solved from $A_{UV2}$ to $A_{UCc}$. Because $A_{UVc} = 50$ we believe that at this point we are in the UV and here the tachyon decouples again from $q$ and $\Phi$.

The UV equations of motion are
\begin{align}
&\frac{dq}{dA} =  \frac{4}{9} q {\left(\frac{\frac{d\lambda}{dA}}{\lambda}\right)}^2 \\
&\frac{d^2\lambda}{dA^2} = -\frac{3}{8}  {\left(q \lambda \right)}^2 \frac{d V_g}{d\lambda} + 9 \frac{\lambda^2}{\frac{d\lambda}{dA}} + \frac{3}{4} \frac{{\left(q \lambda \right)}^2}{\frac{d\lambda}{dA}} \left( V_f\left(\lambda, 0\right) - V_g \left(\lambda\right) \right) - 5 \frac{d\lambda}{dA}  + \\
& + \frac{d \log q}{dA} \frac{d\lambda}{d A} + \frac{{\left( \frac{d\lambda}{dA}\right)}^2}{\lambda} + \frac{3}{8} {\left( q \lambda \right)}^2 \frac{d V_f}{d\lambda} \left(\lambda, 0 \right) \\ \notag
& \frac{d^2 \tau_n}{dA^2} = 3 \tau_n - 2 \frac{q^2 \tau_n}{k\left(\lambda\right)} - \tau_n \frac{d \log q}{dA} + \tau_n \frac{d \log V_{f0}}{d\lambda} \frac{d\lambda}{dA} - 2 \frac{d\tau_n}{d A} + \frac{d\log q}{d A} \frac{d \tau_n}{dA} - \\
& - \frac{d \log V_{f0}}{d\lambda} \frac{d\lambda}{dA} \frac{d\tau_n}{dA} + \tau_n \frac{d\lambda}{dA} \frac{d \log k}{d\lambda} - \frac{d\lambda}{dA} \frac{d\tau_n}{dA} \frac{d \log k}{d\lambda}
\end{align}
where $\tau_n = e^A \tau$. This system is solved from $A_{UVc}$ to $A_{UVf}$. With the profile of $\tau_n$ and $\lambda$ in this region we can compute the quark mass $m_q$ through the formula
\begin{align}
&m_q = \frac{\lambda \left(A_{UVf} - 10\right) \tau_n\left(A_{UVf}\right)e^{- \log l_{UV} - \tau_{corr}\left(\lambda\left(A_{UVf}\right)\right)} - \lambda \left(A_{UVf} \right) \tau_n\left(A_{UVf} - 10\right)e^{- \log l_{UV} - \tau_{corr}\left(\lambda\left(A_{UVf} - 10\right)\right)} }{\lambda\left(A_{UVf} - 10\right) - \lambda\left(A_{UVf}\right)}, \\
& \tau_{corr} = \frac{\left(-88 + 16 x + 27 sc k_{U1}\right) \log\left( \frac{24 \pi^2}{\left(11 - 2 x\right) \lambda}\right)}{12 x - 66}
\end{align}

As in the YM case we can determine $z\left(A\right)$ by using equation (\ref{eq: q definition}) and perform the shift $z\left(A\right) \to z\left(A\right) - z\left(A_{UVYM} \right) $ in order to have the UV singularity at $z = 0$. For the spectrum we will consider only  $A$ values that satisfy $z\left(A\right) > 10^{-6}$ and $\Phi\left(A\right) < 120$.


\section{EOM and gravitational coupling of the $U(1)$ gauge field}
\label{appendix:b}

In this appendix we start deriving the action of the vector meson sector since the non-normalisable solution of the corresponding equations of motion is dual to the external photon in the boundary. Then, by linearising the action we find the coupling of the $U(1)$ gauge field to the graviton of the bulk theory. All of this is made in the Einstein frame. Since we will be dealing later with calculations on the string frame we will explain how to translate these results to the string frame. Finally we generalise the coupling to the graviton to any spin J field in the graviton's Regge trajectory.

\subsection*{The action of the vector $U(1)$ gauge field}
As mentioned previously, for the QCD vacuum we set $A^R_a = 0 = A^L_a$.
When we turn on the gauge fields, the flavour action becomes
\begin{align}
&S_f = - \frac{M^3 N_c}{2}  \int d^5 X V_f \left( \lambda, \tau \right) \bold{Tr} \left[ \sqrt{\rm{det} \left(g_{\rm{eff.} AB} + T^{L}_{AB}\right)} + \sqrt{\rm{det} \left(g_{\rm{eff.} AB} + T^{R}_{AB}\right)} \right] \, , \\
&g_{\rm{eff.} MN} = g_{MN} +  k\left(\lambda\right) \partial_M \tau \partial_N \tau \, , \\
&T^{L}_{AB} = w\left(\lambda\right) F^{L}_{AB} + k\left(\lambda\right)D_{(A}\tau D_{B)}\tau -   k\left(\lambda\right) \partial_A \tau \partial_B \tau \, , \\
&T^{R}_{AB} = w\left(\lambda\right) F^{R}_{AB} + k\left(\lambda\right)D_{(A}\tau D_{B)}\tau - k\left(\lambda\right) \partial_A \tau \partial_B \tau \, .
\end{align}
Here we remember that $\bold{Tr}$ is a trace over the gauge indices while the determinant is computed with respect to the space-time indices. Writting the determinats inside the square roots as 
\begin{align}
\rm{det}\left(g_{\rm{eff.} AB} + T^{L/R}_{AB} \right) = \rm{det} \left(g_{\rm{eff.}} \right)  \rm{det} \left( 1 + g_{\rm{eff.}}^{-1} T^{L/R} \right) \, ,
\end{align}
defining $X^{L/R} = g_{eff}^{-1} T^{L/R}$ and using the identity
\begin{align}
\ln \text{det} \left(1+X^{L/R}\right) = \text{tr} \ln\left(1 + X^{L/R}\right) = \text{tr} X^{L/R} - \frac{1}{2} \text{tr} {\left(X^{L/R}\right)}^2 + \cdots \, ,
\end{align}
it follows that
\begin{align}
&\sqrt{\text{det}  \left(1+X^{L/R}\right)} = 1 + \frac{1}{2} \text{tr} \, X^{L/R} - \frac{1}{4} \text{tr} \, {\left(X^{L/R}\right)}^2 + \cdots \, , \\
& \text{tr} \, X^{L/R} = \sum_{A,B}  \left( g^{-1}_{\rm{eff.}} \right)_{AB} T^{L/R}_{AB} \, , \\
& \text{tr} \, {\left(X^{L/R}\right)}^2 = \sum_{A,B,C,D}  \left( g^{-1}_{\rm{eff.}} \right)_{AC} T^{L/R}_{CB} \left( g^{-1}_{\rm{eff.}} \right)_{BD} T^{L/R}_{DA} \, .
\end{align}
The expressions for the matrix elements of  $g^{-1}_{\rm{eff.}}$ and $T^{L/R}$ are
\begin{align}
&\left(g^{-1}_{\rm{eff.}}\right)_{zz} = \frac{e^{-2A}}{G^2} \quad , \quad  \left(g^{-1}_{\rm{eff.}} \right)_{\mu \nu} = \eta^{\mu \nu} e^{-2A} \, , \\
&\left(g^{-1}_{eff}\right)_{z\mu} = \left(g^{-1}_{eff}\right)_{\mu z} = 0, \\
& T^{L/R}_{zz} = 0 \, , \\
& T^{L/R}_{z \mu} = - T^{L/R}_{\mu z} =  w\left(\lambda \right) \left(  \pm \partial_z A_\mu + \partial_z V_\mu \right) \, , \\
& T^{L/R}_{\mu \nu} = 4 k\left(\lambda\right) \tau^2 A_\mu A_\nu + w\left(\lambda \right) \left( \pm A_{\mu \nu} + V_{\mu \nu} \right) \, .
\end{align}
where the vector and axial gauge fields $V_M$ and $A_M$ are linear combinations of the left and right gauge fields
\begin{equation}
V_M = \frac{A^L_M + A^R_M}{2}\, , \, A_M = \frac{A^L_M - A^R_M}{2}
\end{equation}
in the gauge $V_z = 0 = A_z$, $V_{\mu \nu} = \partial_\mu V_\nu  - \partial_\nu V_\mu$ and $A_{\mu \nu} = \partial_\mu A_\nu  - \partial_\nu A_\mu$.

%Using the expressions for $g^{-1}_{eff}$ and $T^{L/R}$ we get
%\begin{align}
%\text{tr} C^{L/R} = 4 e^{-2A} k \left( \lambda \right) \tau^2 A_\mu A^\mu
%\end{align}
%and
%\begin{align}
%\text{tr} \left( {C^{L/R}}^2\right) = - 2 \left( g^{-1}_{eff} \right)_{zz} \left( g^{-1}_{eff} \right)_{\mu \nu} T^{L/R}_{z \mu} T^{L/R}_{z \nu} +  \left( g^{-1}_{eff} \right)_{\mu \nu} \left( g^{-1}_{eff} \right)_{\alpha \beta} T^{L/R}_{\nu \alpha} T^{L/R}_{\beta \mu} 
%\end{align}
%The expressions for each term in the last equation are
%\begin{align}
%&\sum_{\mu, \nu} \left( g^{-1}_{eff} \right)_{zz} \left( g^{-1}_{eff} \right)_{\mu \nu} T^{L/R}_{z \mu} T^{L/R}_{z \nu} = \frac{{w\left(\lambda\right)}^2}{G^2} e^{-4 A} \left( \partial_z A_\mu \partial_z A^\mu \pm 2 \partial_z A_\mu \partial_z V^\mu + \partial_z V_\mu \partial_z V^\mu \right) , \\
%& \sum_{\alpha, \beta, \mu, \nu} \left( g^{-1}_{eff} \right)_{\mu \nu} \left( g^{-1}_{eff} \right)_{\alpha \beta} T^{L/R}_{\nu \alpha} T^{L/R}_{\beta \mu} = - e^{-4A} {w\left(\lambda\right)}^2 \left( V_{\mu \nu} V^{\mu \nu} \mp 2 V_{\mu \nu} A^{\mu \nu} + A_{\mu \nu} A^{\mu \nu} \right)
%\end{align}
From these identities and after some mechanical calculations one gets
\begin{align}
&S_f = - \frac{1}{2} M^3 N_c \int d^5 X V_f \left(\lambda, \tau \right) \sqrt{- g_{eff}} \bold{Tr} \left[ 2 + 4 e^{-2A} k\left( \lambda \right) \tau^2 A_\mu A^\mu + \right. \notag \\
& \left. + \frac{{w\left(\lambda\right)}^2 e^{-4A}}{G^2} \left( \partial_z A_\mu \partial_z A^\mu +  \partial_z V_\mu \partial_z V^\mu \right) + \frac{1}{2} e^{-4A} {w\left(\lambda\right)}^2 \left( V_{\mu \nu}V^{\mu \nu} + A_{\mu \nu} A^{\mu \nu} \right) \right] \, .
\end{align}
The first term is the background term, while the the two other terms can be packed into the actions
\begin{align}
\label{eq:VM_action}
&S_V = - \frac{1}{2} M^3 N_c \bold{Tr} \int d^5 XV_{f} \left( \lambda, \tau \right) {w\left(\lambda\right)}^2 G^{-1} e^{A} \left( \partial_z V_\mu \partial_z V^\mu + \frac{1}{2} G^2 V_{\mu \nu} V^{\mu \nu} \right), \\
& S_A = - \frac{1}{2} M^3 N_c \bold{Tr} \int d^5 X V_{f} \left( \lambda, \tau \right) {w\left(\lambda\right)}^2 G^{-1} e^{A} \left( \partial_z A_\mu \partial_z A^\mu + \right. \notag \\
& \left. +  \frac{1}{2} G^2 A_{\mu \nu} A^{\mu \nu} + 4 e^{2A} \frac{k\left(\lambda\right) G^2 \tau^2}{{w\left(\lambda\right)}^2} A_\mu A^\mu \right)
\end{align}
which are the actions for the Non-singlet Vector and Axial-Vector mesons presented in~\cite{Arean:2013tja}.
The equation (\ref{eq:VM_action}) can still be rewritten as
\begin{equation}
S = - \frac{1}{4} M^3 N_c N_f \int d^5 X \sqrt{-g} V_f w^2 G F_{AB}F^{AB},
\label{eq:VM_new_action}
\end{equation}
where here the notation $F_{AB}F^{AB}$ means
\begin{equation}
F_{AB}F^{AB} =  \sum_{A,B,C,D}F_{AB} \left(g^{-1}_{\rm{eff.}}\right)_{AC} \left(g^{-1}_{\rm{eff.}}\right)_{BD} F_{CD} %=  \frac{2}{G^2}e^{-4A}\left( \frac{1}{2} G^2 V_{\mu \nu} V^{\mu \nu} + \partial_z V_\mu \partial_z V^\mu \right)
\end{equation}

%\subsection*{Equation of motion of the EOM in terms of u}
%We wont derive the equation of motion of the non-normalizable mode of the $U\left(1\right)$ gauge field because it is done so in~\cite{Arean:2013tja}. Introducing the variable
%\begin{equation}
%\frac{du}{dz} = G
%\end{equation}
%we get
%\begin{equation}
%\frac{d^2 f_Q}{d u^2} - Q^2 f_Q + \frac{d f_Q}{du} \left( \frac{dA}{du} + 2 \frac{w'\left(\Phi\right)}{w\left(\Phi\right)} \frac{d\Phi}{du} + \frac{\frac{\partial V_f}{\partial \tau}}{V_f} \frac{d\tau}{du} +  \frac{\frac{\partial V_f}{\partial \Phi}}{V_f} \frac{d\Phi}{du} \right) = 0.
%\end{equation}
%Using the fact that
%\begin{equation}
%frac{dA}{du} = \frac{e^A}{G q} , \quad  \frac{d\Phi}{du} = \frac{e^A}{G q} \frac{d\Phi}{dA} , \quad \frac{d\tau}{du} = \frac{e^A}{G q} \frac{d\tau}{dA}
%\end{equation}
%we finally get
%\begin{equation}
%\frac{d^2 f_Q}{d u^2} - Q^2 f_Q + \frac{d f_Q}{du} \frac{e^A}{G q} \left(1 + 2 \frac{w'\left(\Phi\right)}{w\left(\Phi\right)} \frac{d\Phi}{dA} + \frac{\frac{\partial V_f}{\partial \tau}}{V_f} \frac{d\tau}{dA} +  \frac{\frac{\partial V_f}{\partial \Phi}}{V_f} \frac{d\Phi}{dA} \right) = 0.
%\end{equation}
%where
%\begin{equation}
 %\frac{\frac{\partial V_f}{\partial \Phi}}{V_f} = \frac{V_{f0}'\left(\Phi\right)}{V_{f0}}, \quad   \frac{\frac{\partial V_f}{\partial \tau}}{V_f} = - \frac{2\tau\left( -a_1 + a_2 + a_1 a_2 \tau^2\right)}{1 +a_1 \tau^2}
%\end{equation}

\subsection*{From Einstein frame to string frame}
To compute the action of the $U(1)$ gauge field in the Einstein frame we approximated the square roots of the determinants present in $S_f$ as
\begin{align}
\sqrt{-\det \left( g_{\rm{eff.}} + T^{L/R} \right)} = \sqrt{-g_{\rm{eff.}}} \left[ 1 + \frac{1}{2} \text{tr} X - \frac{1}{4} \text{tr} X^2 + \frac{1}{8} {\left( \text{tr} X \right)}^2 + \dots \right]
\end{align}
with $X = g^{-1}_{\rm{eff.}} T^{L/R}$ for some tensor $T^{L/R}$.
The string frame warp factor $A_s$ is related to the Einstein frame warp factor $A$ through $A_s = A + \frac{2}{3} \Phi$. From this it follows that $\sqrt{-g_{\rm{eff.}}} = e^{- 10 \Phi / 3} \sqrt{-g_{\rm{eff.} s}}$ and the matrix $X$ in the previous equation can be written as $X = g_s^{-1} T^{L/R}_s$ with $T^{L/R}_s = e^{4 \Phi / 3} T^{L/R}_E$. This implies that
\begin{align}
\sqrt{-\det \left( g_{\rm{eff.}} + T^{L/R} \right)} = e^{- 10 \Phi / 3} \sqrt{-\det \left( g_{\rm{eff.} s} + T^{L/R}_s \right)} \, ,
\end{align}
i.e.  $S_f$ in the string frame is obtained by simply substituting the metric by the string frame metric $g_s$, substituting $k\left(\lambda\right)$ and $w\left(\lambda\right)$ by $k_s \left(\lambda\right) = e^{4 \Phi / 3} k\left(\lambda\right) $ and $w_s \left(\lambda\right) = e^{4 \Phi / 3} w\left(\lambda\right) $ and multiply $V_f$ by the factor $e^{-10 \Phi / 3}$. The derivation of $S_V$ in the string frame will be formally the same and hence the results equal to the Einstein frame ones but with $w_s$, $k_s$ and $e^{-10 \Phi / 3} V_f$ in place of $w$, $k$ and $V_f$ respectively. The action (\ref{eq:VM_new_action}) in the string frame takes the form
\begin{align}
&S = - \frac{1}{4} M^3 N_c N_f \int d^5 X \sqrt{-g_s} e^{-\frac{10}{3} \Phi} V_f w_s^2 G F_{AB}F^{AB}  \, , \\
& F_{AB}F^{AB} =  \sum_{A,B,C,D}F_{AB} \left(g^{-1}_{\rm{eff.} s}\right)_{AC} \left(g^{-1}_{\rm{eff.} s}\right)_{BD} F_{CD} \, .
\label{eq:VM_new_action_string_frame}
\end{align}


\subsection*{Coupling between the graviton and the $U(1$ vector gauge field}
We will now determine the gravitational coupling between the $U(1)$ gauge field and the spin J fields in the graviton's Regge trajectory. We first compute the coupling with the graviton and generalise to any even spin J field. All of this is done in the Einstein frame. To find the coupling in the string frame we just substitute the functions $w$, $k$ and $V_f$ by $w_s$, $k_s$ and $e^{-10 \Phi / 3} V_f$ respectively, as discussed previously. 

Again, we start by writing the square roots of the determinants as
\begin{align}
	\sqrt{-\det g_{\rm{eff.}}} \left[ 1 + \frac{1}{2} \text{tr} \left( g^{-1}_{\rm{eff.}} T^{L/R}\right) - \frac{1}{4} \text{tr} \left( g^{-1}_{\rm{eff.}} T^{L/R} g^{-1}_{\rm{eff.}} T^{L/R} \right) + \dots \right]
\end{align}
The coupling with the graviton is found by linearising the above equation around the background metric, i.e. $g_{AB} = \bar{g}_{AB} + h_{AB}$. 
To study the graviton Regge trajectory in our background we need to decompose the metric in $SO(1,3)$ irreducible representations. 
We will be only interested in the graviton TT components $h_{\alpha \beta}$, satisfying $\partial^\alpha h_{\alpha \beta} = 0$ and $h^\alpha_\alpha = 0$, and also set $h_{z \alpha} = h_{\alpha z} = h_{zz} = 0$.

 For our purposes we can ignore the perturbation of $\sqrt{-\det g_{\rm{eff.}}}$ because it will involve a term proportional to $h = h^A_A = 0$. We will also neglect the terms $A_{\mu}$ since we are only interested in the coupling of $V_\mu$. We wish then to compute $\delta \text{tr} \left( g^{-1}_{\rm{eff.}} T^{L/R}\right)$ and $\delta \text{tr} \left( g^{-1}_{\rm{eff.}} T^{L/R} g^{-1}_{\rm{eff.}} T^{L/R} \right)$. Using the fact that $\delta g^{AB}_{\rm{eff.}} = - g_{\rm{eff.}}^{AM} g_{\rm{eff.}}^{BN} h_{MN}$ one can show
\begin{align}
&\delta \text{tr} \left( g^{-1}_{\rm{eff.}} T^{L/R}\right) = -  g_{\rm{eff.}}^{AM} g_{\rm{eff.}}^{BN} h_{MN} T^{L/R}_{AB}\, , \\
&\delta \text{tr} \left( g^{-1}_{\rm{eff.}} T^{L/R} g^{-1}_{\rm{eff.}} T^{L/R} \right) = - 2 g^{AM}_{\rm{eff.}} g^{BN}_{\rm{eff.}} g^{CD}_{\rm{eff.}} T^{L/R}_{BC} T^{L/R}_{DA} h_{MN}\, .
\end{align}
Using the expressions for the matrix elements of $g^{-1}_{\rm{eff.}}$ and $T^{L/R}$ we get
\begin{align}
&\delta \text{tr} \left( g^{-1}_{\rm{eff.}} T^{L/R}\right) = 0 \, , \\
& \delta \text{tr} \left( g^{-1}_{\rm{eff.}} T^{L/R} g^{-1}_{\rm{eff.}} T^{L/R} \right)  = 2 e^{-6A} {w\left( \lambda \right)}^2 h^{\mu \nu} \left( V_{\mu \sigma} \eta^{\sigma \rho} V_{\nu \rho} + \frac{1}{G^2} \partial_z V_\mu \partial_z V_\nu \right) \, .
\end{align}
Hence the coupling between the vector $U(1)$ gauge field and the graviton is given by
\begin{align}
\frac{M^3 N_c N_f}{2}  \int d^5 X \sqrt{-\det g_{\rm{eff.}}} V_{f} \left(\lambda, \tau \right) e^{-6 A} {w\left(\lambda\right)}^2 h^{\mu \nu} \left( V_{\mu \sigma} \eta^{\sigma \rho} V_{\nu \rho} + \frac{1}{G^2} \partial_z V_\mu \partial_z V_\nu \right)
\end{align}
%From this result we wish to generalise to any spin J field. The coupling came from the term $\delta \text{tr} \left( g^{-1}_{eff} T^{L/R} g^{-1}_{eff} T^{L/R} \right)$ wich can be written as
%\begin{align}
%2 g^{\alpha \mu} g^{\beta \nu} g_{eff}^{AB} T^{L/R}_{A \alpha} T^{L/R}_{B \beta} h_{\mu \nu}
%\end{align}
%Here $T^{L/R}_{AB} = w\left(\lambda\right) F^V_{AB}, \, F^V_{AB} = \nabla_A V_B - \nabla_B V_A$ since $A_M = 0$ for all space-time indices. The left and right contributions for the coupling are equal to
%\begin{align}
%2 g^{\alpha \mu} g^{\beta \nu} g_{eff}^{AB} {w\left(\lambda\right)}^2 F^{V}_{A \alpha} F^{V}_{B \beta} h_{\mu \nu}
%\end{align}
%and the coupling is given by
or simply
\begin{align}
\frac{M^3 N_c N_f}{2} \int d^5 X \sqrt{-g} G V_{f} \left(\lambda, \tau \right)  {w\left(\lambda\right)}^2 g^{C M} g^{D N} g_{eff}^{AB} F^V_{A C} F^V_{B D} h_{M N} \, .
\end{align}
In the string frame this coupling takes the form
\begin{align}
\frac{M^3 N_c N_f}{2} \int d^5 X \sqrt{-g_{s}} G e^{-\frac{10}{3}\Phi} V_{f} \left(\lambda, \tau \right)  {w_s\left(\lambda\right)}^2 g^{C M}_s g^{D N}_s g_{eff. s}^{AB} F^V_{A C} F^V_{B D} h_{M N}^s \, .
\end{align}
We now generalise this coupling to the case of an interaction between the gauge field and a symmetric, transverse and traceless spin J field, $h_{A_1 \cdots A_J}$. The pomeron trajectory includes such higher spin fields of even J. Again there are several possibilities, but we shall focus on the simplest extension of the graviton coupling considered above.
For a spin J field we propose the coupling
\begin{align}
k_J \int d^5X \sqrt{-g_s} G e^{-\frac{10}{3} \Phi} V_f \left( \lambda, \tau \right) {w_s\left(\lambda \right)}^2  g_{eff. s}^{AB} F^V_{A C} \nabla_{A_1} \dots \nabla_{A_{J-2}} F^V_{B D} h^{C D A_1 \dots A_{J-2}}
\label{eq:spin_J_graviton_coupling}
\end{align}
We note that the transverse condition of the spin J field $h_{A_1 \cdots A_J}$ guarantees that this term is unique up to dilaton and tachyon derivatives.

So far we have discussed the coupling of the $U(1)$ gauge field with the spin J fields of the graviton's Regge trajectory. However, we will also consider exchange of spin J fields dual to the spin J twist two operators made of quark bilinears. A strategy could be to compute the coupling between the $U(1)$ gauge field and the $\rho$ meson and generalise the result to higher spin fields. This coupling can be obtained by analysing the cubic couplings of the DBI action which are not present in this model. Another approach is to find the coupling between the $U(1)$ gauge field with the bulk field dual to the $f_2$ meson and extrapolate to all the other fields in the meson trajectories, since it is known that these trajectories are almost degenerate. In~\cite{Katz:2005ir} the dual field of the tensor meson $f_2$ has the same equation of motion as a 5D graviton in $\rm{AdS}_5$ and they predicted correctly the mass of these mesons and the rate of $f_2 \to \gamma \gamma$ after fixing the mass of the $\rho$ meson. These tensor mesons also have the same quantum numbers of the tensor glueballs  $J^{PC} = 2^{++}$ which are believed to be the particles in the hard and soft pomeron trajectories. Hence, we will assume that the coupling of the  $U(1)$ gauge field with the $f_2$ meson is the same as the coupling with the graviton and, in general, the coupling of any bulk spin J field in the meson trajectory is also given by equation~\ref{eq:spin_J_graviton_coupling}. However, the equation of motion of the spin-2 field dual to $f_2$ will be obtained by extrapolation to higher spin J from the equation of motion of the $\rho$.


\bibliographystyle{elsarticle-num}
\bibliography{bib/HVQCD.bib}

\end{document}